\documentclass{HM}
\newcommand\course{HM 2}
\newcommand\hwnumber{3}

\begin{document}
	\begin{enumerate}
		\item[3.3] Berechne die folgenden Integrale:
		\begin{enumerate}
			\item $\int_0^\pi\cos(x)\cdot e^{\sin(x)}dx$
			\begin{align*}
				\intertext{Die Kettenregel rückwärts angewandt ergibt:}
				=&e^{\sin(x)}dx\biggr|_0^\pi\\
				=&0-0\\
				=&0
			\end{align*}
			\item $\int_{-1}^1x^2e^{4x}dx$
			\begin{align*}
				\intertext{dreifache partielle Integration mit $f(x)=x^2, g(x)=e^{4x}$}
				=&x^2\frac{1}{4}e^{4x}\biggr|_{-1}^1 - \int_{-1}^12x\frac{1}{4}e^{4x}dx\\
				=&x^2\frac{1}{4}e^{4x}\biggr|_{-1}^1 - 2x\frac{1}{16}e^{4x}\biggr|_{-1}^1 + \int_{-1}^12\frac{1}{16}e^{4x}dx\\
				=&x^2\frac{1}{4}e^{4x}\biggr|_{-1}^1 - 2x\frac{1}{16}e^{4x}\biggr|_{-1}^1 + 2\frac{1}{64}e^{4x}\biggr|_{-1}^1-0\\
				=&\frac{e^4}{4}-\frac{1}{4e^4}-\frac{2e^4}{16}-\frac{2}{16e^4}+\frac{2e^4}{64}-\frac{2}{64e^4}\\
				=&\frac{16e^8-16-8e^8-8+2e^8-2}{64e^4}	\\
				=&\frac{5e^4}{32}-\frac{13}{32e^4}\\
				=&\approx 8.5235
			\end{align*}
			\item $\int_0^1\ln(x)dx$
			\begin{align*}
				\intertext{partielle Integration mit $f(x)=\ln(x), g(x)=1$:}
				=&\ln(x)x\biggr|_0^1 - \int_0^1\frac{1}{x}xdx\\
				=&\ln(x)x\biggr|_0^1 - \int_0^11dx
				=&0-0-1\\
				=&-1
			\end{align*}
			\item $\int_0^\infty\frac{1}{4\sqrt{x}+\sqrt{x^3}}dx$
		\end{enumerate}
		
		\item[3.4] Nutze das Integralvergleichskriterium zur Entscheidung, ob die folgenden Reihen konvergieren:
		\begin{enumerate}
			\item $\sum\limits_{n=2}^\infty\frac{1}{n\ln(n)}$
			\item $\sum\limits_{n=2}^\infty\frac{1}{n(\ln(n))^2}$
		\end{enumerate}
		
		\item[3.5] Berechne die folgenden Determinanten:
		$$A\coloneqq\begin{vmatrix}
			1&-2&0&5\\
			2&3&1&-2\\
			1&2&0&3\\
			3&1&0&2
		\end{vmatrix},
		B\coloneqq\begin{vmatrix}
			0&1&2&3\\
			-1&0&1&2\\
			-2&-1&0&1\\
			-3&-2&-1&0
		\end{vmatrix}$$
		\begin{enumerate}
		\item[A]:
		\begin{align*}
			\intertext{Laplace'scher Entwicklungssatz der $k=3$. Spalte}
			\det(A)=&\sum\limits_{j=1}^n(-1)^{j+k}a_{jk}\det(A_{jk})\\
			=&-\det(A_{23})\\
			=&-\det\left(\begin{vmatrix}
				1&-2&5\\
				1&2&3\\
				3&1&2
			\end{vmatrix}\right)\\
			\intertext{Regel von Sarrus:}
			=&-(4-18+5-30-3+4)\\
			=&38
		\end{align*}
		\item [B]:
		\begin{align*}
			\det(B)=&0\cdot\det(B_{11})-1\det(B_{12})+2\cdot\det(B_{13})-3\cdot\det(B_{14})\\
			=&-\det\left(\begin{vmatrix}
				-1&1&2\\
				-2&0&1\\
				-3&-1&0
			\end{vmatrix}\right)
			+2\det\left(\begin{vmatrix}
				-1&0&2\\
				-2&-1&1\\
				-3&-2&0
			\end{vmatrix}\right)
			-3\det\left(\begin{vmatrix}
				-1&0&1\\
				-2&-1&0\\
				-3&-2&1
			\end{vmatrix}\right)\\
			\intertext{Regel von Sarrus:}
			=&(-3+4-1)+2(8-6-2)-3(-1+4-3)\\
			=&0
		\end{align*}
		\end{enumerate}
		
		\item[3.6] Eine Matrix $A\in\R^{n\times n}$ heißt nilpotent, falls $A^k=0$ für ein $k\in\N$ ist, und idempotent, falls $A^2=A$ ist. Zeige:
		\begin{enumerate}
			\item $\det(A^k) = (\det A)^k$ für alle $k\in\N$.
			\item $A$ nilpotent $\Rightarrow \det(A)=0$.
			\item $A$ idempotent $\Rightarrow \det(A)\in\{0,1\} ; A$ idempotent und $\det(A)=1\Leftrightarrow A=E_n$.
		\end{enumerate}
	\end{enumerate}
\end{document}
