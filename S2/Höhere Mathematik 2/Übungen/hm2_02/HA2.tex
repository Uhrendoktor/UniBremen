\documentclass{HM}
\newcommand\course{HM 2}
\newcommand\hwnumber{2}

\begin{document}
	\begin{enumerate}
		\item[2.4] Berechne die folgenden Integrale:
		\begin{enumerate}
			\item $\int\limits_{-1}^1(3x^3-2x^2+x-1)dx$
			$$f(x)=3x^3-2x^2+x-1 \Rightarrow F(x)=\frac{3}{4}x^4-\frac{2}{3}x^3+\frac{1}{2}x^2-x+c$$
			$$\Rightarrow\int\limits_{-1}^1(3x^3-2x^2+x-1)dx = F(x)\biggr|_{-1}^1 = F(1)-F(-1) = -\frac{10}{3}$$\\
			
			\item $\int_{-1}^1\frac{1}{1+x^2}dx$
			$$f(x)=\frac{1}{1+x^2} \Rightarrow F(x)=\arctan(x)$$
			$$\Rightarrow\int_{-1}^1\frac{1}{1+x^2}dx=F(x)\biggr|_{-1}^1 = F(1)-F(-1)=\frac{\pi}{2}$$\\
			
			\item $\int_{-1}^2\frac{e^x-1}{e^x+1}dx$
			\begin{align*}
				&\int_{-1}^2\frac{e^x-1}{e^x+1}dx\\
				=&\int_{-1}^2\frac{e^x}{e^x+1}dx-\int_{-1}^2\frac{1}{e^x+1}dx\\
			\intertext{Mit $g(x)=e^x+1\Rightarrow g'(x)=e^x$ folgt aus $\int_a^b\frac{g'(x)}{g(x)}dx = \ln(|g(x)|)\biggr|_a^b$:}
				=&\ln(e^x+1)\biggr|_{-1}^2-\int_{-1}^2\frac{1}{e^x+1}dx\\
			\intertext{Mit $s\coloneqq e^x$ folgt aus der Substitutionsregel:}
				=&\ln(e^x+1)\biggr|_{-1}^2-\int_{e^{-1}}^{e^2}\frac{1}{s+1}\frac{ds}{s}\\
				=&\ln(e^x+1)\biggr|_{-1}^2-\int_{e^{-1}}^{e^2}\frac{1}{s(s+1)}ds\\
			\intertext{Aus der Partialbruchzerlegung von $\frac{1}{s(s+1)}$ zu $\frac{1}{2}-\frac{1}{s+1}$ folgt:}
				=&\ln(e^x+1)\biggr|_{-1}^2-\int_{e^{-1}}^{e^2}\frac{1}{s}ds+\int_{e^{-1}}^{e^2}\frac{1}{s+1}ds
			\intertext{Mit $u\coloneqq s+1$ folgt aus der Substitutionsregel:}
				=&\ln(e^x+1)\biggr|_{-1}^2-\int_{e^{-1}}^{e^2}\frac{1}{s}ds+\int_{e^{-1}+1}^{e^2+1}\frac{1}{u}du\\
				=&\ln(e^x+1)\biggr|_{-1}^2-\ln(s)\biggr|_{e^{-1}}^{e^2}+\ln(u)\biggr|_{e^{-1}+1}^{e^2+1}\\
				=&-2\ln(e^2+1)+2\ln(e+1)-1
			\end{align*}
			$\Rightarrow \int_{-1}^2\frac{e^x-1}{e^x+1}dx=-2\ln(e^2+1)+2\ln(e+1)-1$\\
			
			\item $\int_0^\frac{1}{2}\frac{x}{\sqrt{1-x^2}}dx$\\
			\begin{align*}
			\intertext{partielle Integration mit $f_1(x)=x, g_2(x)=\frac{1}{\sqrt{1-x^2}}$:}
				\Rightarrow &\int_0^\frac{1}{2}\frac{x}{\sqrt{1-x^2}}dx\\
				\rightarrow&x\arcsin(x)\biggr|_0^{\frac{1}{2}} - \int_0^{\frac{1}{2}}1\cdot\arcsin(x)dx
			\intertext{Substitution mit $v\coloneqq \arcsin(x)$}
				\rightarrow&x\arcsin(x)\biggr|_0^{\frac{1}{2}}-\int_0^{\arcsin (\frac{1}{2})}v\cos(v)dv		
			\intertext{partielle Integration mit $f_2(v)=v, g_2(v)=\cos(v)$}
				=&x\arcsin(x)\biggr|_0^{\frac{1}{2}}-v\sin(v)\biggr|_0^{\arcsin(\frac{1}{2})}+\int_0^{\arcsin(\frac{1}{2})}1\cdot\sin(v)dv\\
				=&x\arcsin(x)\biggr|_0^{\frac{1}{2}}-v\sin(v)\biggr|_0^{\arcsin(\frac{1}{2})}-\cos(v)\biggr|_0^{\arcsin(\frac{1}{2})}\\
				=&\tfrac{1}{2}\arcsin(\tfrac{1}{2})-0-\arcsin(\tfrac{1}{2})\tfrac{1}{2}+0-\cos(\arcsin(\tfrac{1}{2}))+1\\
				=&-\cos(\arcsin(\tfrac{1}{2}))+1\\
				=&\frac{2-\sqrt{3}}{2} \approx 0.133975
			\end{align*}	
		\end{enumerate}
		\item[2.5] Bestimme mittels geeigneter Integrationstechniken Stammfunktionen zu folgenden Funktionen:
		\begin{enumerate}
			\item $f(x)=3e^x\sqrt{e^x+1}$
			\begin{align*}
			&\int3e^x\sqrt{e^x+1}dx\\
			\intertext{Substituition mit $s\coloneqq e^x\Rightarrow dx=\frac{1}{s}$}
			=&\int3\sqrt{s+1}ds\\
			\intertext{partielle Integration mit $f(s)=3, g(s)=\sqrt{s+1}$}
			=&3G(s)-0\\
			=&3\int\sqrt{s+1}ds\\
			\intertext{Substitution mit $u\coloneqq\sqrt{s+1}\Rightarrow ds=2u$}
			=&3\int\sqrt{2u^2}du\\
			=&3\frac{2}{3}u^3\\
			\intertext{$u=\sqrt{s+1}=\sqrt{e^x+1}$}
			=&2\sqrt{e^x+1}^3
			\end{align*}
			\item $f(x)= x\ln(x)\quad(x>0)$
			\begin{align*}
			&\int x\ln(x)dx\\
			\intertext{partielle Integration mit $f(x)=x, g(x)=\ln(x)$}
			=&\tfrac{1}{2}\ln(x)x^2-\int\tfrac{1}{2}xdx\\
			=&\tfrac{1}{2}\ln(x)x^2-\tfrac{1}{4}x^2\\
			=&\tfrac{1}{2}x^2(\ln(x)-\tfrac{1}{2})
			\end{align*}
			\item $f(x)=\frac{1}{\sqrt{x}(1+\sqrt[3]{x})} \quad(x>0)$
			\begin{align*}
			&\int\frac{1}{\sqrt{x}(1+\sqrt[3]{x})}dx\\
			=&\int\frac{1}{x^{\frac{1}{2}}+x^{\frac{1}{2}+\frac{1}{3}}}dx\\
			=&\int\frac{1}{x^{\frac{1}{6}^3}+x^{\frac{1}{6}^5}}dx\\
			\intertext{Substitution mit $s\coloneqq x^{\frac{1}{6}}\Rightarrow dx=6s^5$}
			=&6\int\frac{s^5}{s^3+s^5}ds\\
			=&6\int\frac{s^2}{1+s^2}ds\\
			=&6\left(\int ds-\int\frac{1}{1+s^2}ds\right)\\
			=&6(s-\arctan(s))\\
			\intertext{$s=x^{\frac{1}{6}}$}
			=&6(\sqrt[6]{x}-\arctan(\sqrt[6]{x}))
			\end{align*}
			\item $f(x)=\frac{x^2+9x+17}{x^3-3x^2-4}$
			\begin{align*}
			&\int\frac{x^2+9x+17}{x^3-3x^2-4}dx\\
			\intertext{Partialbruchzerlegung zu $A=3, B=-2, C=-1$}
			=&\int\frac{3}{x-1}-\frac{2}{x+2}-\frac{1}{(x+2)^2}dx\\
			=&3\int\frac{1}{x-1}dx-2\int\frac{1}{x+2}dx-\int\frac{1}{(x+2)^2}dx\\
			\intertext{Substitution mit $s_1\coloneqq x-1, s_2\coloneqq x+2, s_3\coloneqq (x+2)^2 \Rightarrow dx=\frac{1}{2\sqrt{s_3}}$}
			=&3\ln(s_1)-2\ln(s_2)-\int\frac{1}{2}s_3^{-\frac{3}{2}}ds_3\\
			\intertext{$s_1=x-1,s_2=x+2$}
			=&3\ln(x-1)-2\ln(x+2)+s_3^{-\frac{1}{2}}\\
			\intertext{$s_3=(x+2)^2$}
			=&3\ln(x-1)-2\ln(x+2)+\frac{1}{x+2}
			\end{align*}
		\end{enumerate}
		\item [2.6] Die Graphen der Funktionen $f_1,f_2,g_1,g_2:(0,\infty)\to(0,\infty)$
		$$f_1(x)\coloneqq x^2, f_2(x)\coloneqq 2x^2, g_1(x)\coloneqq\frac{1}{x},g_2(x)\coloneqq\frac{4}{x}$$
		begrenzen eine Fläche im $\R^2$. Berechne den Flächeninhalt.
		\begin{align*}
		w_1 &\Leftarrow f_2=g_1 \Leftrightarrow 2x^2=\frac{1}{x}&\Leftrightarrow x=&\sqrt[3]{\frac{1}{2}}\\
		w_2 &\Leftarrow f_1=g_1 \Leftrightarrow x^2=\frac{1}{x}&\Leftrightarrow x=&1\\
		w_3 &\Leftarrow f_2=g_2 \Leftrightarrow 2x^2=\frac{4}{x}&\Leftrightarrow x=&\sqrt[3]{2}\\
		w_4 &\Leftarrow f_1=g_2\Leftrightarrow x^2=\frac{4}{x}&\Leftrightarrow x=&\sqrt[3]{4}
		\end{align*}
		\begin{align*}
		A=&\int_{w_1}^{w_2}f_2(x)-g_1(x)dx+\int_{w_2}^{w_3}f_2(x)-f_1(x)dx+\int_{w_3}^{w_4}g_2(x)-f_1(x)dx\\
		=&\int_{\sqrt[3]{\tfrac{1}{2}}}^{1}2x^2-\frac{1}{x}dx+\int_{1}^{\sqrt[3]{2}}2x^2-x^2dx+\int_{\sqrt[3]{2}}^{\sqrt[3]{4}}\frac{4}{x}-x^2dx\\
		=&\left(\frac{2}{3}x^3-\ln(x)\right)\biggr|_{\sqrt[3]{\tfrac{1}{2}}}^1+\left(\frac{1}{3}x^3\right)\biggr|_1^{\sqrt[3]{2}}+\left(4\ln(x)-\frac{1}{3}x^3\right)\biggr|_{\sqrt[3]{2}}^{\sqrt[3]{4}}\\
		=&\frac{1}{3}-\frac{1}{3}\ln(2)+\frac{1}{3}-\frac{2}{3}+\frac{4}{3}\ln(2)\\
		=&\ln(2)
		\end{align*}
		
		\item[2.7]
		\begin{enumerate}
			\item Seien $f,\varphi,\psi:\R\to\R$ differenzierbar. Berechne $\frac{d}{dr}\int_{\varphi(r)}^{\psi(r)}f(x)dx$\\\\
			$f$ differenzierbar\\
			$\Rightarrow f$ stetig auf dem gesamten Definitionsbereich\\
			$\Rightarrow f$ weist keine Sprünge auf.\\
			$\Rightarrow f$ besitzt eine kontinuierliche Veränderung der Fläche die sie begrenzt.\\
			$\Rightarrow f$ ist integrierbar.\\
			$\Rightarrow f$ besitzt eine Stammfunktion
			\begin{align*}
			&\frac{d}{dr}\int_{\varphi(r)}^{\psi(r)}f(x)dx\\
			=&\frac{d}{dr}(F(\psi(r)))-F(\varphi(r)))\\
			\intertext{Kettenregel:}
			=&f(\psi(r))\psi'(r)-f(\varphi(r))\varphi'(r)
			\end{align*}
			\item Berechne $\frac{d}{dr}\int_{\sqrt{\ln(r)}}^{2\sqrt{\ln(r)}}e^{x^2}dx$
			\begin{align*}
			&\frac{d}{dr}\int_{\sqrt{\ln(r)}}^{2\sqrt{\ln(r)}}e^{x^2}dx\\
			=&e^{(2\sqrt{\ln(r))}^2}\frac{1}{r\sqrt{\ln(r)}}-e^{\sqrt{\ln(r)}^2}\frac{1}{2r\sqrt{r}}\\
			=&e^{4\ln(r)}\frac{1}{r\sqrt{\ln(r)}}-e^{\ln(r)}\frac{1}{2r}\frac{1}{\sqrt{\ln(r)}}\\
			=&\frac{1}{\sqrt{\ln(r)}}(r^3-\frac{1}{2})
			\end{align*}
		\end{enumerate}
		\item [2.8] Ermittle den Grenzwert
		\begin{align*}
		&\lim_{n\to\infty}n\cdot\left(\frac{1}{n^2}+\frac{1}{(n+1)^2}+\hdots+\frac{1}{(2n-1)^2}\right)\\
		=&\lim_{n\to\infty}\frac{1}{n}\left(\frac{n^2}{n^2}+\frac{n^2}{(n+1)^2}+\hdots+\frac{n^2}{(2n-1)^2}\right)\\
		=&\lim_{n\to\infty}\frac{1}{n}\left(\frac{1}{(1+\tfrac{0}{n})^2}+\frac{1}{(1+\tfrac{1}{n})^2}+\hdots+\frac{1}{(1+\tfrac{n-1}{n})^2}\right)\\
		=&\lim_{n\to\infty}\frac{1}{n}\sum\limits_{k=0}^{n-1}\frac{1}{(1+\tfrac{k}{n})^2}\\
		\intertext{$x=\frac{k}{n}; 0\leq x < 1$:}
		=&\int_0^1\frac{1}{(1+x)^2}dx\\
		\intertext{$a=x+1; 1\leq a<2$:}
		=&\int_1^2\frac{1}{a^2}da\\
		\intertext{$f(x)=\frac{1}{a^2}\Rightarrow F(x)=-\frac{1}{a}$:}
		=&F(2)-F(1)\\
		=&-\frac{1}{2}+1\\
		=&\frac{1}{2}
		\end{align*}
	\end{enumerate}
\end{document}
