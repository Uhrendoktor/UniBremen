\documentclass{HM}
\begin{document}
\begin{enumerate}
\item [5.2] Entscheide (mit Begründung!), ob die folgenden 4 Abbildungen linear sind:
\begin{enumerate}
\item $\varphi\colon\R^2\to\R^3, \begin{pmatrix}
x\\
y
\end{pmatrix}\mapsto\begin{pmatrix}
x+1\\
2y\\
x+y
\end{pmatrix};$\\
Bedingung 1:
\begin{align*}
\eqnf{\varphi(a+b)}{\varphi(a)+\varphi(b)}
\eqn{\varphi\left(\begin{pmatrix}
x\\
y
\end{pmatrix}+\begin{pmatrix}
x\\
y
\end{pmatrix}\right)}{\varphi\left(\begin{pmatrix}
x\\
y
\end{pmatrix}\right)+\varphi\left(\begin{pmatrix}
x\\
y
\end{pmatrix}\right)}
\eqn{\begin{pmatrix}
2x\\
2y
\end{pmatrix}}{\begin{pmatrix}
x+1\\
2y\\
x+y
\end{pmatrix}+\begin{pmatrix}
x+1\\
2y\\
x+y
\end{pmatrix}}
\eqni{\begin{pmatrix}
2x+1\\
4y\\
2x+2y
\end{pmatrix}}{\begin{pmatrix}
2x+2\\
4y\\
2x+2y
\end{pmatrix}}{\times}
\end{align*}\\
$\Rightarrow 2x+1\neq 2x+2 \Rightarrow\varphi$ ist keine lineare Abbildung, da eine der Bedingungen für lineare Abbildungen nicht erfüllt werden kann.\\
\item $\varphi\colon\R^n\to\R, \begin{pmatrix}
x_1\\
\vdots\\
x_n
\end{pmatrix}\mapsto\sum_{k=1}^n |x_k|;$\\
Bedingung 1:
\begin{align*}
\eqnf{\varphi(a+b)}{\varphi(a)+\varphi(b)}
\eqn{\varphi\left(\begin{pmatrix}
x_1\\
\vdots\\
x_n
\end{pmatrix}\right)+\varphi\left(\begin{pmatrix}
x_1\\
\vdots\\
x_n
\end{pmatrix}\right)}{\varphi\left(\begin{pmatrix}
x_1\\
\vdots\\
x_n
\end{pmatrix}\right)+\varphi\left(\begin{pmatrix}
x_1\\
\vdots\\
x_n
\end{pmatrix}\right)}
\eqn{\varphi\left(\begin{pmatrix}
2x_1\\
\vdots\\
2x_n
\end{pmatrix}\right)}{\sum_{k=1}^n|x_k|+\sum_{k=1}^n|x_k|}
\eqn{\sum_{k=1}^n|2x_k|}{\sum_{k=1}^n2|x_k|}
\eqni{\sum_{k=1}^n|2x_k|}{\sum_{k=1}^n|2x_k|}{\surd}
\end{align*}
Bedingung 2:
\begin{align*}
\eqnf{\varphi(\lambda a)}{\lambda\varphi(a)}
\eqn{\varphi\left(\lambda\begin{pmatrix}
x_1\\
\vdots\\
x_n
\end{pmatrix}\right)}{\lambda\varphi\left(\begin{pmatrix}
x_1\\
\vdots\\
x_n
\end{pmatrix}\right)}
\eqn{\varphi\left(\begin{pmatrix}
\lambda x_1\\
\vdots\\
\lambda x_n
\end{pmatrix}\right)}{\lambda\sum_{k=1}^n |x_k|}
\eqni{\sum_{k=1}^n |\lambda x_k|}{\sum_{k=1}^n \lambda|x_k|}{\times}
\end{align*}
Da $\lambda|x_k|$ für $\lambda\in\R$ positive und negative Werte annehmen kann, $|\lambda x_k|$ jedoch stets positiv ist, folgt $|\lambda x_k|\neq \lambda|x_k|$. Somit erfüllt $\varphi$ die 2. Bedingung, also die Kriterien für lineare Abbildungen nicht. $\Rightarrow\varphi$ ist keine lineare Abbildung.\\
\item $\varphi\colon\R^n\to\R, \begin{pmatrix}
x_1\\
\vdots\\
x_n
\end{pmatrix}\mapsto\sum_{k=1}^n x_k;$\\
Bedingung 1:
\begin{align*}
\eqnf{\varphi(a+b)}{\varphi(a)+\varphi(b)}
\eqn{\varphi\left(\begin{pmatrix}
x_1\\
\vdots\\
x_n
\end{pmatrix}+\begin{pmatrix}
x_1\\
\vdots\\
x_n
\end{pmatrix}\right)}{\varphi\left(\begin{pmatrix}
x_1\\
\vdots\\
x_n
\end{pmatrix}\right)+\varphi\left(\begin{pmatrix}
x_1\\
\vdots\\
x_n
\end{pmatrix}\right)}
\eqn{\varphi\left(\begin{pmatrix}
2x_1\\
\vdots\\
2x_n
\end{pmatrix}\right)}{\sum_{k=1}^n x_k + \sum_{k=1}^n x_k}
\eqni{\sum_{k=1}^n 2x_k}{\sum_{k=1}^n 2x_k}{\surd}
\end{align*}
Bedingung 2:
\begin{align*}
\eqnf{\varphi(\lambda a)}{\lambda\varphi(a)}
\eqn{\varphi\left(\lambda\begin{pmatrix}
x_1\\
\vdots\\
x_n
\end{pmatrix}\right)}{\lambda\varphi\left(\begin{pmatrix}
x_1\\
\vdots\\
x_n
\end{pmatrix}\right)}
\eqn{\varphi\left(\begin{pmatrix}
\lambda x_1\\
\vdots\\
\lambda x_n
\end{pmatrix}\right)}{\lambda\sum_{k=1}^n x_k}
\eqni{\sum_{k=1}^n \lambda x_k}{\sum_{k=1}^n \lambda x_k}{\surd}
\end{align*}
Da beide Bedingungen erfüllt sind, ist $\varphi$ eine lineare Abbildung.\\
\item $\varphi\colon V\to\R, f\mapsto f(x_0)$, wobei $V$ der Vektorraum aller Funktionen von $\R$ nach $\R$ ist, und $x_0\in\R$.\\
Bedingung 1:
\begin{align*}
\eqnf{\varphi(a+b)}{\varphi(a)+\varphi(b)}
\eqn{\varphi(f+f)}{\varphi(f)+\varphi(f)}
\eqn{(2f)(x_0)}{f(x_0)+f(x_0)}
\eqni{2f(x_0)}{2f(x_0)}{\surd}
\end{align*}
Bedingung 2:
\begin{align*}
\eqnf{\varphi(\lambda a)}{\lambda\varphi(a)}
\eqn{\varphi(\lambda f)}{\lambda\varphi(f)}
\eqn{(\lambda f)(x_0)}{\lambda f(x_0)}
\eqni{\lambda f(x_0))}{\lambda f(x_0)}{\surd}
\end{align*}
Da beide Bedingungen erfüllt sind, ist $\varphi$ eine lineare Abbildung.\\
\end{enumerate}
\item [5.3] Begründe jeweils, warum $V$ kein Vektorraum ist:
\begin{enumerate}
\item $V\coloneqq\R^2$ mit $\begin{pmatrix}
x_1\\
x_2
\end{pmatrix}+\begin{pmatrix}
y_1\\
y_2
\end{pmatrix}\coloneqq\begin{pmatrix}
x_1+y_1\\
x_2+y_2
\end{pmatrix}$ und $\lambda\begin{pmatrix}
x_1\\
x_2
\end{pmatrix}\coloneqq\begin{pmatrix}
\lambda x_1\\
x_2
\end{pmatrix}$.
\item $V\coloneqq\R^2$ mit $\begin{pmatrix}
x_1\\
x_2
\end{pmatrix}+\begin{pmatrix}
y_1\\
y_2
\end{pmatrix}\coloneqq\begin{pmatrix}
x_1+y_2\\
x_2+y_1
\end{pmatrix}$ und $\lambda\begin{pmatrix}
x_1\\
x_2
\end{pmatrix}\coloneqq\begin{pmatrix}
\lambda x_1\\
\lambda x_2
\end{pmatrix}$.
\item $V\coloneqq\R^2$ mit $\begin{pmatrix}
x_1\\
x_2
\end{pmatrix}+\begin{pmatrix}
y_1\\
y_2
\end{pmatrix}\coloneqq\begin{pmatrix}
x_1+y_1\\
x_2+y_2
\end{pmatrix}$ und $\lambda\begin{pmatrix}
x_1\\
x_2
\end{pmatrix}\coloneqq\begin{pmatrix}
\lambda x_1\\
0
\end{pmatrix}$.
\item $V\coloneqq\lbrace\begin{pmatrix}
x\\
y
\end{pmatrix}\in\R^2; x=y^2 \rbrace$ mit $\begin{pmatrix}
x_1\\
y_1
\end{pmatrix}+\begin{pmatrix}
x_2\\
y_2
\end{pmatrix}\coloneqq\begin{pmatrix}
x_1+x_2\\
y_1+y_2
\end{pmatrix}$ und $\lambda\begin{pmatrix}
x\\
y
\end{pmatrix}\coloneqq\begin{pmatrix}
\lambda x\\
\lambda y
\end{pmatrix}$.
\end{enumerate}
\item [5.4] Sei $\varphi\colon\R^2\to\R^3$ eine lineare Abbildung mit
$$\varphi\begin{pmatrix}
1\\
1
\end{pmatrix}=\begin{pmatrix}
1\\
0\\
-2
\end{pmatrix},
\varphi\begin{pmatrix}
1\\
2
\end{pmatrix}=\begin{pmatrix}
0\\
1\\
-1
\end{pmatrix}.$$
Durch welche Matrix $A\in\R^{m\times n}$ ist die Abbildung $\varphi$ gegeben? Berechne $\varphi\begin{pmatrix}
7\\
12
\end{pmatrix}$.\\
$$A\coloneqq\begin{pmatrix}
x_1&y_1\\
x_2&y_2\\
x_3&y_3
\end{pmatrix}$$
$$\varphi(x)=Ax
\Rightarrow \begin{pmatrix}
1\\
0\\
-2
\end{pmatrix}=A\begin{pmatrix}
1\\
1
\end{pmatrix},
\begin{pmatrix}
0\\
1\\
-1
\end{pmatrix}=A\begin{pmatrix}
1\\
2
\end{pmatrix}$$\\
\begin{align*}
&x_1+y_1=&1\\
&x_1+2y_1=&0\\
&x_2+y_2=&0\\
&x_2+2y_2=&1\\
&x_3+y_3=&-2\\
&x_3+2y_3=&-1\\
\end{align*}
$$\Rightarrow A=\begin{pmatrix}
2&-1\\
-1&1\\
-3&1
\end{pmatrix}$$
$$\varphi(A)=\begin{pmatrix}
2&-1\\
-1&1\\
-3&1
\end{pmatrix}\begin{pmatrix}
7\\
12
\end{pmatrix}=\begin{pmatrix}
2\\
5\\
-9
\end{pmatrix}$$
\item [5.5] Sei $V$ ein Vektorraum, und $v^1,v^2,v^3,v^4$ seien linear unabhängige Vektoren in $V$. Ermittle in jedem der folgenden 3 Fälle, ob die gegebenen Vektoren linear unabhängig sind:
\begin{enumerate}
\item $v^1, v^1+v^2, v^1+v^2+v^3, v^1+v^2+v^3+v^4$,
\item $v^1-v^2, v^2+v^3, v^3-v^4, v^4+v^1$,
\item $v^1+v^2, v^2+v^3, v^3+v^4, v^4-v^1$.
\end{enumerate}
\end{enumerate}
\end{document}