\documentclass{HM}
\usepackage{listings}
\begin{document}
\begin{enumerate}
\item[8.3] Untersuche die Reihe $\sum\limits_{n=1}^\infty a_n$ auf Konvergenz und absolute Konvergenz, wobei\\
\begin{enumerate}

\item $a_n=\left(\frac{1+i}{2}\right)^n$;
\begin{align*}
 	&\sum\limits_{n=1}^\infty\left(\frac{1+i}{2}\right)^n\\
	=&\sum\limits_{n=0}^\infty\left(\frac{1+i}{2}\right)^n-1
\end{align*}
Geometrische Reihe mit $|\frac{1+i}{2}| = \frac{|1+i|}{|2|} = \frac{\sqrt{2}}{2} < 1$:\\
$\Rightarrow$ konvergent.\\\\
Überprüfen absoluter Konvergenz:\\
\begin{align*}-
	&\sum\limits_{n=1}^\infty\left|\left(\frac{1+i}{2}\right)^n\right|\\
	=&\sum\limits_{n=1}^\infty\left|\frac{1+i}{2}\right|^n\\
	=&\sum\limits_{n=1}^\infty\left(\frac{\sqrt{2}}{2}\right)^n\\
	=&\sum\limits_{n=0}^\infty\left(\frac{\sqrt{2}}{2}\right)^n-1\\
\end{align*}
Aus der geometrischen Reihe mit $|\frac{\sqrt{2}}{2}| < 1$ folgt $\sum\limits_{n=1}^\infty\left|\left(\frac{1+i}{2}\right)^n\right|$ konvergiert.\\
$\Rightarrow$ absolut konvergent\\

\item $a_n=\frac{(-1)^n}{\sqrt[3]{n}}=(-1)^n\frac{1}{\sqrt[3]{n}}$;\\

Nach dem Leipniz-Kriterium, konvergiert eine Reihe, wenn die Folge $(b_n)\coloneqq \frac{1}{\sqrt[3]{n}}$ eine monoton fallende Nullfolge ist.
Es gilt: $a>b\Rightarrow\sqrt{a}>\sqrt{b}$.\\
$\Rightarrow \sqrt[3]{n}\xrightarrow{n\to\infty}\infty\Rightarrow\frac{1}{\sqrt[3]{n}}\xrightarrow{n\to\infty}0$ (monoton fallend)\\
$\Rightarrow$ konvergent.\\\\
Absolute Konvergenz:\\
\begin{align*}
	&\left|\frac{(-1)^n}{\sqrt[3]{n}}\right|\\
	=&\frac{\left|(-1)^n\right|}{\left|\sqrt[3]{n}\right|}\\
	=&\frac{1}{\sqrt[3]{n}}\\
	=&\sqrt[n]{\frac{1}{\sqrt[3]{n}}}^n\\
	=&\frac{1}{\sqrt[3n]{n}}^n
\end{align*}
Geometrische Reihe mit $\frac{1}{\sqrt[3]{n}}<1$:\\
\begin{align*}
	&\frac{1}{1-\frac{1}{\sqrt[3n]{n}}}\\
	=&\frac{1}{\frac{\sqrt[3n]{n}-1}{\sqrt[3n]{n}}}\\
	=&\frac{\sqrt[3n]{n}}{\sqrt[3n]{n}-1}
\end{align*}
\begin{align*}
	\sqrt[3n]{n}&\xrightarrow{n\to\infty}1\\
	\Rightarrow\sqrt[3n]{n}-1&\xrightarrow{n\to\infty}0\\
	\Rightarrow\frac{\sqrt[3n]{n}}{\sqrt[3n]{n}-1}&\xrightarrow{n\to\infty}\infty
\end{align*}
$\Rightarrow$ divergent $\Rightarrow$ nicht absolut konvergent.\\
(Umrechnungen von $\sum\limits_{n=1}^\infty$ zu $\sum\limits_{n=0}^\infty$ mit $n+1$ wurden weggelassen, weil sie durch den allgemeineren Fall von $n$ gezeigt wurden).

\item $a_n=(-1)^n\frac{n+2}{2n}$.\\

Nach dem Leipniz-Kriterium, konvergiert eine Reihe, wenn die Folge $(b_n)\coloneqq \frac{n+2}{2n}$ für $n\geq 1$ eine monoton fallende Nullfolge ist.\\
\begin{align*}
	\geqnfi{\frac{n+2}{2n}}{>}{\frac{n+3}{2n+2}}{\cdot(2n+2)}
	\geqn{\frac{(n+2)(2n+2)}{2n}}{>}{n+3}
	\geqni{\frac{2n^2+6n+4}{2n}}{>}{n+3}{-(n+3)}
	\geqn{\frac{2n^2+6n+4}{2n}-\frac{2n(n+3)}{2n}}{>}{0}
	\geqn{\frac{2n^2+6n+4}{2n}-\frac{2n^2+6n}{2n}}{>}{0}
	\geqni{\frac{4}{2n}=\frac{2}{n}}{>}{0}{\surd \text{ für alle } n\geq 1}
\end{align*}
Die Folge ist für alle Elemente, die in der Reihe vertreten sind ($n\geq 1$) monoton fallend.\\\\
Grenzwert \textit{(siehe Umformung Ungleichung zu $\frac{2}{n}>0$)}:\\
$$a_n=a_{n+1}\Leftrightarrow a=\frac{1}{2}$$
\begin{tabular}{rl}
	$\frac{1}{2}>0\Rightarrow$  &keine Nullfolge.\\
	$\Rightarrow$ &nicht konvergent $\Rightarrow$ nicht absolut konvergent\\
	$\Rightarrow$ &divergent.
\end{tabular}

\end{enumerate}
\item[8.4] Sei $s_n\coloneqq \sum\limits_{k=0}^n\frac{1}{k!}$ und $e=\lim\limits_{n\to\infty}s_n$ die Euler'sche Zahl.
\begin{enumerate}
\item Zeige die Ungleichungen $0<e-s_n<\frac{1}{n*n!}$ für $n\in\N$, mit Hilfe einer geeigneten geometrischen Reihe.\\
\begin{align*}
	\geqnf{0}{<}{e-s_n}
	\geqn{0}{<}{\sum\limits_{k=0}^\infty\frac{1}{k!}-\sum\limits_{k=0}^n\frac{1}{k!}}
	\geqn{0}{<}{\sum\limits_{k=n+1}^\infty\frac{1}{k!}}
\end{align*}
Da alle Summanden $>0$, muss $e-s_n>0$ sein.\\
$e-s_n$ wird größer, je kleinere Werte $s_n$ annimmt. Da es sich bei $s_n$ um eine Summe mit rein positiven Summanden handelt, ist $s_n$ am kleinsten, wenn es die wenigstens möglichen Summanden besitzt ($n=1$).\\
$\Rightarrow \min s_n = \sum\limits_{k=0}^1\frac{1}{k!}=2$.
\begin{align*}
	\geqnf{e-s_n}{<}{\frac{1}{n\cdot n!}}
	\geqn{e-2}{<}{1}
\end{align*}
\item Bestimme mit Hilfe von (a) eine Zahl $n\in\N$, für die $|e-s_N|\leq 0.5\cdot 10^{-4}$ gilt, und gib den Wert von $s_N$ an.
\item Zeige, dass die Euler'sche Zahl $e$ irrational ist.
\end{enumerate}
\item[8.5]
\begin{enumerate}
\item Konvergiert die Reihe $\sum\limits_{n=2}^\infty\left(\frac{1}{\sqrt{n}-1}+\frac{1}{\sqrt{n}+1}\right)$? Majoranten
\item Berechne $\sum\limits_{n=1}^\infty\frac{n-1}{n!}$.\\
\begin{align*}
	\eqnf{A}{\sum_{n=1}^{\infty}\frac{n-1}{n!}}
	\eqn{A}{\sum_{n=1}^{\infty}\left(\frac{n}{n!}-\frac{1}{n!}\right)}
	\frac{n}{n!}&=\frac{1}{(n-1)!}\\
	\Rightarrow A&=\sum_{n=1}^{\infty}\left(\frac{1}{(n-1)!}-\frac{1}{!n}\right)\\
	\eqn{A}{1-1+1-\frac{1}{2!}+\frac{1}{2!}-\frac{1}{3!}+\frac{1}{3!}-\frac{1}{4!}+\dots-\frac{1}{n!}}
	\eqn{A}{1-\frac{1}{n!}}
	\lim_{n\to \infty}\frac{1}{n!} &= 0\\
	\Rightarrow A &= 1\\
\end{align*}
\item Für welche $z\in\C\setminus\lbrace-1\rbrace$ konvergiert die Reihe $\sum\limits_{n=1}^\infty\frac{z^{n-1}}{(1+z)^n}?$\\
Bestimme den Grenzwert, falls er existiert.\\\\
Quotientenkriterium:
\begin{align*}
	&\left|\frac{z^n (1+z)^n}{(1+z)^{n+1}z^{n-1}}\right|\\
	\Leftrightarrow &\left|\frac{z}{1+z}\right|
\end{align*}
Da $|z|<|1+z|$ muss $\frac{|z|}{|1+z|} < 1$ und damit wissen wir dass:
$\sum\limits_{n=1}^\infty\frac{z^{n-1}}{(1+z)^n}$ für all $z$ konvergiert\\
\begin{align*}
	\sum\limits_{n=1}^\infty\frac{z^{n-1}}{(1+z)^n}\\
	\Leftrightarrow \frac{1}{z}\sum\limits_{n=1}^\infty\left(\frac{z}{1+z}\right)^n\\
	\text{da wir schon wissen dass }\left|\frac{z}{1+z}\right|<1\\
	\text{können wir die geometrische Reihe verwenden}\\
	\eqnf{\frac{1}{z}\sum\limits_{n=1}^\infty\left(\frac{z}{1+z}\right)^n}{\frac{1}{z}\left(\sum\limits_{n=0}^\infty\left(\frac{z}{1+z}\right)^n-1\right)}
	\eqn{\frac{1}{z}\sum\limits_{n=1}^\infty\left(\frac{z}{1+z}\right)^n }{\frac{1}{z}\left(\frac{1}{1-\frac{z}{1+z}}-1\right)}
	\eqn{\sum\limits_{n=1}^\infty\frac{z^{n-1}}{(1+z)^n}}{1}
\end{align*}

\end{enumerate}
\item[8.6] Ermittle (durch Probieren) das kleinste $n\in\N$, für dass $\sum\limits_{k=1}^n\frac{1}{k}>3$ ist. benutze einen Computer, um herauszufinden, wie groß man $n$ wählen muss, damit die Summe $>6$ bzw. $>9$ wird.\\
\begin{minipage}{\textwidth /2}
	\begin{scriptsize}
		\lstinputlisting[language=Python,firstline=8]{8.6.py}
	\end{scriptsize}
\end{minipage}
\begin{minipage}{\textwidth /2}
	\begin{align*}
		\sum_{k=1}^{11}&=3.0198773448773446\\
		\sum_{k=1}^{227}&=6.004366708345567\\
		\sum_{k=1}^{4550}&=9.000208062931115\\	
		\mathbb{L}&=\{11,227,4550\}
	\end{align*}
	
\end{minipage}

\end{enumerate}
\end{document}