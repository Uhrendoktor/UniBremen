\documentclass{../HM}
\newcommand\course{HM 2}
\newcommand\hwnumber{10}
\usepackage{gauss}
\usepackage{tikz}
\usepackage{pgfplots}

\newcommand{\mue}{m_{\textit{ü}}}
\begin{document}
	\begin{enumerate}
		\item[10.2] Eine (idealisiert homogene) Kette der Länge $l$ liegt auf einem horizontalen, reibungsfreien Tisch, wobei ein Stück Kette der Länge $a$ über den Rand hängt. Wie lange dauert es, bis die kette vom Tisch geglitten ist?\\\\
		Hinweis: Bezeichne die Gesamtmasse der Kette mit $m$ und die Länge des überhängenden Stücks zur Zeit $t$ mit $y(t)$. Berechne die Masse $\mue(t)$ des überhängenden Stücks zur Zeit $t$. Die Gravitationskraft $\mue(t)g$ muss dann gleich $F=m\ddot{y}(t)$ sein.
		
		\begin{eqnn}
			\eqnf{\mue(t)}{y(t)\frac{m}{l}}
			\eqnf{F}{m\ddot{y}(t)}
			\eqn{F}{\mue(t)g}
			\geqn[][\Rightarrow]{m\ddot{y}(t)}{\feq}{\mue(t)g}
			\eqn{\ddot{y}(t)-y(t)\frac{g}{l}}{0}
			\eqntext{Homogene Differentialgleichung lösen:}
			\eqn[][\Rightarrow]{p(\lambda)}{\lambda^2-\frac{g}{l}=0}
			\eqn[][\Rightarrow]{\lambda}{\pm\sqrt{\frac{g}{l}}}
			\eqn[][\Rightarrow]{y(t)}{c_1e^{t}\sqrt{\frac{g}{l}}+c_2e^{-t\sqrt{\frac{g}{l}}}}
			\eqntext{Bedingung $y(0)=a$ einsetzen, da die Länge des überhängenden Stücks der Kette zum Zeitpunkt $t=0\to a$ ist:}
			\eqn[][\Rightarrow]{c_1+c_2}{a}
			\eqn{c_1}{a-c_2}
			\eqn[][\Rightarrow]{y(t)}{(a-c_2)e^{t}\sqrt{\frac{g}{l}}+c_2e^{-t\sqrt{\frac{g}{l}}}}
			\eqnspace
			\eqntext{Da die Kette eine idealisiert homogene Verteilung ihres Gewichts aufweist, folgt $\ddot{\mue}(t)\feq 0$:}
			\eqnf{\mue(t)}{y(t)\frac{m}{l}}
			\eqn[][\Rightarrow]{\dot{\mue}(t)}{\sqrt{\frac{g}{l}}\frac{(a-c_2)m}{l}e^{t\sqrt{\frac{g}{l}}}-\sqrt{\frac{g}{l}}\frac{c_2m}{l}e^{-t\sqrt{\frac{g}{l}}}}
			\eqn[][\Rightarrow]{\ddot{\mue}(t)}{\frac{mg}{l^2}\left((a-c_2)e^{t\sqrt{\frac{g}{l}}}+c_2e^{-t\sqrt{\frac{g}{l}}}\right)\feq 0}
			\eqn{0}{ae^{t\sqrt{\frac{g}{l}}}-c_2e^{t\sqrt{\frac{g}{l}}}+c_2e^{-t\sqrt{\frac{g}{l}}}}
			\eqn{0}{c_2(e^{-t\sqrt{\frac{g}{l}}}-e^{t\sqrt{\frac{g}{l}}})}
			\eqn{c_2}{0}
			\eqntext{$c_2$ in $y$ einsetzen:}
			\eqn[][\Rightarrow]{y(t)}{ae^{t\sqrt{\frac{g}{l}}}}
			\eqntext{Zum Zeitpunkt des Abrutschen, muss $y(t)=l$ betragen (Die gesamte Kette ist vom Tisch gerutscht):}
			\geqnf{ae^{t\sqrt{\frac{g}{l}}}}{\feq}{l}
			\eqn{t\sqrt{\frac{g}{l}}}{\ln(\frac{l}{a})}
			\eqn{t}{\frac{\ln(\frac{l}{a})}{\sqrt{\frac{g}{l}}}}
		\end{eqnn}
		
		\item[10.3] Löse die folgenden Anfangswertaufgaben:
		\begin{enumerate}
			\item $\ddot{y}+2\dot{y}-3y=e^{2t}, y(0)=0, \dot{y}(0)=1$.
			
			\item $\ddot{y}+y=t+2\cos(t), y(\pi)=2\pi, \dot{y}(\pi)=\pi$.
		\end{enumerate}
		
		\item[10.4]
		\begin{enumerate}
			\item Stelle eine homogene lineare Differentialgleichung 3.Ordnung auf, so dass eine Lösungsbasis gegeben ist durch $y_1(t)=1, y_2(t)=t, y_3(t)=e^t$.
			
			\item Finde eine homogene lineare Differentialgleichung, so dass $y_1:\R\to\R, y_1(t)=t$ und $y_2:\R\to\R, y_2(t)=\sin(t)$ Lösungen sind.
			
			\item Warum ist es in (b) unmöglich, eine homogene lineare Differentialgleichung 2.Ordnung aufzustellen, selbst wenn man zeitabhängige Koeffizienten zulässt, d.h. wenn man eine Differentialgleichung der Form
			$$\ddot{y}+a_1(t)\dot{y}+a_0(t)y=0$$\\
			sucht, mit $a_0,1_1:\R\to\R$?
			
			\item Warum ist es in (b) auch nicht möglich, eine homogene lineare Differentialgleichung 3.Ordnung aufzustellen, selbst wenn man zeitabhängige Koeffizienten zulässt?
			
			\item[] Hinweis zu (c) und (d): Wronski-Determinante, Satz 10.1, $n$ Lösungen $y_1,\hdots,y_n$ einer homogenen linearen Differentialgleichung $n$.Ordnung sind genau dann linear unabhängig, wenn
			$$\md{
				y_1(t)&y_2(t)&\hdots&y_n(t)\\
				\dot{y}_1(t)&\dot{y}_2(t)&\hdots&\dot{y}_n(t)\\
				\vdots&\vdots&\ddots&\vdots\\
				y^{(n-1)}_1(t)&y^{(n-1)}_2(t)&\hdots&y^{(n-1)}_n(t)
			}\neq 0$$\\
			für alle $t$. Was passiert hier für $t=0$?
		\end{enumerate}
		
		\item[10.5] Eine gedämpfte Schwingung ohne Anregung sei modelliert durch
		$$\ddots{y}+a\dot{y}+4y=0, \quad y(0)=5,\quad \dot{y}(0)=-1$$\\
		Bestimme den Parameter $a>0$ so, dass gerade der aperiodische Grenzfall eintritt, und berechne die Lösung für diesen Fall. Stelle den zeitlichen Verlauf der Schwingungen graphisch dar.
	\end{enumerate}
\end{document}
