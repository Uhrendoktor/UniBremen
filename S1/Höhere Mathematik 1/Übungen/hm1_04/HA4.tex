\documentclass{HM}

\usepackage{stmaryrd}
\newenvironment{LGS}{
 \begin{tabular}{}
}{
 \end{tabular}
}

\begin{document}
\begin{enumerate}
\item [4.3] Berechne den Rang der Matrizen
$$A=\begin{pmatrix}
1&1&2&0\\
7&2&-6&4\\
2&1&s&t
\end{pmatrix},
B=\begin{pmatrix}
1&8&-1&-1&2\\
-3&15&-3&0&3\\
-3&29&-7&2&5
\end{pmatrix}$$
Der Rang von $A$ hängt dabei von den Parametern $s,t\in\R$ ab.\\

\begin{align*}
	\eqnf{A}{\begin{pmatrix}
			1&1&2&0\\
			7&2&-6&4\\
			2&1&s&t
		\end{pmatrix}}
	\eqn{A}{\begin{pmatrix}
			1&1&2&0\\
			0&-5&-20&4\\
			0&-1&s-4&t
	\end{pmatrix}}
		\eqn{A}{\begin{pmatrix}
			1&1&2&0\\
			0&-5&-20&4\\
			0&0&s&t-\frac{4}{5}
	\end{pmatrix}}
\end{align*}
für $s\neq 0\land t \neq \frac{4}{5}$ ist der Rang 3\\
für $s=0\land t=\frac{4}{5}$ ist der Rang 2\\
\begin{align*}
	\eqnf{B}{\begin{pmatrix}
			1&8&-1&-1&2\\
			-3&15&-3&0&3\\
			-3&29&-7&2&5\\
	\end{pmatrix}}
	\eqn{B}{\begin{pmatrix}
			1&8&-1&-1&2\\
			0&39&-6&-3&9\\
			0&53&-10&-1&11\\
	\end{pmatrix}}
	\eqn{B}{\begin{pmatrix}
			1&8&-1&-1&2\\
			0&13&-2&-1&3\\
			0&0&-\frac{24}{13}&\frac{40}{13}&-\frac{16}{13}\\
	\end{pmatrix}}
		\eqn{B}{\begin{pmatrix}
			1&8&-1&-1&2\\
			0&13&-2&-1&3\\
			0&0&-24&40&-16\\
	\end{pmatrix}}
\end{align*}
Rang ist 3\\

\item [4.4] Bestimme alle Lösungen des linearen Gleichungssystems
$$\begin{matrix}
 &     & & 3x_2 &-& 5x_3 &+& x_4  &=& 0\\
-&x_1  &-& 3x_2 & &      &-& x_4  &=& -5\\
-&2x_1 &+& x_2  &+& 2x_3 &+& 2x_4 &=& 2\\
-&3x_1 &+& 4x_2 &+& 2x_3 &-& 2x_4 &=& 8
\end{matrix}$$
LGS definiert als A:
\begin{align*}
\eqnf{A}{\left(\begin{array}{cccc|c}
0&3&-5&1&0\\
-1&-2&0&-1&-5\\
-2&1&2&2&2\\
-3&4&2&2&8
\end{array}\right)} 
\eqn{A}{\left(\begin{array}{cccc|c}
-1&-3&0&-1&-5\\
-2&1&2&2&2\\
-3&4&2&2&8\\
0&3&-5&1&0
\end{array}\right)}
\eqn{A}{\left(\begin{array}{cccc|c}
-1&-3&0&-1&-5\\
0&7&2&4&12\\
0&13&2&5&23\\
0&3&-5&1&0
\end{array}\right)}
\eqn{A}{\left(\begin{array}{cccc|c}
-1&-3&0&-1&-5\\
0&7&2&4&12\\
0&0&-12&-17&5\\
0&0&-41&-5&-36
\end{array}\right)}
\eqn{A}{\left(\begin{array}{cccc|c}
-1&-3&0&-1&-5\\
0&7&2&4&12\\
0&0&-12&-17&5\\
0&0&0&1&-1
\end{array}\right)}
\end{align*}
$$\Rightarrow x_4=-1$$
\begin{align*}
\eqnfi{-12x_3-17x_4}{5}{(x_4=-1) \text{ einsetzen}}
\eqni{-12x_3+17}{5}{-17, /-12}
\eqn{x_3}{1}\\
\eqnfi{7x_2+2x_3+4x_4}{12}{(x_4=-1, c_3=1) \text{ einsetzen}}
\eqni{7x_2+2-4}{12}{+2, / 7}
\eqn{x_2}{2}\\
\eqnfi{-1x_1-3x_2-1x_4}{-5}{(x_2=2, x_4=-1) \text{ einsetzen}}
\eqni{-1x_1-6+1}{-5}{+5}
\eqn{x_1}{0}
\end{align*}
$\Rightarrow x_1=0, x_2=2, x_3=1, x_4=-1$
\item [4.5] Gegeben ist das lineare Gleichungssystem
$$\begin{matrix}
 &2x_1&+&x_2&+&x_3&=&0\\
 &2x_1&+&2x_2&+&tx_3&=&1\\
-&2tx_1&+&tx_2&+&9x_3&=&6
\end{matrix}$$
mit einem Parameter $t\in\R$.
\begin{enumerate}
\item Für welche $t\in\R$ ist das System eindeutig lösbar? Wie lautet die Lösung?
\item Für welche $t\in\R$ gibt es unendlich viele Lösungen? Gib alle lösungen an.
\item Für welche $t\in\R$ gibt es keine Lösung?\\\\
LGS als matrix $A$:
\begin{align*}
	\eqnf{A}{\left(\begin{array}{ccc|c}
		2&1&1&0\\
		2&2&t&1\\
		-2t&t&9&6\\
	\end{array}\right)}
	\eqn{A}{\left(\begin{array}{ccc|c}
		2&1&1&0\\
		0&1&t-1&1\\
		0&2t&9+t&6\\
	\end{array}\right)}
	\eqn{A}{\left(\begin{array}{ccc|c}
		2&1&1&0\\
		0&1&t-1&1\\
		0&0&9+3t-2t^2&6-2t\\
	\end{array}\right)}
\end{align*}

(b)
Berrechnen für welche $t$ dies $9+3t-2t^2$ 0 ergibt\\
\begin{align*}	
	\eqnfi{9+3t-2t^2}{0}{*-\frac{1}{2}}
	\eqn{-4.5-1.5t+t^2}{0}
	\text{pq-Formel mit }p=-1.5\text{ und }q=-4.5\\
	t\in\{3,-\frac{3}{2}\}: 9+3t-2t^2=0
\end{align*}
Bei betrachtung von Zeile 3 von $A$ fällt jetzt auf das für $t=3$ die komplette Zeile zu $0$ wird.
Damit haben wir zu wenig Informationen um eine eindeutige Lösung zu finden damit gibt es unentlich viele Lösungen.\\
\begin{align*}
	\eqn{A}{\left(\begin{array}{ccc|c}
			2&1&1&0\\
			0&1&t-1&1\\
			0&0&9+3t-2t^2&6-2t\\
		\end{array}\right)}
	\text{mit }t=3\\
	\eqn{A}{\left(\begin{array}{ccc|c}
			2&1&1&0\\
			0&1&2&1\\
		\end{array}\right)}
	\eqn{A}{\left(\begin{array}{ccc|c}
			1&0&-\frac{1}{2}&-\frac{1}{2}\\
			0&1&2&1\\
		\end{array}\right)}
	\eqnf{x_1-\frac{x_3}{2}}{-\frac{1}{2}}
	\eqn{x_1}{\frac{x_3-1}{2}}\\
	\eqnf{x_2+2x_3}{1}
	\eqn{x_2}{1-2x_3}
\end{align*}

Für $t=3$ hat $A$ unentlich viele Lösungen definiert durch:
$$x_1=\frac{x_3-1}{2} \text{ und }x_2=1-2x_3$$ 
\\
(c)
Unterste Zeile von $A$ mit $t=-\frac{3}{2}$ aus (b):\\
$$0x_3=9\lightning$$
Damit gibt es keine Lösung bei $$t=-\frac{3}{2}$$
\\\\
(a)
\begin{align*}
	A&=\left(\begin{array}{ccc|c}
			2&1&1&0\\
			0&1&t-1&1\\
			0&0&9+3t-2t^2&6-2t\\
		\end{array}\right)\\\\
	\text{Daraus folgt }
	\eqnf{x_3}{\frac{6-2t}{9+3t-2t^2}}
	\eqn{x_3}{\frac{3-t}{-(t^2-1.5t-4.5)}}
	\eqn{x_3}{\frac{3-t}{-((t-3)(t+1.5))}}
	\eqn{x_3}{\frac{2}{2t+3}}
	\\
	\eqnf{x_2}{1-(\frac{2}{2t+3}(t-1))}
	\eqn{x_2}{1-(\frac{2t-2}{2t+3})}
	\text{poly div: }\\
	\eqn{x_2}{\frac{5}{2t+3}}\\
	\eqnf{x_1}{-\frac{1}{2}-((-\frac{t}{2}+1)\frac{2}{2t+3})}
	\eqn{x_1}{-\frac{1}{2}+\frac{t-2}{2t+3}}
	\text{poly div:}\\
	\eqn{x_1}{\frac{-3.5}{2t+3}}
\end{align*}
Aus (b) unc (c) folgen das $t\notin\{3,-\frac{3}{2}\}$ sein muss
$$\text{für } t\notin\{3,-\frac{3}{2}\}:\mathbb{L}=\begin{pmatrix}
	x_1\\x_2\\x_3
\end{pmatrix}=\begin{pmatrix}
	-3.5\\5\\2
\end{pmatrix}*\frac{1}{2t+3}$$

\end{enumerate}
\item [4.6] Sei $A=(a_{j,k})\in\R^{m\times n}$
\begin{enumerate}
\item In jedem der folgenden fünf Fälle finde Matrizen $x$ und/oder $y$ mit folgenden Eigenschaften: Eines der Produkte $Ax, yA, yAx$ ist
\begin{enumerate}
\item die $j$-te zeile von $A$,
\item die $k$-te Spalte von $A$,
\item das Element $a_{jk}$,
\item die Summe der Einträge der $j$-ten zeile von $A$,
\item die Summe der Einträge der $k$-ten Spalte von $A$.
\end{enumerate}
\item Sei $B\in\R^{m\times n}$
\begin{enumerate}
\item die $j$-te und die $k$-te Spalte von $A$ vertauscht,
\item die $j$-te und die $k$-te Zeile von $A$ vertauscht,
\item das $\lambda$-Fache der $j$-ten Zeile zur $k$-ten zeile von $A$ addiert.
\end{enumerate}
In jedem der deri Fälle finde eine matrix $C$, so dass entweder $B=CA$ oder $B=AC$
\end{enumerate}
\end{enumerate}
\end{document}