\documentclass{HM}
\newcommand\course{HM 2}
\newcommand\hwnumber{1}

\newcommand{\temlim}{\lim\limits_{n\to\infty}}

\begin{document}
	\begin{enumerate}
		\item[1.5] Sei $f:\R\to\R, f(x)\coloneqq\sin x$. Für $n\in\N$ bestimme das Taylor-Polynom von $f$ vom Grad $n$ um den Entwicklungspunkt $a=0$. Zeige, dass die Taylor-Reihe von $f$ auf ganz $\R$ gegen $f$ konvergiert, d.h. für jedes (feste) $x\in\R$ gilt $R_n(x)\to 0$. Benutze dazu die Formel für das Lagrange-Restglied $R_n$ aus HM1, Satz 21.1.

		\begin{align*}
			\eqnf{T}{\sum_{k=0}^{n}\left(\frac{sin^{(k)}(0)}{k!}x^k\right)}
			\eqn{T}{\sum_{k=0}^{\frac{n-1}{2}}\left(\frac{(-1)^k}{(1+2k)!}x^{1+2k}\right)}\\
			R_n=\frac{sin^{(n+1)}(t)}{(n+1)!}x^{n+1}\\
			R_n\text{ läuft für }n\to\infty\text{ gegen $0$, wenn}\\
			\geqnf{(n+1)!}{>}{x^{n+1}}
			\geqn{n n!}{>}{x^n x}
		\end{align*}
		Da $n$ größer als $x$ wird und $n!$ größer als $x^n$ ist, folgt das $R_n$ für n$\to \infty$ gegen $0$ läuft
		\item[1.6] Sind die angegebenen Funktionen $\varphi_k:[0,2]\to\R(k=1,2,3,4)$ Treppenfunktionen? Wenn ja, ist ihr Integral zu ermitteln.
		\begin{enumerate}
			\item $\varphi_1(x)=\floor{x}$
				$$\varphi_1(x)=\floor{x}=\begin{cases}
					\vdots& \vdots\\
					0&0\leq x < 1\\
					1&1\leq x < 2\\
					2&2\leq x < 3\\
					\vdots& \vdots\\
					n&n\leq x < n+1
				\end{cases}$$
				$\Rightarrow \varphi_1\in T$ ($\varphi_1$ ist eine Treppenfunktion) mit Treppenpunkten $= \{\hdots,1,2,3,\hdots,n\}$\\
				\begin{align*}
					\Rightarrow &\int_a^b\varphi_1(x)dx = \int_0^b\varphi_1(x)dx-\int_0^a\varphi_1(x)dx\\
					\text{mit }&\int_0^c\varphi_1(x)dx
					=\left(\sum\limits_{k=1}^{\floor{c}}k-1\right) + (c-\floor{c})\floor{c}\\
					=&\left(\sum\limits_{k=1}^{\floor{c}}k\right) - \floor{c} + (c-\floor{c})\floor{c}\\
					=&\frac{\floor{c}(\floor{c}+1)}{2}+(c-\floor{c}-1)\floor{c}
				\end{align*}
				$$\int_0^2\varphi_1(x)dx=\frac{6}{2}-2-0=1$$\\
			\item $\varphi_2(x)=\floor{2x}$
				$$\varphi_2(x)=\varphi_1(2x)$$
				$\Rightarrow \varphi_2\in T$ da $\varphi_1\in T$
				$$\Rightarrow \int_a^b\varphi_2(x)dx=\int_a^b\varphi_1(2x)dx=\frac{\int_{2a}^{2b}\varphi_1(x)dx}{2}$$
				$$\Rightarrow\int_0^2\varphi_2(x)dx=\frac{\int_0^4\varphi_1(x)}dx{2}=\frac{\frac{20}{2}-4-0}{2}=3$$\\
			\item $\varphi_3(x)=7\floor{x}-5\floor{2x}$
				$$\varphi_3(x)=7\varphi_1(x)-5\varphi_2(x)=7\varphi_1(x)-5\varphi_1(2x)$$
				$\Rightarrow \varphi_3\in T$ da sie sie eine lineare Kombination aus den beiden Treppenfunktionen $\varphi_1$ und $\varphi_2$ ist. (Vektorraum)
				$$\Rightarrow \int_a^b\varphi_3(x)dx=
				7\int_a^b\varphi_1(x)dx-5\int_a^b\varphi_1(2x)dx=
				7\int_a^b\varphi_1(x)dx-\frac{5}{2}\int_{2a}^{2b}\varphi_1(x)dx$$
				$$\Rightarrow 7\int_0^2\varphi_1(x)dx-5\int_0^2\varphi_2(x)dx=7-15=-8$$\\
			\item $\varphi_4(x)=\begin{cases}
				0&\text{falls }x=0,\\
				\floor{\frac{1}{x}}&\text{falls }x\neq 0.
			\end{cases}$
			$$\int_0^2\varphi_4(x)dx=\int_0^1\varphi_4(x)dx+\int_1^2\varphi_4(x)dx$$\\
			$\forall x>1:\frac{1}{x}<1\Rightarrow\forall x>1:\floor{\frac{1}{x}}=0$
			$$\Rightarrow\int_1^2\varphi_4(x)dx=0\Rightarrow\int_0^2\varphi_4(x)dx=\int_0^1\varphi_4(x)dx$$\\
			$$a_n\coloneqq\left(\frac{1}{1},\frac{1}{2},\frac{1}{3},\hdots,\frac{1}{n}\right)$$
			berechnen der Treppenfunktion (Fläche unter den letzten n Stufen): $$\sum_{k=1}^n(a_k-a_{k+1})k=\sum_{k=1}^n(\frac{1}{k}-\frac{1}{k+1})k=\sum_{k=1}^n\frac{1}{k+1}$$
			\begin{align*}
				&\int_0^1\varphi_4(x)dx\\
				=&\lim\limits_{n\to\infty}\sum_{k=1}^n\frac{1}{k+1}\\
				=&\sum\limits_{k=1}^\infty\frac{1}{k+1}\\
				=&\left(\sum\limits_{k=0}^\infty\frac{1}{k+1}\right)-1\\
				=&\left(\sum\limits_{k=1}^\infty\frac{1}{k}\right)-1\\
			\end{align*}
			$\Rightarrow$ harmonische Reihe\\
			$\Rightarrow$ divergiert\\
			$\Rightarrow$ nicht Riemann integrierbar.
		\end{enumerate}

		\item[1.7] Die rationalen Zahlen im Intervall $[0,2)$ seien als Folge $(r_n)_{k\in\N}$ geschrieben. Entscheide, ob die angegebenen Funktionen $f_n:[0,2]\to\R(m=1,2,3,4)$ Riemann-integrierbar sind.
		\begin{enumerate}
			\item $f_1(x)=\floor{2x}$;\\
			Laut 1.6 ist $\floor{2x}$ eine Treppenfunktion $\Rightarrow \floor{2x}$ ist R-intbar\\

			\item $f_2(x)=e^{-x^{2}}$;\\
			$e^x$ ist stetig und $-x²$ ist stetig $\Rightarrow e^{-x^2}$ stetig. Stetige Funktionen sind R-intbar also ist $f_2$ R-intbar. \\

			\item $f_3(x)=\sum\limits_{k:r_k<x}2^{-k}$;\\
			
			Da zwischen 0 und x unendlich viele realle Zahlen liegen, gibt es unendlich viele k über die summiert wird. $$\Rightarrow\sum_{k=0}^{\infty}\frac{1}{2}^k$$
			nach geometrischer Reihe:
			$$\Rightarrow\sum_{k=0}^{\infty}\frac{1}{2}^k = \frac{1}{1+\frac{1}{2}}=2$$
			Damit ist $f_3$ eine konstante funktion für alle Stellen außer 0 und damit R-intbar.\\

			\item $f_4(x)=\begin{cases}
				0&\text{falls }x=0,\\
				x^{-2}&\text{falls }x\neq 0.
			\end{cases}$\\
		
			$f_4$ ist unbeschränkt und damit nicht R-intbar.\\
			Es läuft für $x\to0$ nach $\infty$.
		\end{enumerate}

		\newpage
		\item[1.8] Sei $a>1$. gehe ähnlich wie in Aufgabe 1.4 vor, um das Riemann-Integral $\int_1^a\frac{dx}{x}$ zu bestimmen.
		$$\int_1^a\frac{dx}{x}=\int_1^a\frac{1}{x}dx=\int_1^af(x)dx$$
		$\Rightarrow f(x)=\frac{1}{x}$
		$$1=x_0<x_1<x_2<\hdots<x_n=a \text{ mit } x_k\coloneqq a^{\frac{k}{n}} \text{ mit } x,k\in\N; k<x$$
		$$\int_1^a\varphi(x)dx=\sum\limits_{k=0}^n(x_{k+1}-x_k)f(x_k)>\int_1^af(x)dx>\int_1^a\psi(x)dx=\sum\limits_{k=0}^n(x_{k+1}-x_k)f(x_{k+1})$$
		\begin{align*}
			\eqnf{\lim\limits_{n\to\infty}\int_1^a\varphi(x)dx}{\lim\limits_{n\to\infty}\int_1^a\psi(x)dx=\int_1^af(x)dx}
			\eqn{\sum\limits_{k=0}^\infty(x_{k+1}-x_k)f(x_k)}{\sum\limits_{k=0}^\infty(x_{k+1}-x_k)f(x_{k+1})}
			\eqn{\sum\limits_{k=0}^\infty(a^{\frac{k+1}{n}}-a^{\frac{k}{n}})\frac{1}{a^{\frac{k}{n}}}}{\sum\limits_{k=0}^\infty(a^{\frac{k+1}{n}}-a^{\frac{k}{n}})\frac{1}{a^{\frac{k+1}{n}}}}
			\eqn{\sum\limits_{k=0}^\infty(a^{\frac{k}{n}}a^{\frac{1}{n}}-a^{\frac{k}{n}})\frac{1}{a^{\frac{k}{n}}}}{\sum\limits_{k=0}^\infty(a^{\frac{k}{n}}a^{\frac{1}{n}}-a^{\frac{k}{n}})\frac{1}{a^{\frac{k}{n}}}\frac{1}{a^{\frac{1}{n}}}}
			\eqn{\sum\limits_{k=0}^\infty(a^{\frac{1}{n}}-1)}{\sum\limits_{k=0}^\infty(1-a^{-\frac{1}{n}})}
			\eqn{\lim\limits_{n\to\infty}n(a^{\frac{1}{n}}-1)}{\lim\limits_{n\to\infty}n(1-a^{-\frac{1}{n}})}
		\end{align*}
		\begin{align*}
		\Rightarrow &\temlim n(1-a^{-\frac{1}{n}})\\
		=&\temlim n\left(\frac{a^{-\frac{1}{n}}}{a^{-\frac{1}{n}}}-a^{-\frac{1}{n}}\right)\\
		=&\temlim a^{-\frac{1}{n}}n(a^{\frac{1}{n}}-1)\\
		=&\frac{\temlim n(a^{\frac{1}{n}}-1)}{\temlim a^{\frac{1}{n}}}\\
		\end{align*}
		\begin{align*}
		=&\frac{\temlim n(a^{\frac{1}{n}}-1)}{1}\\
		=&\temlim n(a^{\frac{1}{n}}-1)\\
		=&\temlim \frac{a^{\frac{1}{n}}-1}{\frac{1}{n}}\\
		=&\temlim \frac{\frac{d}{dx}(e^{\ln(a)^{\frac{1}{n}}}-1)}{\frac{d}{dx}\frac{1}{n}}\\
		=&\temlim \frac{\frac{d}{dx}\left(\frac{\ln(a)}{n}\right)e^{\ln(a)^{\frac{1}{n}}}}{-\frac{1}{n^2}}\\
		=&\temlim \frac{\frac{\ln(a)}{n^2} a^{\frac{1}{n}}}{-\frac{1}{n^2}}\\
		=&\temlim \ln(a)a^{\frac{1}{n}}\\
		=&\ln(a)
		\end{align*}
		$\Rightarrow \int_1^a\frac{dx}{x}=\ln(a)$
	\end{enumerate}
\end{document}
