\documentclass{HM}
\newcommand\course{HM 2}
\newcommand\hwnumber{5}
\usepackage{gauss}

\newcommand{\Eig}{\text{Eig}}
\newcommand{\Span}{\text{Span}}

\begin{document}
	\begin{enumerate}
	\item[5.2] Es sei
	$$A:=\begin{pmatrix}
		-1-3i&2+i&-1-2i\\
		0&2&0\\
		2+4i&-2-i&2+3i
	\end{pmatrix}$$

Bestimme die Eigenwerte von $A$ samt ihrer algebraischen und geometrischen Vielfachheiten,
sowie die zugehörigen Eigenräume.\\

\begin{align*}
	\eqnf{p_A(\lambda)}{
	\begin{vmatrix}
		-1-3i-\lambda&2+i&-1-2i\\
		0&2-\lambda&0\\
		2+4i&-2-i&2+3i-\lambda\\
	\end{vmatrix}}
	\text{Entwickeln nach 2. Zeile:}\\
	\eqn{p_A(\lambda)}{(2-\lambda)\begin{gmatrix}[v]
			-1-3i-\lambda&-1-2i\\
			2+4i&2+3i-\lambda
		\rowops
		\add{1}{0}
	\end{gmatrix}}
	\eqn{p_A(\lambda)}{(2-\lambda)\begin{gmatrix}[v]
		1+i-\lambda&1+i-\lambda\\
		2+4i&2+3i-\lambda
		\rowops
		\add[-2]{0}{1}
	\end{gmatrix}}
	\eqn{p_A(\lambda)}{(2-\lambda)\begin{gmatrix}[v]
		1+i-\lambda&1+i-\lambda\\
		2i+2\lambda&i+\lambda
	\end{gmatrix}}
	\eqn{p_A(\lambda)}{(2-\lambda)(1+i-\lambda)\begin{vmatrix}
			1&1\\
			2i+2\lambda&i+\lambda\\
	\end{vmatrix}}
	\eqn{p_A(\lambda)}{(2-\lambda)(1+i-\lambda)(-i-\lambda)}\\
	&\Rightarrow \lambda_1=2, \lambda_2=i+1,\lambda_3=-i\\
	\alpha_\lambda&=1\\
\end{align*}
\begin{align*}
		\text{Berechnung der Eigenvektoren: }
	\lambda_1:\\
	&\begin{gmatrix}[p]
		-3-3i&2+i&-1-2i\\
		2+4i&-2-i&3i
		\rowops
		\add{0}{1}
	\end{gmatrix}\\
	&\begin{gmatrix}[p]
		-3-3i&2+i&-1-2i\\
		-1+i&0&-1+i
		\rowops
		\add[2]{1}{0}
		\mult{1}{\frac{1}{-1+i}}
	\end{gmatrix}\\
	&\begin{gmatrix}[p]
		-5-i&2+i&-3\\
		1&0&1
		\rowops
		\add[3]{1}{0}
		\mult{0}{\frac{1}{2+i}}
	\end{gmatrix}\\
	&\begin{gmatrix}[p]
	-1&1&0\\
	1&0&1
\end{gmatrix}\\
&\Rightarrow \Eig(A,\lambda_1)=\Span\left\{\begin{pmatrix}
	1\\1\\-1
\end{pmatrix}\right\}\\
&\gamma_{1}=1\\\\
\lambda_2:\\
&\begin{gmatrix}[p]
	-2-4i&2+i&-1-2i\\
	0&1-i&0\\
	2+4i&-2-i&1+2i
	\rowops
	\add{1}{0}
	\add[-1]{1}{2}
	\mult{1}{\frac{1}{1-i}}
	\add[-3]{1}{0}
	\add[3]{1}{2}
\end{gmatrix}\\
&\begin{gmatrix}[p]
	-2-4i&0&-1-2i\\
	0&1&0\\
	2+4i&0&1+2i
	\rowops
	\mult{2}{\frac{1}{1+2i}}
\end{gmatrix}\\
&\begin{gmatrix}[p]
	0&1&0\\
	2&0&1
\end{gmatrix}\\
&\Rightarrow \Eig(A,\lambda_2)=\Span\left\{\begin{pmatrix}
	-1\\0\\2
\end{pmatrix}\right\}\\
&\gamma_{2}=1\\\\
\lambda_3:\\
&\begin{gmatrix}[p]
	-1-2i&2+i&-1-2i\\
	0&2+i&0\\
	2+4i&-2-i&2+4i
	\rowops
	\add[-1]{1}{0}
	\add{1}{2}
	\mult{1}{\frac{1}{2+i}}
	\mult{0}{\frac{1}{-1-2i}}
	\mult{2}{\frac{1}{2+4i}}
\end{gmatrix}\\
&\begin{gmatrix}[p]
	1&0&1\\
	0&1&0
\end{gmatrix}\\
&\Rightarrow \Eig(A,\lambda_3)=\Span\left\{\begin{pmatrix}
	1\\0\\-1
\end{pmatrix}\right\}\\
&\gamma_{3}=1
\end{align*}
	
	\item[5.3] Sei $A \in \C^{n\times n}$ und $v \in \C^n$ ein Eigenvektor von $A$ zum Eigenwert $\lambda$.
	\begin{enumerate}
		\item[a)] Zeige, dass $v$ auch Eigenvektor von $A^2$ ist. Zu welchem Eigenwert?\\
		\begin{align*}
			\eqnfi{Av}{v\lambda}{\cdot \text{A von links}}
			\eqn{A^2v}{Av\lambda}
			\eqn{A^2v}{v\lambda^2}
		\end{align*}
		$\Rightarrow v$ ist Eingenvektor von A zu $\lambda^2$\\
		
		\item[b)] Zeige, dass $v$ Eigenvektor von $A^{-1}$ ist, wenn $A$ invertierbar ist. Zu welchem Eigenwert?\\
		
		\begin{align*}
			\eqnfi{Av}{v\lambda}{\cdot A^{-1} \text{von links}}
			\eqn{v}{A^{-1}v\lambda}
			\eqn{\frac{1}{\lambda}v}{A^{-1}v}
		\end{align*}
		v ist Eigenvektor von $A^{-1}$ zum Eigenwert $\frac{1}{\lambda}$
		
		\item[c)] Wenn $A^2 = E_n$ ist, wieso ist dann mindestens eine der Zahlen ±1 Eigenwert von $A$?
		Wieso gibt es keine anderen Eigenwerte?\\
		
		\begin{align}
			A&=A^{-1}\\
			&\Rightarrow Av=v\lambda, A^{-1}v=v\lambda\\
			\eqnfi{A^{-1}v}{v\lambda}{\cdot A \text{ von links}}
			\eqni{v}{Av\lambda}{mit (2)}
			\eqn{v}{v\lambda^2}
			\eqn{1}{\lambda^2}
		\end{align}
	$\Rightarrow$ die einzigen möglichen Eigenwerte sind $\pm1$
		\item[d)] Haben $A$ und $A^T$ dieselben Eigenwerte?\\
		\begin{align*}
			\det(A)&=\det(A^T)\\
			\det(A^T-\lambda E_n)&=\det(A-\lambda E_n)^T\\
			\intertext{da die veränderung der Diagonalen nicht die transponierbarkeit beeinflusst}\\
			\Rightarrow \det(A-\lambda E_n)&=\det(A-\lambda E_n)^T\\
		\end{align*}
	Da das charakteristische Polynom gleich ist, sind auch die Eigenwerte gleich.
		
	\end{enumerate}

	\item[5.4] Es Sei
	$$\begin{pmatrix}
		0&-3&0&0\\
		-1&2&-1&0\\
		-1&3&-1&0\\
		-1&3&-1&0
	\end{pmatrix}$$
Berechne die Eigenwerte von $A$ samt ihrer algebraischen und geometrischen Vielfachheiten.
Ist die Matrix $A$ diagonalisierbar?\\

\begin{align*}
	\eqnf{p_A(\lambda)}{\begin{gmatrix}[v]
		-\lambda&-3&0&0\\
		-1&2-\lambda&-1&0\\
		-1&3&-1-\lambda&0\\
		-1&3&-1&-\lambda
	\end{gmatrix}}
\text{Entwickeln nach der 4-ten Spalte}\\
\eqn{p_A(\lambda)}{-\lambda\begin{gmatrix}[v]
	-\lambda&-3&0\\
	-1&2-\lambda&-1\\
	-1&3&-1-\lambda
	\rowops
	\add{0}{2}
\end{gmatrix}}
\eqn{p_A(\lambda)}{-\lambda(-1-\lambda)\begin{gmatrix}[v]
	-\lambda&-3&0\\
	-1&2-\lambda&-1\\
	1&0&1
	\rowops
	\add{2}{1}
\end{gmatrix}}
\eqn{p_A(\lambda)}{-\lambda(-1-\lambda)\begin{gmatrix}[v]
	-\lambda&-3&0\\
	0&2-\lambda&0\\
	1&0&1
\end{gmatrix}}
\text{Entwickeln nach der 3-ten Zeile}\\
\eqn{p_A(\lambda)}{-\lambda(-1-\lambda)\begin{gmatrix}[v]
	-\lambda&-3\\
	0&2-\lambda
\end{gmatrix}}
\eqn{p_A(\lambda)}{-\lambda(-1-\lambda)(-2\lambda+\lambda^2)}
\end{align*}
\begin{align*}
	&\lambda^2(1+\lambda)(-2+\lambda)=0\\
	\Rightarrow &\lambda_1=0, &\alpha_1=2\\
	&\lambda_2=-1, &\alpha_1=1\\
	&\lambda_3=2, &\alpha_1=1\\
\end{align*}
\begin{align*}
	\text{Eigenvektoren ausrechnen: }\lambda_1:\\
	&\begin{pmatrix}
		0&-3&0&0\\
		-1&2&-1&0\\
		-1&3&-1&0\\
		-1&3&-1&0
	\end{pmatrix}\\
&\begin{gmatrix}[p]
	0&-3&0\\
	-1&2&-1\\
	-1&3&-1
	\rowops
	\mult{0}{\frac{1}{-3}}
	\add[-2]{0}{1}
	\add[-3]{0}{2}
	\mult{1}{-1}
	\mult{2}{-1}
\end{gmatrix}\\
&\begin{gmatrix}[p]
	0&1&0\\
	1&0&1\\
	1&0&1
\end{gmatrix}\\
\Rightarrow \Eig(A,\lambda_1)=&\Span\left\{
\begin{pmatrix}1\\0\\-1\\0\end{pmatrix},
\begin{pmatrix}0\\0\\0\\1\end{pmatrix}
\right\}\\
\gamma_1=2
\end{align*}
\begin{align*}
	\lambda_2:\\
	&\begin{gmatrix}[p]
		1&-3&0&0\\
		-1&3&-1&0\\
		-1&3&0&0\\
		-1&3&-1&1
		\rowops
		\add[-1]{1}{3}
		\add{0}{1}
		\mult{1}{-1}
	\end{gmatrix}\\
	&\begin{gmatrix}[p]
		1&-3&0&0\\
		0&0&1&0\\
		0&0&0&1
	\end{gmatrix}\\
	\Rightarrow \Eig(A,\lambda_2)=&\Span\left\{
	\begin{pmatrix}3\\1\\0\\0\end{pmatrix}
	\right\}\\
	\gamma_2=1
\end{align*}
\begin{align*}
	\lambda_3:\\
	&\begin{gmatrix}[p]
		-2&-3&0&0\\
		-1&0&-1&0\\
		-1&3&-3&0\\
		-1&3&-1&-2
		\rowops
		\add{0}{2}
		\add{0}{3}
		\mult{1}{-1}
		\mult{0}{-1}
		\mult{3}{-1}
	\end{gmatrix}\\
	&\begin{gmatrix}[p]
		2&3&0&0\\
		1&0&1&0\\
		3&0&1&2
		\rowops
		\add[-1]{1}{2}
		\mult{2}{\frac{1}{2}}
	\end{gmatrix}\\
	&\begin{gmatrix}[p]
	2&3&0&0\\
	1&0&1&0\\
	1&0&0&1
\end{gmatrix}\\
	\Rightarrow \Eig(A,\lambda_3)=&\Span\left\{
	\begin{pmatrix}-3\\2\\3\\3\end{pmatrix}
	\right\}\\
	\gamma_3=1
\end{align*}
Da $\alpha_\lambda = \gamma_\lambda$ ist A diagonalbar
	\item[5.5]
	Die \textit{Fibonacci-Folge}:\\
	Jemand brachte ein Kaninchenpaar in einen gewissen, allseits von Wänden umgebenen
Ort, um herauszufinden, wieviel [Paare] aus diesem Paar in einem neuen Jahr entstehen
würden. Es sei die Natur der Kaninchen, pro Monat ein neues Paar hervorzubringen und im
zweiten Monat nach der Geburt [erstmals] zu gebären. [Todesfälle mögen nicht eintreten.]
Die Anzahl der Kaninchenpaare im $n$-ten Monat ist somit durch die rekursiv definierte
Folge:\\
	$$x_{n+1}=x_n+x_{n-1}$$
	mit den Startwerten $x_0=1$ und $x_1=1$, gegeben. Diese Folge wird \textit{Fibonacci-Folge} genannt.\\
	\begin{enumerate}
		\item Gib eine Matrix $A\in\R^{2\times 2}$ an, so dass\\\
		$$A\begin{pmatrix}
			x_{n-1}\\
			x_n
		\end{pmatrix}=\begin{pmatrix}
			x_n\\
			x_{n+1}
		\end{pmatrix}$$ für alle $n\in\N$ ist.\\
		\begin{align*}
			\begin{pmatrix}
				0&1\\
				1&1\\		
			\end{pmatrix}						
			\begin{pmatrix}
				x_{n-1}\\
				x_n
			\end{pmatrix}=\begin{pmatrix}
				x_n\\
				x_n+x_{n-1}
			\end{pmatrix}=\begin{pmatrix}
				x_n\\
				x_{n+1}
			\end{pmatrix}
		\end{align*}	
		$\Rightarrow A=\begin{pmatrix}
				0&1\\
				1&1\\		
			\end{pmatrix}$
			
		\item Zeige, dass $A$ diagonalisierbar ist und bestimme eine invertierbare Matrix $T$ mit $T^{-1}AT=D$, wobei $D$ eine Diagonalmatrix ist.\\
		\item Finde eine explizite (d.h. nicht rekursive) Formel für das Folgenglied $x_n$, für beliebiges $n\in\N$.
		\begin{align*}
			\eqnf{A\begin{pmatrix}
				x_{n-1}\\
				x_n
			\end{pmatrix}}{\begin{pmatrix}
				x_n\\
				x_{n+1}
			\end{pmatrix}}
			\eqn{A^n\begin{pmatrix}
				x_0\\
				x_1
			\end{pmatrix}}{\begin{pmatrix}
				x_n\\
				x_{n+1}
			\end{pmatrix}}\\
			\eqn{\begin{pmatrix}
				0&1\\
				1&1
			\end{pmatrix}^n\begin{pmatrix}
				1\\1
			\end{pmatrix}}{\begin{pmatrix}
				x_n\\
				x_{n+1}
			\end{pmatrix}}
		\end{align*}
	\end{enumerate}
	\end{enumerate}
\end{document}
