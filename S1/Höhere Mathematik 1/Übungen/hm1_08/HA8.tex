\documentclass{HM}
\usepackage{listings}
\begin{document}
\begin{enumerate}
\item[8.3] Untersuche die Reihe $\sum\limits_{n=1}^\infty a_n$ auf Konvergenz und absolute Konvergenz, wobei\\
\begin{enumerate}

\item $a_n=\left(\frac{1+i}{2}\right)^n$;
\begin{align*}
 	&\sum\limits_{n=1}^\infty\left(\frac{1+i}{2}\right)^n\\
	=&\sum\limits_{n=0}^\infty\left(\frac{1+i}{2}\right)^n-1
\end{align*}
Geometrische Reihe mit $|\frac{1+i}{2}| = \frac{|1+i|}{|2|} = \frac{\sqrt{2}}{2} < 1$:\\
$\Rightarrow$ konvergent.\\\\
Überprüfen absoluter Konvergenz:\\
\begin{align*}-
	&\sum\limits_{n=1}^\infty\left|\left(\frac{1+i}{2}\right)^n\right|\\
	=&\sum\limits_{n=1}^\infty\left|\frac{1+i}{2}\right|^n\\
	=&\sum\limits_{n=1}^\infty\left(\frac{\sqrt{2}}{2}\right)^n\\
	=&\sum\limits_{n=0}^\infty\left(\frac{\sqrt{2}}{2}\right)^n-1\\
\end{align*}
Aus der geometrischen Reihe mit $|\frac{\sqrt{2}}{2}| < 1$ folgt $\sum\limits_{n=1}^\infty\left|\left(\frac{1+i}{2}\right)^n\right|$ konvergiert.\\
$\Rightarrow$ absolut konvergent\\

\item $a_n=\frac{(-1)^n}{\sqrt[3]{n}}$;\\
Quotientenkriterium:
\begin{align*}
	&\left|\frac{(-1)^n\sqrt[3]{n+1}}{(-1)^n\sqrt[3]{n}}\right|\\
	=&\left|\frac{-1\cdot\sqrt[3]{n+1}}{\sqrt[3]{n}}\right|\\
	=&\left|\frac{\sqrt[3]{n+1}}{\sqrt[3]{n}}\right|\\
\end{align*}
Aus $\sqrt{a}>\sqrt{b}$ für $a>b$ folgt $\left|\frac{\sqrt[3]{n+1}}{\sqrt[3]{n}}\right|>1$.\\
$\Rightarrow$ divergent $\Rightarrow$ nicht absolut konvergent.\\

\item $a_n=(-1)^n\frac{n+2}{2n}$.\\
Quotientenkriterium:\\
\begin{align*}
	&\left|(-1)^{n+1}\frac{n+3}{2(n+1)}\frac{1}{(-1)^n}\frac{2n}{n+2}\right|\\
	=&\left|(-1)\frac{(n+3)(2n)}{(2n+2)(n+2)}\right|\\
	=&\left|\frac{2n^2+6n}{2n^2+6n+4}\right|<1
\end{align*}
$\Rightarrow$ konvergent.\\\\
Überprüfen absoluter Konvergenz:\\
\begin{align*}
	&\left|\frac{\left|(-1)^{n+1}\frac{n+3}{2(n+1)}\right|}{\left|(-1)^n\frac{n+2}{2n}\right|}\right|\\
	=&\left|\frac{(n+3)(2n)}{2(n+1)(n+2)}\right|\\
	=&\left|\frac{2n^2+6n}{2n^2+6n+4}\right|<1
\end{align*}
$\Rightarrow$ absolut konvergent.
\end{enumerate}
\item[8.4] Sei $s_n\coloneqq \sum\limits_{k=0}^n\frac{1}{k!}$ und $e=\lim\limits_{n\to\infty}s_n$ die Euler'sche Zahl.
\begin{enumerate}
\item Zeige die Ungleichungen $0<e-s_n<\frac{1}{n*n!}$ für $n\in\N$, mit Hilfe einer geeigneten geometrischen Reihe.
\item Bestimme mit Hilfe von (a) eine Zahl $n\in\N$, für die $|e-s_N|\leq 0.5\cdot 10^{-4}$ gilt, und gib den Wert von $s_N$ an.
\item Zeige, dass die Euler'sche Zahl $e$ irrational ist.
\end{enumerate}
\item[8.5]
\begin{enumerate}
\item Konvergiert die Reihe $\sum\limits_{n=2}^\infty\left(\frac{1}{\sqrt{n}-1}+\frac{1}{\sqrt{n}+1}\right)$?
\item Berechne $\sum\limits_{n=1}^\infty\frac{n-1}{n!}$.\\
\begin{align*}
	\eqnf{A}{\sum_{n=1}^{\infty}\frac{n-1}{!n}}
	\eqn{A}{\sum_{n=1}^{\infty}\left(\frac{n}{!n}-\frac{1}{!n}\right)}
	\frac{n}{n!}&=\frac{1}{(n-1)!}\\
	\Rightarrow A&=\sum_{n=1}^{\infty}\left(\frac{1}{(n-1)!}-\frac{1}{!n}\right)\\
	\eqn{A}{1-1+1-\frac{1}{2!}+\frac{1}{2!}-\frac{1}{3!}+\frac{1}{3!}-\frac{1}{4!}+\dots+\frac{1}{(n-1)!-\frac{1}{n!}}}
	\eqn{A}{1-\frac{1}{n!}}
\end{align*}
\item Für welche $z\in\C\setminus\lbrace-1\rbrace$ konvergiert die Reihe $\sum\limits_{n=1}^\infty\frac{z^{n-1}}{(1+z)^n}?$\\
Bestimme den Grenzwert, falls er existiert.\\\\
Quotientenkriterium:
\begin{align*}
	&\left|\frac{z^n (1+z)^n}{(1+z)^{n+1}z^{n-1}}\right|\\
	\Leftrightarrow &\left|\frac{z}{1+z}\right|
\end{align*}
Da $|z|<|1+z|$ muss $\frac{|z|}{|1+z|} < 1$ und damit wissen wir dass:
$\sum\limits_{n=1}^\infty\frac{z^{n-1}}{(1+z)^n}$ für all $z$ konvergiert\\
\begin{align*}
	\sum\limits_{n=1}^\infty\frac{z^{n-1}}{(1+z)^n}\\
	\Leftrightarrow \frac{1}{z}\sum\limits_{n=1}^\infty\left(\frac{z}{1+z}\right)^n\\
	\text{da wir schon wissen dass }\left|\frac{z}{1+z}\right|<1\\
	\text{können wir die geometrische Reihe verwenden}\\
	\eqnf{\frac{1}{z}\sum\limits_{n=1}^\infty\left(\frac{z}{1+z}\right)^n}{\frac{1}{z}\sum\limits_{n=0}^\infty\left(\frac{z}{1+z}\right)^n-1}
	\eqn{\frac{1}{z}\sum\limits_{n=1}^\infty\left(\frac{z}{1+z}\right)^n }{\frac{1}{z}\frac{1}{1-\frac{z}{1+z}}-1}
	\eqn{\frac{1}{z}\sum\limits_{n=1}^\infty\left(\frac{z}{1+z}\right)^n }{\frac{1+z}{z}-1}
	\eqn{\sum\limits_{n=1}^\infty\frac{z^{n-1}}{(1+z)^n}}{\frac{1}{z}}
\end{align*}
Fehler

\end{enumerate}
\item[8.6] Ermittle (durch Probieren) das kleinste $n\in\N$, für dass $\sum\limits_{k=1}^n\frac{1}{k}>3$ ist. benutze einen Computer, um herauszufinden, wie groß man $n$ wählen muss, damit die Summe $>6$ bzw. $>9$ wird.\\
\begin{minipage}{\textwidth /2}
	\begin{scriptsize}
		\lstinputlisting[language=Python,firstline=8]{8.6.py}
	\end{scriptsize}
\end{minipage}
\begin{minipage}{\textwidth /2}
	\begin{align*}
		\sum_{k=1}^{11}&=3.0198773448773446\\
		\sum_{k=1}^{227}&=6.004366708345567\\
		\sum_{k=1}^{4550}&=9.000208062931115\\	
		\mathbb{L}&=\{11,227,4550\}
	\end{align*}
	
\end{minipage}

\end{enumerate}
\end{document}