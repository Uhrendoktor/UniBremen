\documentclass{../HM}
\newcommand\course{HM 2}
\newcommand\hwnumber{11}
\usepackage{gauss}
\usepackage{tikz}
\usepackage{pgfplots}

\newcommand{\mue}{m_{\textit{ü}}}
\begin{document}
	\begin{enumerate}
		\item[11.3]
		\begin{enumerate}
			\item Sei $D \coloneqq \{(x,y,z)\in \R^3;x\neq 0\}$. Bestimme den Gradienten $\varDelta f$ der Funktion $f: D \to \R, f(x,y,z) \coloneqq y\sin(xz^2)+\frac{x \cos(y)}{y}$.\\
			$$\varDelta f = \m{
				y\cos(xz^2)z^2+\frac{\cos(y)}{y}\\
				\sin(xz^2)-x(\frac{\sin(y)}{y}+\frac{\cos(y)}{y^2})\\
				y\cos(2xz)
			}$$
			\item Entscheide, ob es partiell differenzierbare Funktionen auf $R^2$ bzw. $R^3$ gibt mit
			$$\varDelta f(x,y)=\m{x+y\\-x+y}\text{ bzw. }\varDelta f(x,y,z)=\m{yz\\xz\\xy}$$
			Bestimme gegebenenfalls ein solches $f$.\\
			\begin{eqnn}
				\eqnf{\varDelta f(x,y)}{\m{x+y\\-x+y}}
				\eqnr{f(x,y)}{\int x+y dx}
				\eqn{f(x,y)}{\int -x+y dy}
				\eqnr{\int x+y dx}{\int -x+y dy}
				\eqn{\frac{x^2}{2}+yx}{\frac{y^2}{2}-yx}
			\end{eqnn}
			$\Rightarrow$ Es existiert keine partiell differenzierbare Funktion $f\in\R^2$\\ mit $\varDelta f(x,y)=\m{x+y\\-x+y}.$\\\\

			\begin{eqnn}
				\eqnf{\varDelta f(x,y,z)}{\m{yz\\xz\\xy}}
				\eqnr{f(x,y,z)}{\int yz dx}
				\eqnf{}{\int xz dy}
				\eqnf{}{\int xy dz}
				\eqnr{f(x,y,z)}{xyz}
			\end{eqnn}

		\end{enumerate}
		\item[11.4] Sei $c>0$, und $f,g: \R \to \R$ seien zweimal stetig differenzierbar. Zeige: Die Funktion $u: \R^2 \to \R$, $u(t,x)\coloneqq f(x+ct)+g(x-ct)$ ist eine Lösung der \textit{Wellengleichung}
		$$\frac{\partial^2}{\partial t^2}u(t,x)=c^2\cdot \frac{\partial^2}{\partial x^2}u(t,x)$$
		
		\begin{eqnn}
			\eqn[][]{\frac{\partial^2}{\partial t^2}u}{\frac{\partial^2}{\partial t^2}f(x+ct)c^2+\frac{\partial^2}{\partial t^2}g(x-ct)c^2}
			
			\eqn[][]{c^2\cdot \frac{\partial^2}{\partial x^2}u1}{\frac{\partial^2}{\partial x^2}f(x+ct)+\frac{\partial^2}{\partial x^2}g(x-ct)}
		\end{eqnn}
		\item[11.5] Bestimme Lage und Art der lokalen Extrema folgender Funktionen:
		\begin{enumerate}
			\item $f:\R^2 \to \R, f(x,y)\coloneqq xy^2-4xy+x^4$
			
			\begin{eqnn}
				\eqntext{Kandidaten finden:}
				\eqn[][]{\nabla f}{\m{y^2+4x+4x^3\\2xy-4x}}
				\eqn[][\Rightarrow]{0}{2xy-4x}
				\eqn{y}{2}
				\eqn[][]{0}{y^2-4x+4x^3}
				\eqntext{$y=2$ Einsetzen:}
				\eqn[][\Rightarrow]{x}{1}
				\eqn[][]{\text{Hess }f}{\m{2&2y-4\\2y-4&2x}}
				\eqntext{Kandidat Einsetzen:}
				\eqn[][\Rightarrow]{\md{2&0\\0&2}}{4}
				\eqntext{Hess f ist positiv definit da $|$Hess $f|> 0$ ist und Hess $f_{11} > 0$,
				daraus folgt das der Kandidat $\m{1\\2}$ein striktes lokales Minimum ist.}
			\end{eqnn}
			\item $g:(0,\pi)\times(0,\pi)\to \R, g(x,y) \coloneqq \sin(x) + \sin(y) + \sin(x+y)$
			\begin{eqnn}
				\eqntext{Kandidaten finden:}
				\eqn[][]{\nabla g}{\m{\cos(x)+\cos(x+y)\\ \cos(y)+\cos(x+y)}}
				\eqn[(I)][\Rightarrow]{\cos(x)}{-\cos(x+y)}
				\eqn[][\Rightarrow]{\cos(y)}{-\cos(x+y)}
				\eqn[][\Rightarrow]{x}{y}
				\eqntext{In (I) Einsetzen:}
				\eqn[][\Rightarrow]{\cos(x)}{-\cos(2x)}
				\eqn{\cos(x)}{-1(\cos(x)^2-\sin(x)^2)}[$+\cos(x)^2$]
				\eqn{\cos(x)+2\cos(x)^2}{1}[$\div 2 +\frac{1}{16}$]
				\eqn{\frac{1}{16}+\frac{\cos(x)}{2}+\cos(x)^2}{\frac{9}{16}}
				\eqn{(\frac{1}{4}+\cos(x))^2}{\frac{9}{16}}
				\eqn[][\Rightarrow]{x}{\frac{\pi}{3}}
				\eqnspace
				\eqn[][]{\text{Hess }f}{\m{-\sin(x)-\sin{2x}&-\sin(2x)\\-\sin(2x)&-\sin(x)-\sin(x+y)}}
				\eqn{\sin(x)^2}{\md{-\sin(x)&0\\0&-\sin(x)}}
				\eqntext{Kandidat Einsetzen:}
				\geqn[][\Rightarrow]{\sin(\frac{\pi}{3})}{>}{0}
				\eqntext{$\Rightarrow$ Hess $f$ ist negativ definit, daraus folgt das der Kandidat $\m{\frac{\pi}{3}}$ ein strikes lokales Maximum ist.}
			\end{eqnn}
			
		\end{enumerate}
		\item[11.6]
		\begin{enumerate}
			\item Berechne die Jacobi-Matrix der Abbildungen (Polar- bzw. Kugelkoordinaten)
			\begin{enumerate}
				\item $F:\R^2 \to \R^2,F(r,\varphi)\coloneqq(r \cos(\varphi), r\sin(\varphi))^T$
				\item $G: \R^3 \to \R^3, G(r, \theta, \varphi)\coloneqq(r \cos(\theta) \cos(\varphi), r\cos(\theta)\sin(\varphi),r\sin(\theta))^T$\\
				Gib die partiellen Ableitungen von $F_2$ und $G_3$ an, wobei $F = (F_1 , F_2 )^T$ und $G = (G_1 , G_2 , G_3 )^T$.
			\end{enumerate}
		\item Bestimme, in welchen Punkten die Jacobi-Matrizen aus (a) invertierbar sind.
		\end{enumerate}

	\end{enumerate}
\end{document}
