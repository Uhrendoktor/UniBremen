\documentclass[12pt,letterpaper]{article}
\usepackage{fullpage}
\usepackage{fancyhdr}
\usepackage{amsfonts}
\usepackage{amsmath, amssymb}
\usepackage{mathtools}
\usepackage{polynom}

% Edit these as appropriate
\newcommand\course{HM 1}
\newcommand\hwnumber{2}                  % <-- homework number
\newcommand\NetIDa{6113829}           % <-- NetID of person #1
\newcommand\NetIDb{6111554}           % <-- Matrikelnummer of person #2 (Comment this line out for problem sets)

%custom math commands
\newcommand{\alignleft}[1]{\tag*{\llap{\makebox[\linewidth][l]{$#1$}}}}

\newcommand{\LLeftrightarrow}{ \alignleft{\Leftrightarrow}}

\newcommand{\eqinfo}[1]{&\makebox[2cm][l]{$\vert #1$}}

\newcommand{\Eqn}[3]{#1 &#2 #3}
\newcommand{\eqnf}[2]{\Eqn{#1}{=}{#2}\\}
\newcommand{\eqn}[2]{\LLeftrightarrow\Eqn{#1}{=}{#2}\\}
\newcommand{\eqni}[3]{\LLeftrightarrow\Eqn{#1}{=}{#2}\eqinfo{#3}\\}
\newcommand{\eqnfi}[3]{\Eqn{#1}{=}{#2}\eqinfo{#3}\\}
\newcommand{\geqnf}[3]{\Eqn{#1}{#2}{#3}\\}
\newcommand{\geqn}[3]{\LLeftrightarrow\Eqn{#1}{#2}{#3}\\}
\newcommand{\geqni}[4]{\LLeftrightarrow\Eqn{#1}{#2}{#3}\eqinfo{#4}\\}

\newcommand{\comment}[1]{}
%end custom math commands

\pagestyle{fancyplain}
\headheight 35pt
\lhead{\NetIDa}
\lhead{\NetIDa\\\NetIDb}                 % <-- Comment this line out for problem sets (make sure you are person #1)
\chead{\textbf{\Large Übung \hwnumber}}
\rhead{\course \\ \today}
\lfoot{}
\cfoot{}
\rfoot{\small\thepage}
\headsep 1.5em

\begin{document}
\begin{enumerate}
\item [2.5] Bestimme die reellen Lösungen der Gleichungen
\begin{enumerate}
\item $\sqrt{x+1}-\sqrt{9-x}=\sqrt{2x-12}$\\
\begin{align*}
\eqnfi{\sqrt{x+1}-\sqrt{9-x}}{\sqrt{2x-12}}{^2}
\eqn{(\sqrt{x+1}-\sqrt{9-x})^2}{2x-12}
\eqn{(x+1)-2\sqrt{x+1}\sqrt{9-x}+(9-x)}{2x-12}
\eqn{x+1-x+9-2\sqrt{(x+1)(9-x)}}{2x-12}
\eqni{10-2\sqrt{(x+1)(9-x)}}{2x-12}{/-2}
\eqni{-5+\sqrt{(x+1)(9-x)}}{-x+6}{+5}
\eqni{\sqrt{(x+1)(9-x)}}{-x+11}{^2}
\eqn{(x+1)(9-x)}{(-x+11)^2}
\eqni{-x^2+8x+9}{11^2-22x+x^2}{-x^2; +22x; -11^2}
\eqni{-2x^2+30x-112}{0}{/(-2)}
\eqn{x^2-15x+56}{0}
\\
\Rightarrow x_{1/2} = -\frac{-15}{2}\pm\sqrt{(\frac{-15}{2})^2+56}
= 7.5\pm 0.5\\
\Rightarrow x_1=7 \land x_2=8
\end{align*}
\item $|x-3|+|x+2|-|x-4|=3$\\
Wenn man $|x-3|$, $|x+2|$ und $|x-4|$ als drei separate Abbildungen $f,g,h:\mathbb{R}\to\mathbb{R}$ begreift, so ergibt sich für:\\
$f(x) = \begin{cases}
x-3 &x\geq 3\\
-x-3 &x<3\\
\end{cases}$\\
$g(x) = \begin{cases}
x+2 &x\geq -2\\
-x-2 &x<-2
\end{cases}$\\
$h(x) = \begin{cases}
x-4 &x\geq 4\\
-x+4 &x<4\\
\end{cases}$\\
Wenn man $gh:\mathbb{R}\to\mathbb{R}, gh(x)\coloneqq g(x)-h(x)$ definiert, folgt:\\
$gh(x) = \begin{cases}
1 &x\geq 4\\
2x-7 &3<x<4\\
-1 &x\leq3\\
\end{cases}$\\
\newpage
Definiert man weiter $fgh:\mathbb{R}\to\mathbb{R}, fgh(x)\coloneqq gh(x)+f(x)$, folgt:\\
$fgh(x) = \begin{cases}
x+3 &x\geq 4\\
3x-5 &3<x<4\\
x+1 &-2\leq x\leq 3\\
-x-3 &x<-2\\
\end{cases}$\\
\begin{align*}
\Rightarrow x+3&=3 \Leftrightarrow x=0, &0\ngeq 4|\times\\
3x-5&=3 \Leftrightarrow x=\frac{8}{3}, &3\nless\frac{8}{3}< 4|\times\\
x+1&=3 \Leftrightarrow x=2, &-2\leq 2\leq 3|\surd\\
-x-3&=3 \Leftrightarrow x=-6, &-6<-2|\surd\\
\end{align*}
$\Rightarrow x_1 = 2 \land x_2 = -6$
\end{enumerate}
\item [2.6] bestimme sämtliche reellen Lösungen der Ungleichungen
\begin{enumerate}
\item $\dfrac{2}{x+1}<\dfrac{1}{x-3}$\\\\
Es gilt für $a,b\in\mathbb{R}_{x > 0}$:\\
$\frac{1}{a}<\frac{1}{b}\Leftrightarrow a>b$\\
$-\frac{1}{a}<-\frac{1}{b}\Leftrightarrow -a>-b$\\
Damit $\frac{2}{x+1}$ und $\frac{1}{x-3}$ beide entweder positiv oder negativ sind,\\
muss $x>-1 \land x>3 \Leftrightarrow x>3$ bzw. $x<-1 \land x<3 \Leftrightarrow x<-1$ sein.\\
Daraus ergibt sich für $x<-1 \lor x>3$:
\begin{align*}
\geqnf{\frac{2}{x+1}}{<}{\frac{1}{x-3}}
\geqn{\frac{x+1}{2}}{>}{\frac{x-3}{1}}
\geqni{\frac{1}{2}x+\frac{1}{2}}{>}{x-3}{-\frac{1}{2}x; +3}
\geqni{3.5}{>}{0.5x}{*2}
\geqn{7}{>}{x}
\geqn{x}{<}{7}
\end{align*}
Innerhalb des restlichen Intervalls $(-1,3)$ nimmt $f:\mathbb{R}\to\mathbb{R}, f(x)\coloneqq\frac{2}{x+1}$ alle Werte zwischen $(\frac{1}{2},\infty)$ und $h:\mathbb{R}\to\mathbb{R}, h(x)\coloneqq\frac{1}{x-3}$ alle Werte zwischen $(-\frac{1}{4}, -\infty)$ an.\\
Somit gilt: $\forall x\in\mathbb{R}_{-1<x<3}:(f(x)\nless h(x))$.\\\\
Insgesamt gilt: $\forall x\in\mathbb{R}_{x<7}\setminus\{-1\leq x\leq3\}:(\frac{2}{x+1}<\frac{1}{x-3})$

\comment{
\item [(a)] ORIGINAL - MORITZ\\
Da $\frac{2}{x+1} > 0$ mit $x\in\mathbb{R};x>-1$\\
Und $\frac{1}{x-3} < 0$ mit $x\in\mathbb{R};x<3$\\
Gibt gibt es keine reellen Lösungen in $x\in\mathbb{R};-1<x<3$\\
\\
Nullpunkt bestimmen:\\
\begin{align*}
	\geqnf{\frac{2}{x+1}}{<}{\frac{1}{x-3}}
	\geqni{\frac{2x-6}{x+1}}{<}{1}{*(x-3)}	
\end{align*}
Polynom Division:\\
\polylongdiv[style=C]{2x-6}{x+1}
\begin{align*}
	\geqnf{2-\frac{8}{x+1}}{<}{1}
	\geqni{1}{<}{\frac{8}{x+1}}{-1+\frac{8}{x+1}}
	\geqni{x+1}{<}{8}{x+1}
	\geqni{x}{<}{7}{-1}	
\end{align*}
Somit gilt $\forall x\in\mathbb{R};(x<7;x\notin\{-1\leq x\leq3\}):\frac{2}{x+1}<\frac{1}{x-3}$
}

\item $\left(x+2\right)\left(4-x\right)\left(x-2\right)^2>0$\\
da $x^2$ für $x\in\mathbb{R}$ immer $\geq 0$, kann der Term nur negativ werden, wenn $(x+2)$ oder $(4-x)$ negativ sind.\\
Die Bedingung erfüllen alle Elemente von $M \lbrace x\in\mathbb{R}; x\neq 2, (x+2)(4-x)>0 \rbrace$\\
\begin{align*}
\geqnf{(x+2)(4-x)}{>}{0}
\geqni{-x^2+2x+8}{>}{0}{*(-1)}
\geqn{x^2-2x-8}{<}{0}
\end{align*}
$f: \mathbb{R}\to\mathbb{R}, f(x)\coloneqq x^2-2x-8$ stellt eine nach oben geöffnete Parabel da. Somit müssen alle Werte zwischen den beiden Nullstellen $<0$ sein.\\
Die Nullstellen lassen sich direkt aus der Parameterform oben ablesen.\\
$\Rightarrow x_1=(-2) \land x_2=4$\\
Somit gilt $\forall x\in\mathbb{R};(-2<x<4, x\neq 2):(\left(x+2\right)\left(4-x\right)\left(x-2\right)^2>0)$
\end{enumerate}
\item [2.7] Beweise durch Induktion, dass für alle $n\in\mathbb{N}$ gilt:
\begin{enumerate}
\item $\sum\limits_{k=1}^{n}k^3 = \left(\sum\limits_{k=1}^{n}k\right)^2$.

IA: n=1 $\sum\limits_{k=1}^{n}k^3=1=\left(\sum\limits_{k=1}^{n}k\right)^2$\\
\\
IV: $n\to n+1$ Für alle $n\in \mathbb{N}$ gilt $\sum\limits_{k=1}^{n}k^3 = \left(\sum\limits_{k=1}^{n}k\right)^2$\\
\\
IS: 
\begin{align*}
	\eqnf{\sum_{k=1}^{n+1}k^3}{\left(\sum_{k=1}^{n+1}k\right)^2}
	\eqn{\sum_{k=1}^{n}k^3+(n+1)^3}{\left(\sum_{k=1}^{n}k+(n+1)\right)^2}\\
	&\text{IV: einsetzen}\\
	\eqn{\left(\sum_{k=1}^{n}k\right)^2+(n+1)^3}{\left(\sum_{k=1}^{n}k+(n+1)\right)^2}\\
	\sum_{k=1}^{n}k&=\frac{n}{2}(n+1)\text{: einsetzen}\\ 
	\eqn{(\frac{n}{2}(n+1))^2+(n+1)^3}{(\frac{n}{2}(n+1)+(n+1))^2}
	\eqni{(\frac{n}{2}(n+1))^2+(n+1)^3}{(\frac{n}{2}(n+1))^2+(n+1)^2+n(n+1)^2}{-(\frac{n}{2}(n+1)^2)}
	\eqni{(n+1)^3}{(n+1)^2+n(n+1)^2}{(n+1)^2}
	\eqn{n+1}{n+1}
\end{align*}

\item $\sum\limits_{k=0}^{n}q^k = \frac{1-q^{n+1}}{1-q}$ (wobei $q\in\mathbb{R}\setminus\{1\}$).\\ \\
IA: n=1
\begin{align*}
	\sum_{k=0}^{1}q^k &= \frac{1-q^{1+1}}{1-q}\\
	\sum_{k=0}^{1}q^k&=q^0+q^1=1+q\\
	\text{Polynom Division:}\\
	\frac{1-q^2}{1-q} &= (1-q^2)\div (1-q) = q+1
\end{align*}\\

IV: $\sum\limits_{k=0}^{n}q^k = \frac{1-q^{n+1}}{1-q}$ (wobei $q\in\mathbb{R}\setminus\{1\}$)\\
\\
IS:\\
\begin{align*}
	\eqnf{\sum_{k=0}^{n+1}q^k}{\frac{1-q^{n+1+1}}{1-q}}
	\eqn{\sum_{k=0}^{n}q^k+q^{n+1}}{\frac{1-q^{n+1+1}}{1-q}}
	\text{IV einsetzen:}\\
	\eqni{\frac{1-q^{n+1}}{1-q}+q^{n+1}}{\frac{1-q^{n+1+1}}{1-q}}{*(1-q)}
	\eqni{1-q^{n+1}+(1-q)*q^{n+1}}{1-q^{n+1}*q}{/(q^{n+1});-1}
	\eqn{q}{q}	
\end{align*}
\\
\item $6^{2n-2}+3^{n+1}+3^{n-1}$ ist durch 11 teilbar.\\
IA mit $n=1$: 
\begin{align*}
\eqnf{6^{2*1-2}+3^{1+1}+3^{1-1} \pmod{11}}{0}
\eqn{6^0+3^2+3^0 \pmod{11}}{0}
\eqn{1+9+1 = 11 \pmod{11}}{0\makebox[1cm]{}\surd}
\end{align*}
IS $n\to n+1$:
\begin{align*}
&6^{2(n+1)-2}+3^{n+1+1}+3^{n+1-1}\\
= &6^{2n}+3^{n+2}+3^{n}\\
= &6^2*6^{2n-2}+3*3^{n+1}+3*3^{n-1}\\ 
= &3*12*6^{2n-2}+3*3^{n+1}+3*3^{n-1}\\
= &3*(11*6^{2n-2}+6^{2n-2}+3^{n+1}+3^{n-1})\\
= &3*11*6^{2n-2}+ 3*(6^{2n-2}+3^{n+1}+3^{n-1})\\
\end{align*}
Da $3*11*6^{2n-2}$ und wie im Induktionsanfang gezeigt $6^{2n-2}+3^{n+1}+3^{n-1}$ und somit auch $3*(6^{2n-2}+3^{n+1}+3^{n-1})$ Vielfache von 11 sind, muss die Gesamtsumme ebenfalls für alle $n\in\mathbb{N}$ durch 11 teilbar sein.
\end{enumerate}
\item [2.8]
\begin{enumerate}
\item Berechne $|5+12i|$\\
$|5+12i| = \sqrt(5^2+12^2) = \sqrt{25+144} = \sqrt{169} = 13$
\item Berechne $\sum\limits_{k=2}^{4}(2i)^k$\\
$\sum\limits_{k=2}^{4}(2i)^k = (2i)^2+(2i)^3+(2i)^4 = -4-6i+16 = 12-6i$
\item Bestimme Real und Imaginärteil von $\cfrac{1+i\sqrt{2}}{1-i\sqrt{2}}+\cfrac{1-i\sqrt{2}}{1+i\sqrt{2}}$.\\
$z = 1+i\sqrt{2} = a+ib,\\
\bar{z} = 1-i\sqrt{2} = a-ib$
\begin{align*}
\frac{z}{\bar{z}}+\frac{\bar{z}}{z} = \frac{z^2+\bar{z}^2}{z\bar{z}} = \frac{a^2+2iab-b^2+a^2-2iab-b^2}{a^2+b^2} = \frac{2a^2-2b^2}{a^2+b^2} = \frac{2(a^2-b^2)}{a^2+b^2}\\
\Rightarrow\frac{2(1^2-\sqrt{2}^2)}{1^2+\sqrt{2}^2}=\frac{2-4}{1+2} = -\frac{2}{3}
\end{align*}
Re: $-\frac{2}{3}$,\\
Im: $0$
\item Bestimme Real und Imaginärteil von $\dfrac{1+i^{15}}{2-i^{21}}$.
\begin{align*}
\frac{1+i^{15}}{2-i^{21}} = \frac{1+(-1)^{7.5}}{2-(-1)^{10.5}} = \frac{1+(-1)^7*(-1)^{\frac{1}{2}}}{2-(-1)^{10}*(-1)^{\frac{1}{2}}} = \frac{1+(-1)*\sqrt{-1}}{2-(1)*\sqrt{-1}} = \frac{1-i}{2-i}\\
=\frac{(1-i)(2+i)}{(2-i)(2+i)} = \frac{2-i}{5} = 0.4+0.2i
\end{align*}
Re: $0.4$,\\
Im: $0.2$
\end{enumerate}
\end{enumerate}
\end{document}