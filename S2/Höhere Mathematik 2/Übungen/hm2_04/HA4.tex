\documentclass{HM}
\newcommand\course{HM 2}
\newcommand\hwnumber{4}

\newcommand{\Eig}{\text{Eig}}
\newcommand{\Span}{\text{Span}}

\begin{document}
	\begin{enumerate}
	\item[4.2] Sei $n\in\N$ ungerade und $A\in\R^{n\times n}$ eine \textit{schiefsymmetrische} Matrix, d.h. $A^T=-A$. Zeige, dass $A$ nicht invertierbar ist. Gilt dies auch für gerade $n$? Gib einen Beweis oder ein Gegenbeispiel an.\\\\
		$A$ invertierbar $\Leftrightarrow \det(A)\not=0$
		Für $n$ ungerade:\\
		\begin{align*}
			\det(A)&=\det(A^T)=\det(-A)\\
			\intertext{Aus jeder Zeile von $-A$ wird der Faktor $-1$ gezogen:}
			\det(A)&=(-1)^n\det(A)
		\end{align*}
		Für ungerade $n$ folgt: $\det(A)=-\det(A)$.\\
		$\Rightarrow \det(A)=0$\\
		$\Rightarrow A$ nicht invertierbar für ungerade $n$\\\\
		Für $n$ gerade (Gegenbeispiel):
		$$\begin{vmatrix}
			0&1\\
			1&0
		\end{vmatrix} = -1$$
		$\Rightarrow$ invertierbar.
	
	\newpage
	\item[4.3]
		\begin{enumerate}
			\item Berechne\\
			$$A\coloneqq\begin{vmatrix}
				1&0&\hdots&0&1\\
				0&2&\ddots&\vdots&2\\
				\vdots&\ddots&\ddots&0&\vdots\\
				0&\hdots&0&n-1&n-1\\
				1&2&\hdots&n-1&n
			\end{vmatrix}$$
			
			$$C^{(m,n)}\coloneqq\begin{pmatrix}
				1&0&\hdots&1\\
				0&1&\hdots&1\\
				\vdots&\vdots&\ddots&\vdots\\
				\tfrac{m}{n}&\tfrac{m+1}{n}&\hdots&\tfrac{n}{n}
			\end{pmatrix}, a\coloneqq n-m+1; a\in\N\setminus\{1\}; C^{(m,n)}\in \C^{a\times a}$$
			\begin{align*}
				A&=\begin{vmatrix}
					1&0&\hdots&0\\
					0&2&\hdots&0\\
					\vdots&\vdots&\ddots&\vdots\\
					0&0&\hdots&n\\
				\end{vmatrix}\cdot \det(C^{(1,n)})\\
				&=n!\cdot \det(C^{(1,n)})
			\end{align*}	
			\begin{align*}
				\intertext{Entwickeln der $1.$ Spalte von $C$ ergibt:}
				\det(C^{(m,n)})&=(-1)^{n+1}\frac{m}{n}\cdot\begin{vmatrix}
					0&0&\hdots&0&1\\
					1&0&\hdots&0&1\\
					\vdots&\ddots&\ddots&\vdots&\vdots\\
					0&\hdots&1&0&1\\
					0&\hdots&0&1&1
				\end{vmatrix}+1\cdot \det(C^{(m+1,n)})\\
				\intertext{Entwickeln der $1.$ Zeile ergibt:}
				&=(-1)^{a+1}\frac{m}{n}(-1)^{a}\det(E_{a-2})+\det(C^{(m+1,n)})\\
				&=-\frac{m}{n}+\det(C^{(m+1,n)})\\
				&=-\frac{m}{n}-\frac{m+1}{n}-\hdots-\frac{n-1}{n}+\begin{vmatrix}
					\tfrac{n}{n}
				\end{vmatrix}\\
				&=-\frac{\sum\limits_{k=m}^{n-1}k}{n}+1\\
				&=-\frac{n(n-1)-m(m-1)}{n}+1\\
				\Rightarrow \det(C^{(1,n)}) &= \frac{3-n}{2}
			\end{align*}
			$\Rightarrow \det(A)=n!\frac{3-n}{2}$
			
			\item Finde Matrizen $A,B,$ so dass $\det(A+B)\not=\det(A)+\det(B)$\\
			\begin{align*}
				A=\begin{vmatrix}
					1&0\\
					0&2\\
				\end{vmatrix},
				B=\begin{vmatrix}
					3&0\\
					0&4\\
				\end{vmatrix}\\
			\det(A+B)  = \begin{vmatrix}
				2&0\\
				0&6\\
			\end{vmatrix}= 24\\
			\det(A) + \det(B) = 2 + 12 = 14\\
			\end{align*}\\
		
		\end{enumerate}
		
	\item[4.4] Berechne die \textit{Vandermonde-Determinante}\\
		$$\begin{vmatrix}
			1&x_1&x_1^2&\hdots&x_1^{n-1}\\
			1&x_2&x_2^2&\hdots&x_2^{n-1}\\
			\vdots&\vdots&\vdots&&\vdots\\
			1&x_n&x_n^2&\hdots&x_n^{n-1}
		\end{vmatrix}$$
		für $n\in\N$ und $x_1,\hdots,x_n\in\C$.\\
		$V^{(m,n)}\coloneqq\begin{vmatrix}
			1&x_m&x_m^2&\hdots&x_m^{n-m}\\
			1&x_{m+1}&x_{m+1}^2&\hdots&x_{m+1}^{n-m}\\
			\vdots&\vdots&\vdots&&\vdots\\
			1&x_n&x_n^2&\hdots&x_n^{n-m}
		\end{vmatrix}$ mit $m<n; m\in\N$
		\begin{align*}
			&=\det\begin{pmatrix}1&v_1&v_2&\hdots&v_{n-1})\end{pmatrix}\\
			\intertext{beginnend bei $v_{n-1}$ wird $v_a-x_mv_{a-1}$ subtrahiert:}
			&=\det\begin{pmatrix}1&v_1-x_m\cdot 1&v_2-x_m\cdot v_1&\hdots&v_{n-m}-x_m\cdot v_{n-(m+1)}\end{pmatrix}\\
			\intertext{$x_a^{n-b}-x_m\cdot x_a^{n-b-1} = x_a^{n-b-1}(x_a-x_m)$}
			&=\begin{vmatrix}
				1&0&0&\hdots&0\\
				1&x_{m+1}-x_m&x_{m+1}(x_{m+1}-x_m)&\hdots&x_{m+1}^{n-(m+1)}(x_{m+1}-x_m)\\
				\vdots&\vdots&\vdots&&\vdots\\
				1&x_n-x_m&x_n(x_n-x_m)&\hdots&x_n^{n-(m+1)}(x_n-x_m)
			\end{vmatrix}\\
			\intertext{Entwickeln der ersten Zeile:}
			&=\begin{vmatrix}
				x_{m+1}-x_m&x_{m+1}(x_{m+2}-x_m)&\hdots&x_{m+1}^{n-(m+1)}(x_{m+1}-x_m)\\
				\vdots&\vdots&&\vdots\\
				x_n-x_m&x_n(x_n-x_m)&\hdots&x_n^{n-(m+1)}(x_n-x_m)
			\end{vmatrix}\\
			&=\prod\limits_{k=m+1}^n(x_k-x_m)\begin{vmatrix}
				1&x_{m+1}&\hdots&x_{m+1}^{n-(m+1)}\\
				\vdots&\vdots&&\vdots\\
				1&x_n&\hdots&x_n^{n-(m+1)}
			\end{vmatrix}\\
			&=\prod\limits_{k=m+1}^n(x_k-x_m)V^{(m+1, n)}\\
			&=\prod\limits_{k_1=m+1}^n\left(\prod\limits_{k_2=k_1}^n(x_k-x_m)\right)
		\end{align*}
		
	\item[4.5] Berechne das charakteristische Polynom, die Eigenwerte und Eigenvektoren von
	$$A=\begin{pmatrix*}[r]
		2&2&3\\
		1&2&1\\
		2&-2&1
	\end{pmatrix*} \text{ und } B=\begin{pmatrix*}[r]
		-5&0&7\\
		6&2&-6\\
		-4&0&6
	\end{pmatrix*}$$
	\begin{align*}
		\intertext{A:}
		p_A(\lambda)=&\det(A -\lambda E)\\
		=&\begin{vmatrix}
			2-\lambda&2&3\\
			1&2-\lambda&1\\
			2&-2&1-\lambda
		\end{vmatrix}\\
		=&-\begin{vmatrix}
			2&3\\
			-2&1-\lambda
		\end{vmatrix}+(2-\lambda)\begin{vmatrix}
			2-\lambda&3\\
			2&1-\lambda
		\end{vmatrix}-\begin{vmatrix}
			2-\lambda&2\\
			2&-2
		\end{vmatrix}\\
	\end{align*}
	\begin{align*}
		=&-2+2\lambda-6+(2-\lambda)((2-\lambda)(1-\lambda)-6)+4-2\lambda+4\\
		=&-4\lambda+(-4-\lambda)(2-3\lambda+\lambda^2)\\
		=&-\lambda^3+5\lambda^2-2\lambda-8\\
		\Rightarrow\lambda_1=&-1,\\
		\lambda_2=& 2,\\
		\lambda_3=& 4\\
	\end{align*}
	
	$$\Eig(A,\lambda)\coloneqq (A-\lambda E_n)x=0$$
	\begin{minipage}{.33\textwidth}
		\begin{align*}
		\intertext{$\Eig(A, \lambda_1)$:}
		\begin{pmatrix}
			3&2&3\\
			1&3&1\\
			2&-2&2
		\end{pmatrix}x=\begin{pmatrix}
			0\\0\\0
		\end{pmatrix}\\
		\intertext{Gauß-Algorithmus:}
		\Rightarrow\begin{pmatrix}
			3&2&3\\
			0&7&0\\
			0&0&0
		\end{pmatrix}\\
		\Rightarrow\Span\left\lbrace\begin{pmatrix}
			1\\
			0\\
			-1
		\end{pmatrix}\right\rbrace
		\end{align*}
	\end{minipage}
	\begin{minipage}{.33\textwidth}
		\begin{align*}
		\intertext{$\Eig(A, \lambda_2)$:}
		\begin{pmatrix}
			0&2&3\\
			1&0&1\\
			2&-2&-1
		\end{pmatrix}x=\begin{pmatrix}
			0\\0\\0
		\end{pmatrix}\\
		\intertext{}
		\Rightarrow\begin{pmatrix}
			2&-2&-1\\
			0&2&3\\
			0&0&0
		\end{pmatrix}\\
		\Rightarrow\Span\left\lbrace\begin{pmatrix}
			-2\\
			-3\\
			2
		\end{pmatrix}\right\rbrace
		\end{align*}
	\end{minipage}
	\begin{minipage}{.33\textwidth}
		\begin{align*}
		\intertext{$\Eig(A, \lambda_3)$:}
		\begin{pmatrix}
			-2&2&3\\
			1&-2&1\\
			2&-2&-3
		\end{pmatrix}x=\begin{pmatrix}
			0\\0\\0
		\end{pmatrix}\\
		\intertext{}
		\Rightarrow\begin{pmatrix}
			-2&2&3\\
			0&-2&5\\
			0&0&0
		\end{pmatrix}\\
		\Rightarrow\Span\left\lbrace\begin{pmatrix}
			8\\
			5\\
			2
		\end{pmatrix}\right\rbrace
		\end{align*}
		\end{minipage}
		$\Rightarrow \Eig(A,\lambda)=\Span\left\lbrace\begin{pmatrix}
			1\\0\\-1
		\end{pmatrix},\begin{pmatrix}
			-2\\-3\\2
		\end{pmatrix},\begin{pmatrix}
			8\\5\\2
		\end{pmatrix}\right\rbrace$
		
		\newpage
		\begin{align*}
			\intertext{B:}
			p_B(\lambda)&=\det(B-\lambda E)\\
			=&\begin{vmatrix}
				-5-\lambda&0&7\\
				6&2-\lambda&-6\\
				-4&0&6-\lambda\\
			\end{vmatrix}\\
			=&(2-\lambda)\begin{vmatrix}
				-5-\lambda&7\\
				-4&6-\lambda\\
			\end{vmatrix}\\
			=&(\lambda-2)(-\lambda^2+\lambda+2)\\
			\Rightarrow\lambda_1=\lambda_2=&2,\\
			\lambda_3=&-1
		\end{align*}
		$$\Eig(B,\lambda)\coloneqq (B-\lambda E_n)x=0$$
		\begin{minipage}{.5\textwidth}
			\begin{align*}
			\intertext{$\Eig(B, \lambda_1)=\Eig(B, \lambda_2)$:}
			\begin{pmatrix}
				-7&0&7\\
				6&0&-6\\
				-4&0&4
			\end{pmatrix}x=\begin{pmatrix}
				0\\0\\0
			\end{pmatrix}\\
			\intertext{}
			\Rightarrow\begin{pmatrix}
				-1&0&1\\
				0&0&0\\
				0&0&0
			\end{pmatrix}\\
			\Rightarrow\Span\left\lbrace\begin{pmatrix}
				1\\
				0\\
				1
			\end{pmatrix},\begin{pmatrix}
				0\\
				1\\
				0
			\end{pmatrix}\right\rbrace
			\end{align*}
		\end{minipage}
		\begin{minipage}{.5\textwidth}
			\begin{align*}
			\intertext{$\Eig(B, \lambda_3)$:}
			\begin{pmatrix}
				-4&0&7\\
				6&3&-6\\
				-4&0&7
			\end{pmatrix}x=\begin{pmatrix}
				0\\0\\0
			\end{pmatrix}\\
			\intertext{}
			\Rightarrow\begin{pmatrix}
				-4&0&7\\
				0&2&3\\
				0&0&0
			\end{pmatrix}\\
			\Rightarrow\Span\left\lbrace\begin{pmatrix}
				-7\\
				-6\\
				4
			\end{pmatrix}\right\rbrace
			\end{align*}
		\end{minipage}
		$\Rightarrow \Eig(B,\lambda)=\Span\left\lbrace\begin{pmatrix}
			1\\0\\1
		\end{pmatrix},\begin{pmatrix}
			0\\1\\0
		\end{pmatrix},\begin{pmatrix}
			-7\\-6\\4
		\end{pmatrix}\right\rbrace$
	\end{enumerate}
\end{document}
