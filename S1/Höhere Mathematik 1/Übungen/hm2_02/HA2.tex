\documentclass[12pt,letterpaper]{article}
\usepackage{fullpage}
\usepackage{fancyhdr}
\usepackage{amsfonts}
\usepackage{amsmath}
\usepackage{mathtools}

% Edit these as appropriate
\newcommand\course{HM 1}
\newcommand\hwnumber{1}                  % <-- homework number
\newcommand\NetIDa{6113829}           % <-- NetID of person #1
\newcommand\NetIDb{6111554}           % <-- Matrikelnummer of person #2 (Comment this line out for problem sets)

%custom math commands
\newcommand{\alignleft}[1]{\tag*{\llap{\makebox[\linewidth][l]{$#1$}}}}

\newcommand{\LLeftrightarrow}{ \alignleft{\Leftrightarrow}}

\newcommand{\eqinfo}[1]{&\makebox[2cm][l]{$\vert #1$}}

\newcommand{\Eqn}[3]{#1 &#2 #3}
\newcommand{\eqnf}[2]{\Eqn{#1}{=}{#2}\\}
\newcommand{\eqn}[2]{\LLeftrightarrow\Eqn{#1}{=}{#2}\\}
\newcommand{\eqni}[3]{\LLeftrightarrow\Eqn{#1}{=}{#2}\eqinfo{#3}\\}
\newcommand{\geqnf}[3]{\Eqn{#1}{#2}{#3}\\}
\newcommand{\geqn}[3]{\LLeftrightarrow\Eqn{#1}{#2}{#3}\\}
\newcommand{\geqni}[4]{\LLeftrightarrow\Eqn{#1}{#2}{#3}\eqinfo{#4}\\}
%end custom math commands

\pagestyle{fancyplain}
\headheight 35pt
\lhead{\NetIDa}
\lhead{\NetIDa\\\NetIDb}                 % <-- Comment this line out for problem sets (make sure you are person #1)
\chead{\textbf{\Large Übung \hwnumber}}
\rhead{\course \\ \today}
\lfoot{}
\cfoot{}
\rfoot{\small\thepage}
\headsep 1.5em

\begin{document}
\begin{enumerate}
\item [2.5] Bestimme die reelen Lösungen der Gleichungen
\begin{enumerate}
\item $\sqrt{x+1}-\sqrt{9-x}=\sqrt{2x-12}$
\item $|x-3|+|x+2|-|x-4|=3$
\end{enumerate}
\item [2.6] bestimme sämtliche reelen Lösungen der Ungleichungen
\begin{enumerate}
\item $\frac{2}{x+1}<\frac{1}{x-3}$\\
Da $\frac{2}{x+1} > 0$ mit $x\in\mathbb{R};x>-1$\\
Und $\frac{1}{x-3} < 0$ mit $x\in\mathbb{R};x<3$\\
Gibt gibt es keine reelen Lösungen in $x\in\mathbb{R};-1<x<3$\\
\\
Nullpunkt bestimmen:\\
\begin{align*}
	\geqnf{\frac{2}{x+1}}{<}{\frac{1}{x-3}}
	\geqni{\frac{2x-6}{x+1}}{<}{1}{*(x-3)}	
\end{align*}
\begin{align*}
	\text{Polynom division: }\\
	(2x&-6)\div(x+1)=2-\frac{8}{x+1}\\
	-2x&-2\\
	&-8
\end{align*}
\begin{align*}
	\geqnf{2-\frac{8}{x+1}}{<}{1}
	\geqni{1}{<}{\frac{8}{x+1}}{-1+\frac{8}{x+1}}
	\geqni{x+1}{<}{8}{x+1}
	\geqni{x}{<}{7}{-1}	
\end{align*}
Somit gilt $\forall x\in\mathbb{R};(x<7;x\notin\{-1\leq x\leq3\}):\frac{2}{x+1}<\frac{1}{x-3}$
\item $\left(x+2\right)\left(4-x\right)\left(x-2\right)^2>0$\\
da $x^2$ für $x\in\mathbb{R}$ immer $\geq 0$, kann der Term nur negativ werden, wenn $(x+2)$ oder $(4-x)$ negativ sind.\\
Die Bedingung erfüllen alle Elemente von $M \lbrace x\in\mathbb{R}; x\neq 2, (x+2)(4-x)>0 \rbrace$\\
\begin{align*}
\geqnf{(x+2)(4-x)}{>}{0}
\geqni{-x^2+2x+8}{>}{0}{*(-1)}
\geqn{x^2-2x-8}{<}{0}
\end{align*}
$f: \mathbb{R}\to\mathbb{R}, f(x)\coloneqq x^2-2x-8$ stellt eine nach oben geöffnete Parabel da. Somit müssen alle Werte zwischen den beiden Nullstellen $<0$ sein.\\
Die Nullstellen lassen sich direkt aus der Parameterform oben ablesen.\\
$\Rightarrow x_1=(-2) \land x_2=4$\\
Somit gilt $\forall x\in\mathbb{R};(-2<x<4, x\neq 2):(\left(x+2\right)\left(4-x\right)\left(x-2\right)^2>0)$
\end{enumerate}
\end{enumerate}
\end{document}