\documentclass{pi1}
\usepackage{listings, amsmath}

\begin{document}
\maketitle{3}{Rodrigue Wete Nguempnang}{Moritz Salger [611554]}
\section{Einmal mit Klasse}
"NPC" ist die Klasse mit den drei Attributen für diese Aufgabe
\lstinputlisting[firstnumber=6, firstline=6, lastline=9]{code/NPC.java}
\begin{tabular}{ll}
\textit{avatar} &die Referenz zum \textit{GameObject}.\\
\textit{patrolDistance} &die Länge des Laufweges vom \textit{NPC}.\\
\textit{progress} &die Anzahl der Schritte des Laufweges, die der \textit{NPC} beim Start schon zurück gelegt hat.
\end{tabular}\\\\
Da \textit{patrolDistance} und \textit{avatar} des \textit{NPCs} nach der initialisierung nicht verändert werden, sind diese \textit{final}.

\section{Konstruktivismus}
\lstinputlisting[firstnumber=12, firstline=12, lastline=20]{code/NPC.java}
Im Konstrukter werden die drei Attribute aus Aufgabe 1. initialisiert,\\ die Werte hierfür werden als Parameter entgegengenommen\\\\
Zusätzlich wird geprüft ob auserhalb 1..patrolDistance liegt und wenn dies der Fall ist auf 1 gesetzt
\newpage

\section{Schritt für Schritt}
\lstinputlisting[firstnumber=22, firstline=22, lastline=30]{code/NPC.java}
In der Funktion \textit{act} wird die \textit{progress} Variable um $1$ erhöt, darauf hin wird überprüft ob das Ende der Laufdistanze ereicht wurde. Falls dies der Fall ist wird progress auf $1$ gesetzt und die \textit{returning} Variable wird invertiert.\\
Am Ende wir noch die \textit{Move} Function mit der korrekten Richtung aufgerufen um die Bewegung durch zu führen.\\
\lstinputlisting[firstnumber=32, firstline=32, lastline=51]{code/NPC.java}
Hier werden locale \textit{x} und \textit{y} Variablen erstellt und auf die aktuelle \textit{NPC} position gesetzt. Darauf hin werden diese Abhängig von der \textit{direction} Variable verändert, worauf hin \textit{avatar.setLocation} mit diesen Werten aufgerufen wird um den \textit{NPC} zu bewegen.\\
Die Werte werden durch eine \textit{MinMax} Funktion auf das Spielfeld beschränkt damit \textit{NPCs} nie aus dem Spielfeld laufen können.

\section{Und Action!}
\lstinputlisting[firstnumber=74, firstline=74, lastline=76]{code/RPGGame.java}
Als erstes werden zwei \textit{NPCs} erstellt, und deren \textit{act} Methode wird nach jedem Spieler Zug aufgerufen.
\lstinputlisting[firstnumber=113, firstline=113, lastline=114]{code/RPGGame.java}
\newpage

\section{Bonusaufgabe}
\lstinputlisting[firstnumber=114, firstline=114, lastline=119]{code/RPGGame.java}
Nachdem sich ein \textit{NPC} bewegt hat, wird überprüft ob die neue Position mit der des Spieler übereinstimmt. Falls dies der Fall sein sollte, wird die Spielfigur unsichtbar gemacht und das Spiel beendet.
\end{document}