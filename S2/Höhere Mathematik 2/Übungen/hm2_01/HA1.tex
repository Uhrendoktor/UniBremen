\documentclass{HM}
\newcommand\course{HM 2}
\newcommand\hwnumber{1}   

\begin{document}
	\begin{enumerate}
		\item[1.5] Sei $f:\R\to\R, f(x)\coloneqq\sin x$. Für $n\in\N$ bestimme das Taylor-Polynom von $f$ vom Grad $n$ um den Entwicklungspunkt $a=0$. Zeige, dass die Taylor-Reihe von $f$ auf ganz $\R$ gegen $f$ konvergiert, d.h. für jedes (feste) $x\in\R$ gilt $R_n(x)\to 0$. Benutze dazu die Formel für das Lagrange-Restglied $R_n$ aus HM1, Satz 21.1.
		
		\item[1.6] Sind die angegebenen Funktionen $\varphi_k:[0,2]\to\R(k=1,2,3,4)$ Treppenfunktionen? Wenn ja, ist ihr Integral zu ermitteln.
		\begin{enumerate}
			\item $\varphi_1(x)=\floor{x}$
				$$\varphi_1(x)=\floor{x}=\begin{cases}
					\vdots& \vdots\\
					0&0\leq x < 1\\
					1&1\leq x < 2\\
					2&2\leq x < 3\\
					\vdots& \vdots\\
					n&n\leq x < n+1				
				\end{cases}$$
				$\Rightarrow \varphi_1\in T$ ($\varphi_1$ ist eine Treppenfunktion)\\
				\begin{align*}
					\Rightarrow &\int_a^b\varphi_1(x)dx = \int_0^b\varphi_1(x)dx-\int_0^a\varphi_1(x)dx\\
					&\int_0^c\varphi_1(x)dx\\
					=&\frac{c^2}{2}-\frac{\floor{c}}{2}-\frac{(c-\floor{c})^2}{2}\\
					=&\frac{c^2-\floor{c}-(c-\floor{c}^2)}{2}\\
					=&\frac{\floor{c}(2c-1)-\floor{c}^2}{2}\\
					\Rightarrow &\int_a^b\varphi_1(x)dx\\
					=&\frac{\floor{b}(2b-1)-\floor{b}^2-\floor{a}(2a-1)+\floor{a}^2}{2}
				\end{align*}
			\item $\varphi_2(x)=\floor{2x}$
				$$\varphi_2(x)=\varphi_1(2x)$$
				$\Rightarrow \varphi_2\in T$
				$$\Rightarrow \int_a^b\varphi_2(x)dx=\int_a^b\varphi_1(2x)dx=\frac{\int_{2a}^{2b}\varphi_1(x)dx}{2}$$
			\item $\varphi_3(x)=7\floor{x}-5\floor{2x}$
				$$\varphi_3(x)=7\varphi_1(x)-5\varphi_2(x)=7\varphi_1(x)-5\varphi_1(2x)$$
				$\Rightarrow \varphi_3\in T$ da sie sie eine lineare Kombination aus den beiden Treppenfunktionen $\varphi_1$ und $\varphi_2$ ist. (Vektorraum)
				$$\Rightarrow \int_a^b\varphi_3(x)dx=
				7\int_a^b\varphi_1(x)dx+5\int_a^b\varphi_1(2x)dx=
				7\int_a^b\varphi_1(x)dx+\frac{5}{2}\int_{2a}^{2b}\varphi_1(x)dx$$
			\item $\varphi_4(x)=\begin{cases}
				0&\text{falls }x=0,\\
				\floor{\frac{1}{x}}&\text{falls }x\neq 0.			
			\end{cases}$
			$$\forall x\in\R\setminus[-1,1]:\floor*{\frac{1}{x}}<1$$
			$$\Rightarrow \int_a^b\varphi_4(x)dx = 0 \text{ für } x\in\R\setminus[-1,1]$$
			Für $x\in[0,1]$ wird die Breite der Treppenstufen für $x\to 0$ zu $0$.\\
			$\Rightarrow \varphi_4\not\in T$
		\end{enumerate}
		
		\item[1.7] Die rationalen Zahlen im Intervall $[0,2)$ seien als Folge $(r_n)_{k\in\N}$ geschrieben. Entscheide, ob die angegebenen Funktionen $f_n:[0,2]\to\R(m=1,2,3,4)$ Riemann-integrierbar sind.
		\begin{enumerate}
			\item $f_1(x)=\floor{2x}$;\\
			Laut 1.6 ist $\floor{2x}$ eine Treppenfunktion $=> \floor{2x}$ ist R-intbar\\
			\item $f_2(x)=e^{-x^{2}}$;\\
			$e^x$ ist stetig und $-x²$ ist stetig $=>e^{-x^2}$ stetig. Stetige Funktionen sind R-intbar also ist $f_2$ R-intbar. 
			
			\item $f_3(x)=\sum\limits_{k:r_k<x}2^{-k}$;
			
			\item $f_4(x)=\begin{cases}
				0&\text{falls }x=0,\\
				x^{-2}&\text{falls }x\neq 0.			
			\end{cases}$\\
			Wenn man den Punkt $x=0$ ignoriert, da einzelne Punkte für Integrale irrelevant sind, bleibt der $\frac{1}{x^2}$ Teil der monoton fallend ist $=>f_4$ R-intbar. 
		\end{enumerate}
		
		\item[1.8] Sei $a>1$. gehe ähnlich wie in Aufgabe 1.4 vor, um das Riemann-Integral $\int_1^a\frac{dx}{x}$ zu bestimmen.
	\end{enumerate}
\end{document}