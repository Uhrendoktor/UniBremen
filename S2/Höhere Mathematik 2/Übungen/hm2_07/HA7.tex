\documentclass{HM}
\newcommand\course{HM 2}
\newcommand\hwnumber{7}

\begin{document}
	\begin{enumerate}
		\item[7.2] Löse die Anfangswertprobleme
		\begin{enumerate}
			\item $y'+y\sin(x)=4x^3e^{\cos(x)}, y(\frac{\pi}{2})=2;$
			\begin{eqnn}
				\eqnf[(H)]{y'+y\sin(x)}{0}
				\eqn[][\Rightarrow]{y_{(H)}(x)}{e^{\cos(x)}}
				\eqnspace
				\eqnf{y}{c\cdot y_{(H)}}
				\eqn[(I)][\Rightarrow]{y(x)}{c(x)e^{\cos(x)}}[ableiten]
				\eqn[(II)][\Rightarrow]{y'(x)}{c'(x)e^{\cos(x)}-c(x)e^{\cos(x)}\sin(x)}
				\eqntext{I und II in Ausgangsgleichung einsetzen:}
				\eqnf{c'(x)}{4x^3}
				\eqn[][\Rightarrow]{c(x)}{x^4+c}
				\eqntext{Einsetzen in I:}
				\eqn[][\Rightarrow]{y(x)}{(x^4+c)e^{\cos(x)}}
				\eqntext{Bedingung $y(\frac{\pi}{2})=2:$}
				\eqn{c}{2-(\frac{\pi}{2})^4}
				\eqn[][\Rightarrow]{y(x)}{(x^4+2-(\frac{\pi}{2})^4)e^{\cos(x)}}
			\end{eqnn}\\			
			
			\item $xy'+y+xe^x=0, y(1)=0.$
			\begin{eqnn}
				\eqnf[(H)]{xy'+y}{0}
				\eqn[][\Rightarrow]{y_{(H)}(x)}{e^{-\ln(x)}=\frac{1}{x}}
				\eqnspace
				\eqnf{y}{c\cdot y_{(H)}}
				\eqn[(I)][\Rightarrow]{y(x)}{c(x)\frac{1}{x}}[ableiten]
				\eqn[(II)][\Rightarrow]{y'(x)}{c'(x)\frac{1}{x}-c(x)\frac{1}{x^2}}
				\eqntext{I und II in Ausgangsgleichung einsetzen:}
				\eqnf{c'(x)}{-xe^x}
				\eqntext{partielle Integration mit $f(x)=-x, g(x)=e^x$}
				\eqn[][\Rightarrow]{c(x)}{(1-x)e^x+c}
				\eqntext{Einsetzen in I:}
				\eqn[][\Rightarrow]{y(x)}{\frac{1-x}{x}e^x+\frac{c}{x}}
				\eqntext{Bedingung $y(1)=0:$}
				\eqn{c}{0}
				\eqn[][\Rightarrow]{y(x)}{\frac{1-x}{x}e^x}
			\end{eqnn}\\
		\end{enumerate}
		
		\item[7.3] Bestimme diejenigen Lösungen der Differentialgleichung
		$$\dot{x}\cdot\sin(2t)=2x+2\cos(t),$$\\
		die für $t\to\frac{\pi}{2}$ beschränkt ist.
		\begin{eqnn}
				\eqnf[(H)]{\dot{x}\sin(2t)-2x}{0}
				\eqn[][\Rightarrow]{x_{(H)}(t)}{e^{\int\frac{2}{\sin(2t)}dt}=\tan(t)}
				\eqntext{Nachweis über Ableitung (Quotientenregel):}
				\eqnf{\frac{d}{dt}\tan(t)}{\frac{\cos^2(t)+\sin^2(t)}{\cos^2(t)} = 1+\tan^2(t)}
				\eqnspace
				\eqnf{x}{c\cdot x_{(H)}}
				\eqn[(I)][\Rightarrow]{x(t)}{c(t)\tan(t)}[ableiten]
				\eqn[(II)][\Rightarrow]{\dot{x}(t)}{c'(t)\tan(t)+c(t)(1+\tan^2(t))}
				\eqntext{I und II in Ausgangsgleichung einsetzen:}
				\eqnf{c'(t)}{\frac{\cos(t)}{\sin^2(t)}}
				\eqn[][\Rightarrow]{c(t)}{-\frac{1}{\sin^2(t)}+c}
				\eqntext{Nachweis über Ableitung (Quotientenregel):}
				\eqnf{\frac{d}{dx}\frac{-1}{\sin(x)}+c}{\frac{\cos(x)}{\sin^2(x)}}
				\eqntext{Einsetzen in I:}
				\eqn[][\Rightarrow]{x(t)}{(-\frac{1}{\sin(t)}+c)\tan(t)}
			\end{eqnn}\\
			Da $\lim_{t\to\frac{\pi}{2}}x(t)=\lim_{t\to\frac{\pi}{2}}(-1+c)\tan(t)$, folgt aus l'Hospital, dass $x$ nur dann für $t\to\frac{\pi}{2}$ beschränkt ist, wenn $-1+c=0\Leftrightarrow c=1$.\\
			$\Rightarrow$ Menge aller Lösungen der Differentialgleichung für $t\to\frac{\pi}{2}$ beschränkt\\ $= \{(-\frac{1}{\sin(t)}+1)\tan(t)\}$
		
		\item[7.4] Bestimme die Lösungen der Differentialgleichungen
		$$y'=\frac{x+y}{x} \quad\text{und}\quad y'=2\frac{y}{x}$$\\
		\begin{enumerate}
			\item durch Betrachten als lineare Differentialgleichung 1. Ordnung.\\
			$y'=\frac{x+y}{x}:$
			\begin{eqnn}
				\eqnf[(H)]{y'-\frac{1}{x}y}{0}
				\eqn[][\Rightarrow]{y_{(H)}(x)}{e^{\int\frac{1}{x}dx}=x}
				\eqnspace
				\eqnf{y}{c\cdot y_{(H)}}
				\eqn[(I)][\Rightarrow]{y(x)}{c(x)x}[ableiten]
				\eqn[(II)][\Rightarrow]{y'(x)}{c'(x)x+c(x)}
				\eqntext{I und II in Ausgangsgleichung einsetzen:}
				\eqnf{c'(x)}{\frac{1}{x}}
				\eqn[][\Rightarrow]{c(x)}{\ln(x)+c}
				\eqntext{Einsetzen in I:}
				\eqn[][\Rightarrow]{y(x)}{\ln(x)x+cx}
			\end{eqnn}\\	\\
			
			$y'=2\frac{y}{x}:$
			\begin{eqnn}
				\eqn{y'}{\frac{2}{x}y}
				\eqn[][\Rightarrow]{y(x)}{e^{\int\frac{2}{x}dx}=e^{2(\ln(x)+c)}=e^{\ln(x)^2}=x^2\cdot c_2}
			\end{eqnn}\\	
			
			\item durch Substitution $z=\frac{y}{x}$\\
			$y'=\frac{x+y}{x}:$
			\begin{eqnn}
				\eqn{y'}{1+\frac{y}{x}}
				\eqn{y'}{1+z}
				\eqntext{$z=\frac{y}{x}\Rightarrow y=zx \Rightarrow y'=z'x+z$:}
				\eqn{z'x+z}{1+z}
				\eqn{z'}{\frac{1}{x}}
				\eqn[][\Rightarrow]{z}{\ln(x)+c}
				\eqntext{Rücksubstitution mit $z=\frac{y}{x}$}
				\eqn{y}{\ln(x)x+cx}
			\end{eqnn}\\\\
			
			$y'=2\frac{y}{x}:$
			\begin{eqnn}
				\eqn{y'}{2z}
				\eqntext{$z=\frac{y}{x}\Rightarrow y=zx \Rightarrow y'=z'x+z$:}
				\eqn{z'x+z}{2z}
				\eqn{z'x-z}{0}
				\eqn[][\Rightarrow]{z}{x\cdot c}
				\eqntext{Rücksubstitution mit $z=\frac{y}{x}$}
				\eqn{y}{x^2\cdot c}
			\end{eqnn}\\
		\end{enumerate}
		
		\item[7.5] Bestimme die allgemeine Lösung der folgenden Bernoulli'schen Differentialgleichungen:
		\begin{enumerate}
			\item $(1+x^2)y'+xy-xy^2=0$
			\begin{eqnn}
				\eqn{y'}{-\frac{x}{1+x^2}y+\frac{x}{1+x^2}y^2}
				\eqnspace
				\eqntext{Substitution mit $z=y^{-1}-1$:}
				\eqn[][\Rightarrow]{z'}{-y^{-2}y'}
				\eqn{z'}{-y^{-2}(\frac{x}{1+x^2}y^2-\frac{x}{^+x^2}y)}
				\eqn{z'}{\frac{x}{1+x^2}(y^{-1}-1)}
				\eqntext{$z$ einsetzen:}
				\eqn{z'}{\frac{x}{1+x^2}z}
				\eqntext{Homogene Differentialgleichung:}
				\eqn[][\Rightarrow]{z}{e^{\int\frac{x}{1+x^2}dx}}
				\eqntext{Substitution mit $u=1+x^2\Rightarrow dx=\frac{1}{2\sqrt{u-1}}du$:}
				\eqn{z}{\sqrt{u}c}
				\eqntext{Rücksubstitution von $z$ und $u$:}
				\eqn{y^{-1}-1}{\sqrt{1+x^2}c}
				\eqn{y}{\frac{1}{\sqrt{1+x^2}c+1}}
			\end{eqnn}			
			
			\item $y'+y+(\sin(x)+e^x)y^3=0$
			\begin{eqnn}
				\eqn{y'}{-y-y^3(\sin(x)+e^x)}
				\eqnspace
				\eqntext{Substitution mit $z=y^{-2}$:}
				\eqn[][\Rightarrow]{z'}{-2y^{-3}y'}
				\eqn{z'}{2y^{-3}(y+y^3(\sin(x)+e^x)}
				\eqn{z'}{2y^{-2}+2(\sin(x)+e^x)}
				\eqntext{$z$ einsetzen:}
				\eqn{z'}{2z+2(\sin(x)+e^x)}
				\eqnspace
				\eqntext{Über Homogene Differentialgleichung berechnen:}
				\eqnf{z'-2z}{0}
				\eqn[(1)][\Rightarrow]{z_H}{e^{\int 2dx}=e^{2x}}
				\eqn{z_H'}{2e^{2x}}
				\eqnspace
				\eqnf{z}{c\cdot z_H}
				\eqn{z'}{c'z_H+cz_H'}
				\eqnspace
				\eqn{c'e^{2x}}{2(\sin(x)+e^x)}
				\eqn{c'}{2(\sin(x)+e^x)e^{-2x}}
				\eqn{c}{2\int\frac{\sin(x)+e^x}{e^{2x}}dx}
				\eqn{c}{2(\int e^{-x}dx + \int\frac{\sin(x)}{e^{2x}})dx}
				\eqn{c}{-2e{-2x}+c+\int\frac{\sin(x)}{e^{2x}})dx}
				\eqnspace
				\eqntext{Aus doppelter partieller Integration von $\int\frac{\sin(x)}{e^{2x}})dx$ mit}
				\\[-40pt]\\\eqntext{$f(x)=e^{-2x}, g_1(x)=\sin(x), g_2(x)=\cos(x)$:}
				\eqn[][\Rightarrow]{\int e^{-2x}\sin(x)}{-e^{-2}\cos(x)-2e^{-2x}\sin(x)-4\int e^{-2x}\sin(x)}
				\eqn{\int e^{-2x}\sin(x)}{-\frac{\cos(x)+2\sin(x)}{5e^{2x}}}
				\eqnspace
				\eqn[][\Rightarrow]{c}{-2e^{-x}+c-\frac{2(\cos(x)+2\sin(x))}{5e^{2x}}}
				\eqntext{In (1) einsetzen:}
				\eqn[][\Rightarrow]{z}{-2e^x+ce^{2x}-\frac{2}{5}(\cos(c)+2\sin(x))}	
				\eqntext{Rücksubstitution von $z$:}
				\eqn[][\Rightarrow]{y}{\frac{1}{\sqrt{-2e^x-\frac{2}{5}(\cos(x)+2\sin(x))+ce^{2x}}}}
			\end{eqnn}
		\end{enumerate}
	\end{enumerate}
\end{document}
