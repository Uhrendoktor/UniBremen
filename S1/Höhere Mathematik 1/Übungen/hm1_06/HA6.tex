\documentclass{HM}
\begin{document}
\begin{enumerate}
	\item[6.3] Zeige mit der Definition des Grenzwertes (ohne Verwendung von "Rechenregeln"), dass $\lim\limits_{n\to\infty}\frac{n^3+3n}{4n^3-5}=\frac{1}{4}$ ist.\\
	Für $n\to\infty$ ist das Ergebnis des Bruchs ausschließlich von $\frac{n^3}{4n^3}$ abhängig. Daraus folgt:\\
	$$\lim\limits_{n\to\infty}\frac{n^3+3n}{4n^3-5}=\lim\limits_{n\to\infty}\frac{n^3}{4n^3}=\lim\limits_{n\to\infty}\frac{1}{4}=\frac{1}{4}$$
	\item[6.4] Untersuche, ob die Folgen konvergent sind, und bestimme gegebenenfalls den Grenzwert.
	\begin{enumerate}
		\item $\left(\frac{4n^2+1}{9n^2-n+3}\right)$\\
		Für $n\to\infty$ ist das Ergebnis des Bruchs ausschließlich von $\frac{4n^2}{9n^2}$ abhängig. Daraus folgt:\\
		$$\lim\limits_{n\to\infty}\frac{4n^2+1}{9n^2-n+3}
		=\lim\limits_{n\to\infty}\frac{4n^2}{9n^2}
		=\lim\limits_{n\to\infty}\frac{4}{9}
		=\frac{4}{9}$$
		Die Folge konvergiert für $n\to\infty$ gegen $\frac{4}{9}$.
		
		\item $\left(\sqrt{n^2+n}-n\right)$\\
		Für $$b,c\in\R;b<c \Rightarrow\sqrt{b}<\sqrt{c},$$
		$$n^2<n^2+n$$
		$$\Rightarrow\sqrt{n^2}<\sqrt{n^2+n}$$
		Mit $a\in\R; a>0$ folgt $\sqrt{n^2+n}=n+an$.\\
		Somit ergibt sich:
		$$\lim\limits_{n\to\infty}\sqrt{n^2+n}-n
		=\lim\limits_{n\to\infty}n+an-n
		=\lim\limits_{n\to\infty}an$$
		$\Rightarrow \sqrt{n^2+n}-n=an\xrightarrow{n\to\infty}\infty$
		
		\item $\left(\prod\limits_{k=2}^n\left(1-\frac{1}{k^2}\right)\right)$\\
		Aus $\frac{1}{k^2}\xrightarrow{k\to\infty}0$ folgt $1-\frac{1}{k^2}\xrightarrow{k\to\infty}1$.\\
		Da $q^n$ für $-1<q<1$ mit $n\to\infty$ gegen 0 läuft und alle Faktoren des Produkts $1-\frac{1}{k^2} \in\left[\frac{2}{3}, 1\right)$ sind, folgt $0\leq \lim\limits_{n\to\infty}\prod\limits_{k=2}^n\left(1-\frac{1}{k^2}\right) \leq \lim\limits_{n\to\infty}a^n$ mit $a\in\R\coloneqq$ "größter reeller Faktor des Produkts".\\
		Aus $\lim\limits_{n\to\infty}a^n=\lim\limits_{n\to\infty}q^n=0$, folgt $\lim\limits_{n\to\infty}\prod\limits_{k=2}^n\left(1-\frac{1}{k^2}\right) = 0$.
		
		\item $\sqrt[n]{n!}$\\
		$\sqrt[n]{n!}=n^\frac{1}{n}\cdot (n-1)^\frac{1}{n}\cdot (n-2)^n\cdot ... \cdot 1^\frac{1}{n}$.\\
		Für alle Grenzwerte der Faktoren $a$ in der Fakultät außer $1^\frac{1}{n}$ gilt für $n\to\infty \Rightarrow a>1$. Für die Ausnahme $b\coloneqq 1^\frac{1}{n}$ gilt $b=1$.\\
		Mit $c\coloneqq\min{a_n}; c>1$ folgt $c\cdot c \cdot ... \cdot c = c^{n}$.\\
		$\Rightarrow c\xrightarrow{n\to\infty}\infty$.\\
		Da $a_n\geq c$ folgt $b*a_1\cdot a_2\cdot ... \cdot a_n\geq c^n \Rightarrow \sqrt[n]{n!}\xrightarrow{n\to\infty}\infty$
		
	\end{enumerate}
	\item[6.5] Seien $(a_n), (b_n)$ Folgen in $\R$. Beweise oder widerlege:
	\begin{enumerate}
		\item $a_n\to a, b_n\neq 0$ für alle $n\in\N, \frac{a_n}{b_n}\to 0 \Rightarrow (b_n)$ keine Nullfolge.\\
		Nein, Widerspruch für $a_n \to 0$ alle $\frac{1_n}{b_n}\to 0$ da $\frac{0}{x}=0$ 
		\item $(b_n)$ Nullfolge, $b_n\neq 0$ für alle $n\in\N\Rightarrow (\frac{1}{b_n})$ nicht konvergent.\\
		\begin{align*}
			b_n \to 0\\
			\lim_{n\to 0}\left(\frac{1}{n}\right)=\infty\\
			\Rightarrow \frac{1}{b_n} \to \infty
		\end{align*}
		\item $a_n\to a, a_n\cdot b_n \to c \Rightarrow (b_n)$ konvergent.\\
		Nein, Wiederspruch mit $a_n \to 0$ und $b_n$ beschränkt\\
		ist $b_n \cdot 0 = 0$\\
		Also konvergiert c zu $0$ 
		\item $a_n\to 0, (b_n)$ beschränkt $\Rightarrow a_n\cdot b_n \to 0$.\\
		Da $b_n$ beschränkt ist $\lim_{n\to \infty} \neq \infty$
		\begin{align*}
			a_n \cdot b_n \to c\\
			\Leftrightarrow 0 \cdot b = c
		\end{align*}
		
	\end{enumerate}
	\item[6.6] Ermittle (ohne Beweis, aber mit kurzer Begründung) das Supremum, Infimum und, falls existent, Maximum und Minimum folgender Mengen:
	\begin{enumerate}
		\item $A=\lbrace 1-\frac{(-1)^n}{n}; n\in\N\rbrace$;
		\begin{align*}
			n\bmod=0:1-\frac{1}{n}\\
			n\bmod\neq0:1+\frac{1}{n}\\
			\lim_{n\to 0} \frac{1}{n} = \infty\\\\
			\Rightarrow \sup A = 2,\inf A = 0
		\end{align*}
		\item $B=\lbrace \frac{1}{n}+\frac{1}{m}; m,n\in\N\rbrace$;
		\begin{align*}
			b_{n,m} = \frac{1}{n}+ \frac{1}{m}&;m,n\in \N\\
\\
			p\in \N:\lim_{p\to \infty}\frac{1}{p} &= 0\\
			p\in \N:\lim_{p\to 0}\frac{1}{p} &= \infty\\
			\Rightarrow \lim_{n,m\to \infty} b_{n,m} &= 0\\
			\lim_{n,m\to 0} b_{n,m} = 
			\lim_{n\to 0,m\to \infty} b_{n,m} &= 
			\lim_{n\to \infty,m\to 0} b_{n,m} = \infty\\\\
			\Rightarrow \sup B = \infty&,\inf B = 0;
		\end{align*}
		\item $C=\lbrace x\in\R; x^2-10x\leq 24\rbrace$;
		\begin{align*}
			\geqnf{x^2-10x}{\leq}{24}
			\geqn{x^2-10x-24}{\leq}{0}
			\text{aus p-q formel}\\
			x_1=-2,x_2=12
		\end{align*}
		Da für $x\in\{-2,12\},x^2-10x=24$ und\\ $x^2-10x>24$ für $x>12\lor x<-2$\\
		sind $-2,12$ das Minimum und Maximum der Menge D
		$$\sup C = 12 = \max C$$
		$$\inf C = -2 = \min C$$\\
		\item $D=\lbrace \frac{x}{x+1}; x\geq 0\rbrace$.\\
		Ist $x\in \R??$\\
		\begin{align*}
			D_n &= \{\frac{1}{2},\frac{2}{3},\frac{3}{4}\dots\frac{n}{n+1}\}\\
			&\Rightarrow \lim_{n\to 0} D_n = 0,\lim_{n\to \infty} D_n = 1\\
			&\Rightarrow \sup D = 1, \inf D = 0
		\end{align*}
	\end{enumerate}
\end{enumerate}
\end{document}