\documentclass{HM}

\begin{document}
\newcommand{\mi}[1]{
\begin{pmatrix}
1 & #1\\
0 & 1
\end{pmatrix}
}
\newcommand{\m}{\mi{1}}
\begin{enumerate}
\item [3.3]
\begin{enumerate}
\item Bestimme die Polardarstellungen von $\frac{i\sqrt{3} - 1}{2}$ und von $\left(\frac{1-i\sqrt{3}}{1+i}\right)^5$.
\begin{enumerate}
	\item \begin{align*}
		&\frac{i\sqrt{3}-1}{2}=-\frac{1}{2}+\frac{i\sqrt{3}}{2}\\
		\\
		|z|&=\sqrt{(\frac{\sqrt{3}}{2})^2 + (-\frac{1}{2})^2}=1\\
		\alpha&=\pi -sin^{-1}\left(\left|\frac{\sqrt{3}}{2}\right|\right)=\pi-cos^{-1}\left(\left|\frac{-1}{2}\right|\right)=\pi-tan^{-1}\left(\left|\frac{\frac{\sqrt{3}}{2}}{-0.5}\right|\right)\\
		\alpha&=\frac{2}{3}\pi \\
		z&=cos\left(\frac{2}{3}\pi\right)isin\left(\frac{2}{3}\pi\right)
	\end{align*}
	\item \begin{align*}
		\eqnf{w}{\sqrt[5]{z}}
		\eqn{w}{\frac{1-i\sqrt{3}}{1+i}}
		\eqn{w}{\frac{(1-i\sqrt{3})(1-i)}{(1+i)*(1-i)}}
		\eqn{w}{\frac{1-i(1+\sqrt{3})-\sqrt{3}}{2}}
		\eqn{w}{\frac{1-\sqrt{3}}{2}-i\frac{1+\sqrt{3}}{2}}
		\\
		|w|&=\sqrt{\left(\frac{1-\sqrt{3}}{2}\right)^2+\left(-\frac{1+\sqrt{3}}{2}\right)^2}=\sqrt{2}
		\\
		\alpha&=\pi+sin^{-1}\left(\left|\frac{-\frac{1+\sqrt{3}}{2}}{\sqrt{2}}\right|\right)=\pi+cos^{-1}\left(\left|\frac{\frac{1-\sqrt{3}}{2}}{\sqrt{2}}\right|\right)\\
		\alpha&=\frac{17}{12}\pi\\
		\text{Polardarstellung von w}\\
		w&=\sqrt{2}\left(cos\left(\frac{17}{12}\pi\right)+isin(\frac{17}{12}\pi)\right)\\
		z&=w^5\\
		\eqnf{z}{4\sqrt{2}\left(cos\left(\frac{85}{12}\pi\right)+isin\left(\frac{85}{12}\pi\right)\right)}
		\text{Winkel um }6\pi\text{ verkleinert}\\
		\eqn{z}{4\sqrt{2}\left(cos\left(\frac{13}{12}\pi\right)+isin\left(\frac{13}{12}\pi\right)\right)}
\end{align*}
\end{enumerate}

\item Bestimme alle Lösungen der Gleichung $z^3 = -8$.
\begin{align*}
	w &= z^3\\	
	\eqnf{|w|}{\sqrt{(-8)^2}}
	\eqn{|w|}{8}\\
\end{align*}
Da der imaginäre Teil von $w=0$ ist und der reale Teil negativ, ist $w$ eine horizontale Linie in Richtung der negativen Reellen Zahlen auf der complexen Ebene und damit ist $\alpha=\pi$ für die Polardarstellung.
\begin{align*}
	w &= 8(cos(\pi)+isin(\pi))\\
	z &= \sqrt[3]{w}\\
	\\
	z_0 &= 2(cos(\frac{\pi}{3})+isin(\frac{\pi}{3}))\\
	z_1 &= 2(cos(\pi)+isin(\pi))\\
	z_2 &= 2(cos(\frac{5\pi}{3})+isin(\frac{5\pi}{3}))
\end{align*}
$$	\mathbb{L}=\left\{2(cos(\frac{\pi}{3})+isin(\frac{\pi}{3})),-2,2(cos(-\frac{\pi}{3})+isin(-\frac{\pi}{3}))\right\}
$$
\end{enumerate}
\item [3.4] Berechne sämtliche möglichen Produkte aus den gegebenen Matrizen
$$
A=\begin{pmatrix}
1 & 2 & 3\\
2 & 0 & 1
\end{pmatrix},
B=\begin{pmatrix}
0 & 1\\
2 & -1
\end{pmatrix},
C=\begin{pmatrix}
-2 & t\\
1 & 0\\
1 & -1
\end{pmatrix},
D=\begin{pmatrix}
1 & 2 & 1\\
-1 & 1 & 0
\end{pmatrix},
$$
wobei $t$ ein reeler Parameter ist.\\

$A\times B\begin{pmatrix}
	2 & 0 & 1 \\
	0 & 4 & 5 \\
\end{pmatrix}$
$A\times C\begin{pmatrix}
	3 & t-3\\
	-3 & 2t-1\\
\end{pmatrix}$\\

$B^2\begin{pmatrix}
	2 & -1 \\
	-2 & 3
\end{pmatrix}$
$B\times D\begin{pmatrix}
	-1 & 1 & 0\\
	3 & 3 & 2\\
\end{pmatrix}$\\

$C\times A\begin{pmatrix}
	-2+2t & -4 & -6+t\\
	1 & 2 & 3\\
	-1 & 2 & 2\\
\end{pmatrix}$
$C\times B\begin{pmatrix}
	2t & -2-t\\
	0 & 1\\
	-2 & 2
\end{pmatrix}$
$C\times D\begin{pmatrix}
	-2-t & -4+t &-2\\
	1 & 2 & 1\\
	2 & 1 & 1
\end{pmatrix}$\\

$D\times C\begin{pmatrix}
	1 & t-1\\
	3 & -t
\end{pmatrix}$
\item [3.5] Berechne
$$\begin{pmatrix}
1 & 1\\
0 & 1
\end{pmatrix}^n$$
für n = 1,2,3,4, stelle eine Vermutung für eine Formel für allgemeines $n\in\N$ und beweise diese Formel durch Induktion.\\\\
$n=1,2,3,4$:\\
\begin{align*}
&\m^1&=\m\\
&\m^2=\m\m &= \mi{2}\\
&\m^3=\m^2\m^1=\mi{2}\m&=\mi{3}\\
&\m^4=\m^3\m^1=\mi{3}\m&=\mi{4}
\end{align*}
IA: $n=1$
$$\m^1=\mi{1}$$\\
IV: 
$$\m^n=\mi{n}$$\\
IS: $n\to n+1$
\begin{align*}
\eqnf{\m^{n+1}}{\mi{n+1}}
\eqn{\m^n\m}{\mi{n+1}}
\eqn{\mi{n}\m}{\mi{n+1}}
\eqn{\mi{n+1}}{\mi{n+1}}
\end{align*}
\item [3.6] Die \textit{Pauli-Matrizen} sind definiert durch
$$\sigma_1 \coloneqq \begin{pmatrix}
0 & 1\\
1 & 0
\end{pmatrix},
\sigma_2 \coloneqq \begin{pmatrix}
0 & -i\\
i & 0
\end{pmatrix},
\sigma_3 \coloneqq \begin{pmatrix}
1 & 0\\
0 & -1
\end{pmatrix}
$$
mit der imaginären Einheit $i$. Zeige für alle $j,k = 1,2,3:$\\
$$\sigma_j\sigma_k = \delta_{jk}E_2+i\sum_{l=1}^{3}\epsilon_{jkl}\sigma_l,$$
wobei $\delta_{jk}$ das Kronecker-Delta ist und
$$\epsilon_{jkl}\begin{cases}
0 &\text{falls mindestens 2 der Inizes }j,k,l \text{denselben Wert haben,}\\
1 &\text{falls} (j,k,l)\in\lbrace(1,2,3),(2,3,1),(3,1,2)\rbrace,\\
-1 &\text{falls} (j,k,l)\in\lbrace(3,2,1),(2,1,3),(1,3,2)\rbrace,
\end{cases}$$
das \textit{Levi-Civita-Symbol}.
$$\sigma_j\sigma_k=\begin{cases}
E_n &j=k\\
i\sigma_l &(j,k)\in\lbrace(1,2),(2,3),(3,1)\rbrace \text{ mit } l\neq j\neq k,\\
-i\sigma_l &(j,k)\in\lbrace(3,2),(2,1),(1,3)\rbrace \text{ mit } l\neq j\neq k
\end{cases}$$
für $j=k$:
$$\sigma_1^2 = E_n \Leftrightarrow \begin{pmatrix}
0&1\\
1&0
\end{pmatrix}^2 = \begin{pmatrix}
1&0\\
0&1
\end{pmatrix} \Leftrightarrow \begin{pmatrix}
1&0\\
0&1
\end{pmatrix}=\begin{pmatrix}
1&0\\
0&1
\end{pmatrix} \surd$$\\
$$\sigma_2^2 = E_n \Leftrightarrow \begin{pmatrix}
0&-i\\
i&0
\end{pmatrix}^2 = \begin{pmatrix}
1&0\\
0&1
\end{pmatrix} \Leftrightarrow \begin{pmatrix}
1&0\\
0&1
\end{pmatrix}=\begin{pmatrix}
1&0\\
0&1
\end{pmatrix} \surd$$\\
$$\sigma_3^2 = E_n \Leftrightarrow \begin{pmatrix}
1&0\\
0&-1
\end{pmatrix}^2 = \begin{pmatrix}
1&0\\
0&1
\end{pmatrix} \Leftrightarrow \begin{pmatrix}
1&0\\
0&1
\end{pmatrix}=\begin{pmatrix}
1&0\\
0&1
\end{pmatrix} \surd$$\\
für $(j,k)\in\lbrace(1,2),(2,3),(3,1)\rbrace \text{ mit } l\neq j\neq k$:
\begin{align*}
\sigma_1\sigma_2&=i\sigma_3 &\Leftrightarrow& \begin{pmatrix}
0&1\\
1&0
\end{pmatrix}&\begin{pmatrix}
0&-i\\
i&0
\end{pmatrix}&=&i&\begin{pmatrix}
1&0\\
0&-1
\end{pmatrix}&\Leftrightarrow& \begin{pmatrix}
i&0\\
0&-i
\end{pmatrix}&=&\begin{pmatrix}
i&0\\
0&-i
\end{pmatrix} &\surd\\
\sigma_2\sigma_3&=i\sigma_1 &\Leftrightarrow& \begin{pmatrix}
0&-i\\
i&0
\end{pmatrix}&\begin{pmatrix}
1&0\\
0&-1
\end{pmatrix}&=&i&\begin{pmatrix}
0&1\\
1&0
\end{pmatrix}&\Leftrightarrow& \begin{pmatrix}
0&i\\
i&0
\end{pmatrix}&=&\begin{pmatrix}
0&i\\
i&0
\end{pmatrix} &\surd\\
\sigma_3\sigma_1&=i\sigma_2 &\Leftrightarrow& \begin{pmatrix}
1&0\\
0&-1
\end{pmatrix}&\begin{pmatrix}
0&1\\
1&0
\end{pmatrix}&=&i&\begin{pmatrix}
0&-i\\
i&0
\end{pmatrix}&\Leftrightarrow& \begin{pmatrix}
0&1\\
-1&0
\end{pmatrix}&=&\begin{pmatrix}
0&1\\
-1&0
\end{pmatrix} &\surd
\end{align*}
für $(j,k)\in\lbrace(3,2),(2,1),(1,3)\rbrace \text{ mit } l\neq j\neq k$:
\begin{align*}
\sigma_3\sigma_2&=-i\sigma_1 &\Leftrightarrow& \begin{pmatrix}
1&0\\
0&-1
\end{pmatrix}&\begin{pmatrix}
0&-i\\
i&0
\end{pmatrix}&=&-i&\begin{pmatrix}
0&1\\
1&0
\end{pmatrix}&\Leftrightarrow& \begin{pmatrix}
0&-i\\
-i&0
\end{pmatrix}&=&\begin{pmatrix}
0&-i\\
-i&0
\end{pmatrix} &\surd\\
\sigma_2\sigma_1&=-i\sigma_3 &\Leftrightarrow& \begin{pmatrix}
0&-i\\
i&0
\end{pmatrix}&\begin{pmatrix}
0&1\\
1&0
\end{pmatrix}&=&-i&\begin{pmatrix}
1&0\\
0&-1
\end{pmatrix}&\Leftrightarrow& \begin{pmatrix}
-i&0\\
0&i
\end{pmatrix}&=&\begin{pmatrix}
-i&0\\
0&i
\end{pmatrix} &\surd\\
\sigma_1\sigma_3&=-i\sigma_2 &\Leftrightarrow& \begin{pmatrix}
0&1\\
1&0
\end{pmatrix}&\begin{pmatrix}
1&0\\
0&-1
\end{pmatrix}&=&-i&\begin{pmatrix}
0&-i\\
i&0
\end{pmatrix}&\Leftrightarrow& \begin{pmatrix}
0&-1\\
1&0
\end{pmatrix}&=&\begin{pmatrix}
0&-1\\
1&0
\end{pmatrix} &\surd
\end{align*}
\end{enumerate}
\end{document}
