\documentclass{HM}

\begin{document}
\begin{enumerate}
\item [3.3]
\begin{enumerate}
\item Bestimme die Polardarstellungen von $\frac{i\sqrt{3} - 1}{2}$ und von $\left(\frac{1-i\sqrt{3}}{1+i}\right)^5$.
\begin{enumerate}
	\item \begin{align*}
		&\frac{i\sqrt{3}-1}{2}=-\frac{1}{2}+\frac{i\sqrt{3}}{2}\\
		\\
		\eqnf{|z|}{\sqrt{(\frac{\sqrt{3}}{2})^2 + (-\frac{1}{2})^2}}
		\eqn{|z|}{1}\\
		\alpha&=\pi -sin^{-1}\left(\left|\frac{\sqrt{3}}{2}\right|\right)=\pi-cos^{-1}\left(\left|\frac{-1}{2}\right|\right)=\pi-tan^{-1}\left(\frac{|\frac{\sqrt{3}}{2}|}{|-0.5|}\right)\\
		\alpha&=\frac{2}{3}\pi \\
		z&=cos\left(\frac{2}{3}\pi\right)isin\left(\frac{2}{3}\pi\right)
	\end{align*}
	\item \begin{align*}
		w&=\sqrt[5]{z}\\
		w&=\frac{1-i\sqrt{3}}{1+i}\\
		w&=\frac{(1-i\sqrt{3})(1-i)}{(1+i)*(1-i)}\\
		w&=\frac{1-i(1+\sqrt{3})-\sqrt{3}}{2}\\
		w&=\frac{1-\sqrt{3}}{2}-i\frac{1+\sqrt{3}}{2}\\
		\\
		|w|&=\sqrt{\left(\frac{1-\sqrt{3}}{2}\right)^2+\left(-\frac{1+\sqrt{3}}{2}\right)^2}\\
		|w|&=\sqrt{2}\\
		\\
		\alpha&=\pi+sin^{-1}\left(\frac{\left|-\frac{1+\sqrt{3}}{2}\right|}{\sqrt{2}}\right)=\pi+cos^{-1}\left(\frac{\left|\frac{1-\sqrt{3}}{2}\right|}{\sqrt{2}}\right)\\
		\alpha&=\frac{17}{12}\pi\\
		\text{Polardarstellung von w}\\
		w&=\sqrt{2}\left(cos(\frac{17}{12}\pi)+isin(\frac{17}{12}\pi)\right))\\
		z&=w^5\\
		z&=4\sqrt{2}\left(cos(\frac{85}{12}\pi)+isin\left(\frac{85}{12}\pi\right)\right)\\
		z&=4\sqrt{2}\left(cos(\frac{13}{12}\pi)+isin\left(\frac{13}{12}\pi\right)\right)\\
\end{align*}
\end{enumerate}

\item Bestimme alle Lösungen der Gleichung $z^3 = -8$.
\end{enumerate}
\item [3.4] Berechne sämtliche möglichen Produkte aus den gegebenen Matrizen
$$
A=\begin{pmatrix}
1 & 2 & 3\\
2 & 0 & 1
\end{pmatrix},
B=\begin{pmatrix}
0 & 1\\
2 & -1
\end{pmatrix},
C=\begin{pmatrix}
-2 & t\\
1 & 0\\
1 & -1
\end{pmatrix},
D=\begin{pmatrix}
1 & 2 & 1\\
-1 & 1 & 0
\end{pmatrix},
$$
wobei $t$ ein reeler Parameter ist.\\

$A\times B\begin{pmatrix}
	2 & 0 & 1 \\
	0 & 4 & 5 \\
\end{pmatrix}$
$A\times C\begin{pmatrix}
	3 & t-3\\
	-3 & 2t-1\\
\end{pmatrix}$\\

$B^2\begin{pmatrix}
	2 & -1 \\
	-2 & 3
\end{pmatrix}$
$B\times D\begin{pmatrix}
	-1 & 1 & 0\\
	3 & 3 & 2\\
\end{pmatrix}$\\

$C\times A\begin{pmatrix}
	-2+2t & -4 & -6+t\\
	1 & 2 & 3\\
	-1 & 2 & 2\\
\end{pmatrix}$
$C\times B\begin{pmatrix}
	2t & -2-t\\
	0 & 1\\
	-2 & 2
\end{pmatrix}$
$C\times D\begin{pmatrix}
	-2-t & -4+t &-2\\
	1 & 2 & 1\\
	2 & 1 & 1
\end{pmatrix}$\\

$D\times C\begin{pmatrix}
	1 & t-1\\
	3 & -t
\end{pmatrix}$
\item [2.5] Berechne
$$\begin{pmatrix}
1 & 1\\
0 & 1
\end{pmatrix}^n$$
für n = 1,2,3,4, stelle eine Vermutung für eine Formel für allgemeines $n\in\N$ und beweise diese Formel durch Induktion.
\item [3.6] Die \textit{Pauli-Matrizen} sind definiert durch
$$\sigma_1 \coloneqq \begin{pmatrix}
0 & 1\\
1 & 0
\end{pmatrix},
\sigma_2 \coloneqq \begin{pmatrix}
0 & -i\\
i & 0
\end{pmatrix},
\sigma_1 \coloneqq \begin{pmatrix}
1 & 0\\
0 & -1
\end{pmatrix}
$$
mit der imaginären Einheit $i$. zeige für alle $j,k = 1,2,3:$\\
$$\sigma_j\sigma_k = \delta_{jk}E_2+i\sum_{l=1}^{3}\epsilon_{jkl}\sigma_l,$$
wobei $\delta_{jk}$ das Kronecker-Delta ist und
$$\epsilon_{jkl}\begin{cases}
0 &\text{falls mindestens 2 der Inizes }j,k,l \text{denselben Wert haben,}\\
1 &\text{falls} (j,k,l)\in\lbrace(1,2,3),(2,3,1),(3,1,2)\rbrace,\\
-1 &\text{falls} (j,k,l)\in\lbrace(3,2,1),(2,1,3),(1,3,2)\rbrace,
\end{cases}$$
das \textit{Levi-Civita-Symbol}.
\end{enumerate}
\end{document}
