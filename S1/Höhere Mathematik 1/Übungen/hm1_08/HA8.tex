\documentclass{HM}
\begin{document}
\begin{enumerate}
\item[8.3] Untersuche die Reihe $\sum\limits_{n=1}^\infty a_n$ auf Konvergenz und absolute Konvergenz, wobei\\
\begin{enumerate}
\item $a_n=\left(\frac{1+i}{2}\right)^n$;
\begin{align*}
 &\sum\limits_{n=1}^\infty\left(\frac{1+i}{2}\right)^n\\
=&\sum\limits_{n=0}^\infty\left(\frac{1+i}{2}\right)^n-1
\end{align*}
Geometrische Reihe mit $|\frac{1+i}{2}| = \frac{|1+i|}{|2|} = \frac{\sqrt{2}}{2} < 1$:\\
$\Rightarrow 
 \sum\limits_{n=0}^\infty\left(\frac{1+i}{2}\right)^n-1
=\frac{1}{1-(\frac{1}{2}+\frac{i}{2})}
=\frac{1}{\frac{1}{2}-\frac{i}{2}}=2-\frac{2}{i}=2+2i$
\item $a_n=\frac{(-1)^n}{\sqrt[3]{n}}$;
\item $a_n=(-1)^n\frac{n+2}{2n}$.
\end{enumerate}
\item[8.4] Sei $s_n\coloneqq \sum\limits_{k=0}^n\frac{1}{k!}$ und $e=\lim\limits_{n\to\infty}s_n$ die Euler'sche Zahl.
\begin{enumerate}
\item Zeige die Ungleichungen $0<e-s_n<\frac{1}{n*n!}$ für $n\in\N$, mit Hilfe einer geeigneten geometrischen Reihe.
\item Bestimme mit Hilfe von (a) eine Zahl $n\in\N$, für die $|e-s_N|\leq 0.5\cdot 10^{-4}$ gilt, und gib den Wert von $s_N$ an.
\item Zeige, dass die Euler'sche Zahl $e$ irrational ist.
\end{enumerate}
\item[8.5]
\begin{enumerate}
\item Konvergiert die Reihe $\sum\limits_{n=2}^\infty\left(\frac{1}{\sqrt{n}-1}+\frac{1}{\sqrt{n}+1}\right)$?
\item Berechne $\sum\limits_{n=1}^\infty\frac{n-1}{n!}$.
\item Für welche $z\in\C\setminus\lbrace-1\rbrace$ konvergiert die Reihe $\sum\limits_{n=1}^\infty\frac{z^{n-1}}{(1+z)^n}?$\\
Bestimme den Grenzwert, falls er existiert.
\end{enumerate}
\item[8.6] Ermittle (durch Probieren) das kleinste $n\in\N$, für dass $\sum\limits_{k=1}^n\frac{1}{k}>3$ ist. benutze einen Computer, um herauszufinden, wie groß man $n$ wählen muss, damit die Summe $>6$ bzw. $>9$ wird.
\end{enumerate}
\end{document}