\documentclass{../HM}
\newcommand\course{HM 2}
\newcommand\hwnumber{11}
\usepackage{gauss}
\usepackage{tikz}
\usepackage{pgfplots}

\newcommand{\mue}{m_{\textit{ü}}}
\begin{document}
	\begin{enumerate}
		\item[11.3]
		\begin{enumerate}
			\item Sei $D \coloneqq \{(x,y,z)\in \R^3;x\neq 0\}$. Bestimme den Gradienten $\varDelta f$ der Funktion $f: D \to \R, f(x,y,z) \coloneqq y\sin(xz^2)+\frac{x \cos(y)}{y}$.\\
			$$\varDelta f = \m{
				y\cos(xz^2)z^2+\frac{\cos(y)}{y}\\
				\sin(xz^2)-x(\frac{\sin(y)}{y}+\frac{\cos(y)}{y^2})\\
				y\cos(2xz)
			}$$
			\item Entscheide, ob es partiell differenzierbare Funktionen auf $R^2$ bzw. $R^3$ gibt mit
			$$\varDelta f(x,y)=\m{x+y\\-x+y}\text{ bzw. }\varDelta f(x,y,z)=\m{yz\\xz\\xy}$$
			Bestimme gegebenenfalls ein solches $f$.\\
			\begin{eqnn}
				\eqnf{\varDelta f(x,y)}{\m{x+y\\-x+y}}
				\eqnr{f(x,y)}{\int x+y dx}
				\eqn{f(x,y)}{\int -x+y dy}
				\eqnr{\int x+y dx}{\int -x+y dy}
				\eqn{\frac{x^2}{2}+yx}{\frac{y^2}{2}-yx}
			\end{eqnn}
			$\Rightarrow$ Es existiert keine partiell differenzierbare Funktion $f\in\R^2$\\ mit $\varDelta f(x,y)=\m{x+y\\-x+y}.$\\\\

			\begin{eqnn}
				\eqnf{\varDelta f(x,y,z)}{\m{yz\\xz\\xy}}
				\eqnr{f(x,y,z)}{\int yz dx}
				\eqnf{}{\int xz dy}
				\eqnf{}{\int xy dz}
				\eqnr{f(x,y,z)}{xyz}
			\end{eqnn}

		\end{enumerate}
		\item[11.4] Sei $c>0$, und $f,g: \R \to \R$ seien zweimal stetig differenzierbar. Zeige: Die Funktion $u: \R^2 \to \R$, $u(t,x)\coloneqq f(x+ct)+g(x-ct)$ ist eine Lösung der \textit{Wellengleichung}
		$$\frac{\partial^2}{\partial t^2}u(t,x)=c^2\cdot \frac{\partial^2}{\partial x^2}u(t,x)$$
		\begin{eqnn}
			\eqnf{\frac{\partial}{\partial t}u(t,x)}{c(f'(x+ct)-g'(x-ct))}
			\eqnr{\frac{\partial^2}{\partial t^2}u(t,x)}{c^2(f''(x+ct)+g''(x-ct))}
			\eqnspace
			\eqnf{\frac{\partial}{\partial x}u(t,x)}{f'(x+ct)+g'(x-ct)}
			\eqnr{\frac{\partial^2}{\partial x^2}u(t,x)}{f''(x+ct)+g''(x-ct)}
			\eqnspace
			\eqnr{\frac{\partial^2}{\partial t^2}u(t,x)}{c^2 \frac{\partial^2}{\partial x^2}u(t,x)}
		\end{eqnn}
		\item[11.5] Bestimme Lage und Art der lokalen Extrema folgender Funktionen:
		\begin{enumerate}
			\item $f:\R^2 \to \R, f(x,y)\coloneqq xy^2-4xy+x^2$
			\begin{eqnn}
				\eqnr{\varDelta f(x,y)}{\m{y^2-4y+2x\\2xy-4x}}
				\eqnr{\Hess f(x,y)}{\m{2&2y-4\\2y-4&2x}}
				\eqnspace
				\eqntext{Nullstellen finden:}
				\geqnf{\varDelta f(x,y)}{\feq}{0}
				\geqn{\m{y^2-4y+2x\\2xy-4x}}{\feq}{\m{0\\0}}
				\eqnspace
				\eqnf{y^2-4y+2x}{0}
				\eqnr[(I)]{x}{\frac{4y-y^2}{2}}
				\eqnspace
				\eqnf{2xy-4x}{0}
				\eqnr[(II)]{x}{0}
				\eqnspace
				\eqntext{I in die 2.Gleichung einsetzen:}
				\eqnf{2y+4y-y^2-8y+2y^2}{0}
				\eqn{y^2-2y}{0}
				\eqnr{y_1}{0,}
				\eqnf{y_2}{2}
				\eqnspace
				\eqntext{II in die 1.Gleichung einsetzen:}
				\eqnf{y^2-4y}{0}
				\eqnr{y_3}{y_1=0,}
				\eqnf{y_4}{4}
				\eqnspace
				\eqntext{$y_1=y_3, y_2, y_4$ in $\varDelta f(x,y)\feq 0$ einsetzen und nach $x$ lösen:}
			\end{eqnn}
			$\Rightarrow \L=\{(0,0),(2,2),(0,4)\}$.\\\\
			Alle Nullstellen in $\Hess f(x,y)$ einsetzen um die Art des Extremums zu bestimmen.
			\begin{eqnn}
				\eqnf{\Hess f(0,0)}{\m{2&-4\\-4&0}}
				\eqntext{negative Determinante $\Rightarrow$ indefinit $\Rightarrow$ Sattelpunkt.}
				\eqnspace
				\eqnf{\Hess f(2,2)}{\m{2&0\\0&4}}
				\eqntext{positive Determinante mit $a_{11}>0$ $\Rightarrow$ definit $\Rightarrow$ Tiefpunkt.}
				\eqnspace
				\eqnf{\Hess f(0,4)}{\m{2&4\\4&0}}
				\eqntext{negative Determinante $\Rightarrow$ indefinit $\Rightarrow$ Sattelpunkt.}
			\end{eqnn}

			\item $g:(0,\pi)\times(0,\pi)\to \R, g(x,y) \coloneqq \sin(x) + \sin(y) + \sin(x+y)$
			
		\end{enumerate}
		\item[11.6]
		\begin{enumerate}
			\item Berechne die Jacobi-Matrix der Abbildungen (Polar- bzw. Kugelkoordinaten)
			\begin{enumerate}
				\item $F:\R^2 \to \R^2,F(r,\varphi)\coloneqq(r \cos(\varphi), r\sin(\varphi))^T$
				\item $G: \R^3 \to \R^3, G(r, \theta, \varphi)\coloneqq(r \cos(\theta) \cos(\varphi), r\cos(\theta)\sin(\varphi),r\sin(\theta))^T$\\
				Gib die partiellen Ableitungen von $F_2$ und $G_3$ an, wobei $F = (F_1 , F_2 )^T$ und $G = (G_1 , G_2 , G_3 )^T$.
			\end{enumerate}
		\item Bestimme, in welchen Punkten die Jacobi-Matrizen aus (a) invertierbar sind.
		\end{enumerate}

	\end{enumerate}
\end{document}
