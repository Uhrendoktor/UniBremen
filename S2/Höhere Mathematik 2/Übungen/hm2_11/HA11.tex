\documentclass{../HM}
\newcommand\course{HM 2}
\newcommand\hwnumber{11}
\usepackage{gauss}
\usepackage{tikz}
\usepackage{pgfplots}

\newcommand{\mue}{m_{\textit{ü}}}
\begin{document}
	\begin{enumerate}
		\item[11.3] a
		\item[11.4] Sei $c>0$, und $f,g: \R \to \R$ seien zweimal stetig differenzierbar. Zeige: Die Funktion $u: \R^2 \to \R$, $u(t,x)\coloneqq f(x+ct)+g(x-ct)$ ist eine Lösung der \textit{Wellengleichung}
		$$\frac{\partial^2}{\partial t^2}u(t,x)=c^2\cdot \frac{\partial^2}{\partial x^2}u(t,x)$$
		\item[11.5] Bestimme Lage und Art der lokalen Extrema folgender Funktionen:
		\begin{enumerate}
			\item $f:\R^2 \to \R, f(x,y)\coloneq xy^2-4xy+x^2$
			\item $g:(0,\pi)\times(0,\pi)\to \R, g(x,y) \coloneq \sin(x) + \sin(y) + \sin(x+y)$
		\end{enumerate}
		\item[11.6]
		\begin{enumerate}
			\item Berechne die Jacobi-Matrix der Abbildungen (Polar- bzw. Kugelkoordinaten)
			\begin{enumerate}
				\item $F:\R^2 \to \R^2,F(r,\varphi)\coloneq(r \cos(\varphi), r\sin(\varphi))^T$
				\item $G: \R^3 \to \R^3, G(r, \theta, \varphi)\coloneq(r \cos(\theta) \cos(\varphi), r\cos(\theta)\sin(\varphi),r\sin(\theta))^T$
			\end{enumerate}
		\end{enumerate}

	\end{enumerate}
\end{document}
