\documentclass{HM}

\begin{document}
\begin{enumerate}
\item [10.3] Ermittle sämtliche Zahlen $z \in \C$, die den folgenden Gleichungen genügen:\\
\begin{enumerate}
	\item $z^4 = -16$\\
	\begin{align*}
		|z^4| &= \sqrt{(-16)^2} = 16\\
		\varphi &= \pi\\
		z^4 &= 16 e^{i\pi}\\
		z_k &= 2 e^{i(\frac{\pi + 2\pi k}{4})}\\\\
		\Rightarrow
		z_0 &= 2 e^{i(\frac{\pi}{4})}\\
		z_1 &= 2 e^{i(\frac{3\pi}{4})}\\
		z_2 &= 2 e^{i(\frac{5\pi}{4})}\\
		z_3 &= 2 e^{i(\frac{8\pi}{4})}\\
		\mathbb{L}&= \{2 e^{i(\frac{\pi}{4})},2 e^{i(\frac{3\pi}{4})},2 e^{-i(\frac{\pi}{4})},2 e^{-i(\frac{3\pi}{4})}\}
	\end{align*}

	\item $z^3 + 3iz^2 - 3z - 9i = 0$ (Hinweis: kubische Ergänzung)\\
	\begin{align*}
		\eqnf{z^3 + 3iz^2 - 3z - 9i}{0}
		\eqn{(z+3i)(z^2-3)}{0}
		\Rightarrow z=-3i,z^2 = 3\\
		\mathbb{bb}=\{-3i,\sqrt{3},-\sqrt{3}\}
	\end{align*}
	\item 
\end{enumerate}

\item [10.4.] Sei $a > 0$, $ a \neq 1$. Berechne ohne Taschenrechner / Computer:\\

\begin{align*}
	\log_2 (0.125)\\
	0.125=\frac{1}{8}=8^{-1}=(2^3)^{-1}=2^{-3}\\
	\Rightarrow \log_2 (0.125) = -3
\end{align*}\\
\begin{align*}
	\log_{a^2}(\sqrt{a})\\
	\sqrt{a}=\sqrt{\sqrt{a^2}}=\sqrt[4]{a^2}=(a^2)^\frac{1}{4}\\
	\Rightarrow \log_{a^2}(\sqrt{a}) = \frac{1}{4}
\end{align*}\\
\begin{align*}
	\log_2\sqrt[3]{32}\\
	\sqrt[3]{32}=\sqrt[3]{2^5}=(2^5)^\frac{1}{3}=2^\frac{5}{3}\\
	\log_2(\sqrt[3]{32}) = \frac{5}{3}	
\end{align*}\\
\begin{align*}
	3^{\log_9 (4)}\\
	3^{\log_9 (4)} = \sqrt{(3^{\log_9 (4)})^2} = \sqrt{3^{2\log_9 (4)}} = \sqrt{9^{\log_9 (4)}} = \sqrt{4} = 2
\end{align*}

\item [10.5.]
\item [10.6.] Die Hyperbelfunktionen oder hyperbolischen Funktionen Cosinus hyperbolicus und
Sinus hyperbolicus sind definiert durch$$cosh(x):=\frac{1}{2}(e^x+e^{-x}) \text{ und } sinh(x):=\frac{1}{2}(e^x-e^{-x})(x\in\R)$$\\
\begin{enumerate}
	\item Bestätige die Formel $$cosh^2(x) - sinh^2(x) = 1\text{      } x\in\R$$\\
	\begin{align*}
		\eqnf{\frac{1}{4}(e^x+e^{-x})^2-\frac{1}{4}(e^x-e^{-x})^2}{1}
		\eqn{e^{2x}+e^{-2x}+2-(e^{2x}+e^{-2x}-2)}{4}
		\eqn{4}{4}
	\end{align*}
	\item Zeige, dass die Funktion $sinh: \R \to \R$ streng monoton wachsend und bijektiv ist, und
	dass die Umkehrfunktion $arsinh: \R \to \R$ (Areafunktion: Area Sinus hyperbolicus) stetig
	ist und durch
	$arsinh(x) = \ln (x + \sqrt{x^2+1})$
	gegeben ist.\\
	
	\begin{align*}
		\sqrt{x^2+1} > x\\
		\ln(x2) \text{ ist streng monoton}\\
		\Rightarrow \ln(\sqrt{x^2+1})\text{ ist streng monoton}\\
		%\ln(x)\text{ stetig }\sqrt{x}\text{ stetig }x^2\text{ stetig}\\
		%\Rightarrow \ln(\sqrt{x^2+1})\text{ stetig und streng monoton}\\
		arcsinh(x):[1,\infty)\\
		\text{Dadurch ist laut Stetigkeit der Umkehrfunktion auch die Umkehrfunktion von arcshin(x) stetig und streng monoton}\\
		sinh(x): [0,\infty]\\
		\text{Hier ist wieder }
	\end{align*}
	
\end{enumerate}


\end{enumerate}
\end{document}