\documentclass{HM}

\usepackage{listings}
\usepackage{color}
\definecolor{lightgray}{rgb}{.9,.9,.9}
\definecolor{darkgray}{rgb}{.4,.4,.4}
\definecolor{purple}{rgb}{0.65, 0.12, 0.82}

\lstdefinelanguage{JavaScript}{
  keywords={typeof, new, true, false, catch, function, return, null, catch, switch, var, if, in, while, do, else, case, break},
  keywordstyle=\color{blue}\bfseries,
  ndkeywords={class, export, boolean, throw, implements, import, this},
  ndkeywordstyle=\color{darkgray}\bfseries,
  identifierstyle=\color{black},
  sensitive=false,
  comment=[l]{//},
  morecomment=[s]{/*}{*/},
  commentstyle=\color{purple}\ttfamily,
  stringstyle=\color{red}\ttfamily,
  morestring=[b]',
  morestring=[b]"
}

\lstset{
   language=JavaScript,
   backgroundcolor=\color{lightgray},
   extendedchars=true,
   basicstyle=\footnotesize\ttfamily,
   showstringspaces=false,
   showspaces=false,
   numbers=left,
   numberstyle=\footnotesize,
   numbersep=9pt,
   tabsize=2,
   breaklines=true,
   showtabs=false,
   captionpos=b
}

\begin{document}
\begin{enumerate}
\item[11.3]
\begin{enumerate}
	\item Gib Polardarstellungen der komplexen Zahlen $1+i$ und $\sqrt{3}-i$ an.
	$$1+i = \sqrt{2}\cdot \left(\cos\left(\frac{\pi}{4}\right)+i\sin\left(\frac{\pi}{4}\right)\right) \Leftarrow \arctan\left(\frac{1}{1}\right)=\frac{\pi}{4}$$
	$$\sqrt{3}-i=2\left(\cos\left(-\frac{\pi}{6}\right)+i\sin\left(-\frac{\pi}{6}\right)\right) \Leftarrow -\arctan\left(\frac{1}{\sqrt{3}}\right)=-\frac{\pi}{6}$$
	\item Sei $\varphi\in\left(-\frac{\pi}{2},\frac{\pi}{2}\right)$ und $x=\tan\varphi$. Zeige $\frac{1+ix}{1-ix}=\exp(2i\varphi)$.
	$$x=\tan\varphi=\frac{\sin\varphi}{\cos\varphi}$$
	\begin{align*}
		\Rightarrow &\frac{1+ix}{1-ix}\\
		=&\frac{1+i\frac{\sin\varphi}{\cos\varphi}}{1-i\frac{\sin\varphi}{\cos\varphi}}\\
		=&\frac{\cos\varphi+i\sin\varphi}{\cos\varphi-i\sin\varphi}\\
		=&\frac{(\cos\varphi+i\sin\varphi)(\cos\varphi+i\sin\varphi)}{\cos^2\varphi+\sin^2\varphi}\\
		=&\frac{e^{i\varphi}\cdot e^{i\varphi}}{1}=e^{2i\varphi}\\
		=&\exp(2i\varphi)	
	\end{align*}
\end{enumerate}
\item [11.4] Berechne die Ableitungen der folgenden Funktionen (auf dem jeweils sinnvollen Definitionsbereich):\\\\
$$f_1(x)=\sin(\cos x)$$
Produktregel:
$$\frac{d}{dx}\sin(\cos x) = -\cos(\cos x)\cdot\sin x$$\\\\
$$f_2(x)=\frac{1-\sin x}{2+\sin x}$$
Quotientenregel:
$$\frac{d}{dx}\frac{1-\sin x}{2+\sin x}=\frac{-\cos(x)\cdot(2+\sin(x))-\cos(x)\cdot(1-\sin(x))}{(2+\sin(x))^2}=\frac{-3\cos(x)}{(2+\sin(x))^2}$$\\\\
$$f_3(x)=x\cdot|x|$$
$$\frac{d}{dx}|x|=\begin{cases}
1&x\geq0\\
-1&x<0
\end{cases}$$
Produktregel:
$$\frac{d}{dx}x\cdot|x|=|x|+x\cdot\frac{d}{dx}|x|$$\\\\
$$f_4(x)=x^{x^x}$$
Produktregel, Kettenregel:
\begin{align*}
	&\frac{d}{dx}x^{x^x}\\
	=&\frac{d}{dx}e^{x^x\cdot\ln(x)}\\
	=&e^{x^x\cdot\ln(x)}\cdot\frac{d}{dx}\left(x^x\cdot\ln(x)\right)\\
	=&e^{x^x\cdot\ln(x)}\cdot\left(x^x\cdot\frac{1}{x}+\left(\frac{d}{dx}e^{x\ln(x)}\right)\cdot\ln(x)\right)\\
	=&e^{x^x\cdot\ln(x)}\cdot\left(x^x\cdot\frac{1}{x}+e^{x\ln(x)}\cdot\left(\frac{d}{dx}x\ln(x)\right)\cdot\ln(x)\right)\\
	=&e^{x^x\cdot\ln(x)}\cdot\left(x^x\cdot\frac{1}{x}+e^{x\ln(x)}\cdot\left(\ln(x)+x\cdot\frac{1}{x}\right)\cdot\ln(x)\right)\\
	=&e^{x^x\cdot\ln(x)}\cdot\left(x^x\cdot\frac{1}{x}+e^{x\ln(x)}\cdot\ln(x)\cdot\ln(x)\right)\\
	=&e^{x^x\cdot\ln(x)}\cdot\left(x^{x-1}+e^{x\ln(x)}\cdot\ln^2(x)\right)\\
	=&e^{\ln(x)\cdot\left(x^x+x\right)}\cdot\ln^2(x)+e^{x^x\cdot\ln(x)}\cdot x^{x-1}\\
	=&xe^{x^x+x}\cdot\ln^2(x)+xe^{x^x}\cdot x^{x-1}\\
	=&xe^{x^x+x}\cdot\ln^2(x)+e^{x^x}\cdot x^x
\end{align*}
\newpage
\item [11.5]
\begin{enumerate}
	\item Die Funktion $f:\R\to\R$ sein in $a\in\R$ differenzierbar. Zeige, dass dann der Grenzwert $\lim\limits_{h\to 0}\frac{f(a+h)-f(a-h)}{2h}$ existiert, und berechne ihn.\\\\
	Da sich $\frac{f(a+h)-f(a-h)}{2h}$ für $h\to 0$ einem festen Wert annähert, muss $\frac{d}{dh}\frac{f(a+h)-f(a-h)}{2h}=0$ sein.
	\begin{align*}
		\Rightarrow\eqnfi{\frac{d}{dh}\frac{f(a+h)-f(a-h)}{2h}}{0}{\text{Quotientenreg.}}
		\eqn{\frac{2h\left(\frac{d}{dh}f(a+h)-\frac{d}{dh}f(a-h)\right)-2(f(a+h)-f(a-h))}{4h^2}}{0}
		\eqn{\frac{\frac{d}{dh}f(a+h)-\frac{d}{dh}f(a-h)}{2h}-\frac{f(a+h)-f(a-h)}{2h^2}}{0}
	\end{align*}
	\begin{align*}
		\eqni{\frac{\frac{d}{dh}f(a+h)-\frac{d}{dh}f(a-h)}{2h}}{\frac{f(a+h)-f(a-h)}{2h^2}}{\cdot 2h}
		\eqn{\frac{d}{dh}f(a+h)-\frac{d}{dh}f(a-h)}{\frac{f(a+h)-f(a-h)}{2h}}
	\end{align*}
	$\Rightarrow \lim\limits_{h\to 0}\frac{f(a+h)-f(a-h)}{2h}=\lim\limits_{h\to 0}\frac{d}{dh}f(a+h)-\frac{d}{dh}f(a-h)=f'(a)$
\end{enumerate}
\end{enumerate}
\end{document}