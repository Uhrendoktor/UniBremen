\documentclass{HM}

\usepackage{listings}
\usepackage{color}
\definecolor{lightgray}{rgb}{.9,.9,.9}
\definecolor{darkgray}{rgb}{.4,.4,.4}
\definecolor{purple}{rgb}{0.65, 0.12, 0.82}

\lstdefinelanguage{JavaScript}{
  keywords={typeof, new, true, false, catch, function, return, null, catch, switch, var, if, in, while, do, else, case, break},
  keywordstyle=\color{blue}\bfseries,
  ndkeywords={class, export, boolean, throw, implements, import, this},
  ndkeywordstyle=\color{darkgray}\bfseries,
  identifierstyle=\color{black},
  sensitive=false,
  comment=[l]{//},
  morecomment=[s]{/*}{*/},
  commentstyle=\color{purple}\ttfamily,
  stringstyle=\color{red}\ttfamily,
  morestring=[b]',
  morestring=[b]"
}

\lstset{
   language=JavaScript,
   backgroundcolor=\color{lightgray},
   extendedchars=true,
   basicstyle=\footnotesize\ttfamily,
   showstringspaces=false,
   showspaces=false,
   numbers=left,
   numberstyle=\footnotesize,
   numbersep=9pt,
   tabsize=2,
   breaklines=true,
   showtabs=false,
   captionpos=b
}

\begin{document}
\begin{enumerate}
\item [10.3] Ermittle sämtliche Zahlen $z \in \C$, die den folgenden Gleichungen genügen:\\
\begin{enumerate}
	\item $z^4 = -16$\\
	\begin{align*}
		|z^4| &= \sqrt{(-16)^2} = 16\\
		\varphi &= \pi\\
		z^4 &= 16 e^{i\pi}\\
		z_k &= 2 e^{i(\frac{\pi + 2\pi k}{4})}\\\\
		\Rightarrow
		z_0 &= 2 e^{i(\frac{\pi}{4})}\\
		z_1 &= 2 e^{i(\frac{3\pi}{4})}\\
		z_2 &= 2 e^{i(\frac{5\pi}{4})}\\
		z_3 &= 2 e^{i(\frac{8\pi}{4})}\\
		\mathbb{L}&= \{2 e^{i(\frac{\pi}{4})},2 e^{i(\frac{3\pi}{4})},2 e^{-i(\frac{\pi}{4})},2 e^{-i(\frac{3\pi}{4})}\}
	\end{align*}

	\item $z^3 + 3iz^2 - 3z - 9i = 0$ (Hinweis: kubische Ergänzung)\\
	\begin{align*}
		\eqnf{z^3 + 3iz^2 - 3z - 9i}{0}
		\eqn{(z+3i)(z^2-3)}{0}
		\Rightarrow z=-3i,z^2 = 3\\
		\mathbb{bb}=\{-3i,\sqrt{3},-\sqrt{3}\}
	\end{align*}
	\newpage
	\item $z^2-3z+3-i=0$
	\begin{align*}
		\eqnf{z^2-3z+3-i}{0}
		\eqni{(z-1.5)^2-1.5^2+3-i}{0}{-\frac{3}{4}+i}
		\eqni{(z-1.5)^2}{-\frac{3}{4}+i}{\sqrt{}}
		\eqni{z-1.5}{\pm\sqrt{-\frac{3}{4}+i}}{+1.5}
		\eqn{z}{\pm\sqrt{-\frac{3}{4}+i}+1.5}
	\end{align*}
	\begin{align*}
		\eqnf{(x+iy)^2}{a+ib}
		\eqn{x+iy}{\sqrt{a+ib}}
	\end{align*}
	$\Rightarrow a=-\frac{3}{4}, b=1$
	\begin{align*}
		\eqnf{x^2+y^2}{\sqrt{a^2+b^2}}
		\eqnf{x^2-y^2}{a}
		\Rightarrow\eqnf{x}{\sqrt{\frac{a+\sqrt{a^2+b^2}}{2}}}
		\Rightarrow\eqnf{y}{\sqrt{\frac{-a+\sqrt{a^2+b^2}}{2}}}
	\end{align*}
	\begin{align*}
		x &= \sqrt{\frac{-\frac{3}{4}+\sqrt{\frac{9}{16}+1}}{2}} = \sqrt{\frac{-\frac{3}{4}+\sqrt{\frac{25}{16}}}{2}} = \sqrt{\frac{-\frac{3}{4}+\frac{5}{4}}{2}} = \sqrt{\frac{\frac{2}{4}}{2}} = \sqrt{\frac{1}{4}} = \frac{1}{2}\\
		y &= \sqrt{\frac{\frac{3}{4}+\sqrt{\frac{9}{16}+1}}{2}} = \sqrt{\frac{\frac{3}{4}+\sqrt{\frac{25}{16}}}{2}} = \sqrt{\frac{\frac{3}{4}+\frac{5}{4}}{2}} = \sqrt{\frac{\frac{8}{4}}{2}} = 1
	\end{align*}\\
	$\Rightarrow z=\pm(0.5+i)+1.5$\\
	$\Rightarrow z_1=2+i, z_2=1-i$
\end{enumerate}

\newpage
\item [10.4.] Sei $a > 0$, $ a \neq 1$. Berechne ohne Taschenrechner / Computer:\\

\begin{align*}
	\log_2 (0.125)\\
	0.125=\frac{1}{8}=8^{-1}=(2^3)^{-1}=2^{-3}\\
	\Rightarrow \log_2 (0.125) = -3
\end{align*}\\
\begin{align*}
	\log_{a^2}(\sqrt{a})\\
	\sqrt{a}=\sqrt{\sqrt{a^2}}=\sqrt[4]{a^2}=(a^2)^\frac{1}{4}\\
	\Rightarrow \log_{a^2}(\sqrt{a}) = \frac{1}{4}
\end{align*}\\
\begin{align*}
	\log_2\sqrt[3]{32}\\
	\sqrt[3]{32}=\sqrt[3]{2^5}=(2^5)^\frac{1}{3}=2^\frac{5}{3}\\
	\log_2(\sqrt[3]{32}) = \frac{5}{3}	
\end{align*}\\
\begin{align*}
	3^{\log_9 (4)}\\
	3^{\log_9 (4)} = \sqrt{(3^{\log_9 (4)})^2} = \sqrt{3^{2\log_9 (4)}} = \sqrt{9^{\log_9 (4)}} = \sqrt{4} = 2
\end{align*}

\item [10.5.]
\begin{enumerate}
	\item Sei $f: [0,1] \to\R$ stetig, mit $f(0)=f(1)$. Dann gibt es ein $t\in[0,^\frac{1}{2}]$ mit $f(t)=f(t+\frac{1}{2})$.\\\\
	Aus $f(0)=f(1)$ und $f(t)=f(t+\frac{1}{2})$ mit $t\in[0,^\frac{1}{2}]$ folgt $f(0) = f(\frac{1}{2}) = f(1)$. Wenn $f:[0,1]\to\R$ stetig ist, müssen auch $f_1:[0,\frac{1}{2}]\to\R$=$f_2:[\frac{1}{2},1]\to\R$ stetig sein.\\\\
	Wenn $g(x)=f_1(2x):[0,1]\to\R$ stetig, mit $g(0)=g(1)\Leftrightarrow f(0)=f(\frac{1}{2})$, müsste es ein $t\in[0,\frac{1}{2}]$ mit $g(t)=g(t+\frac{1}{2})$ geben.\\\\
	Selben Schritt erneut anwenden (Iteration):\\
	$\Rightarrow f,f_1,g,\cdots$ konstant\\
	$\Rightarrow f\not=[0,1]\to\R$. Nicht jede stetige Funktion mit $[0,1]\to\R$ ist konstant.
	$\Rightarrow$ Widerspruch
	
	\item Die Gleichung $x^6-x^5+42x^3-5=0$ hat mindestens zwei reelle Lösungen.\\
	\begin{lstlisting}
function f(x){
    return Math.pow(x,6)-Math.pow(x,5)+42*Math.pow(x,3)-5;
}	
	
for(let i = -20; i <= 20; i++){
    console.log(i+ " %c "+f(i), f(i)<0?'background: #222; color: #bada55':'');
}
	\end{lstlisting}
	Aus $f(0)<0$ und $f(1)>0$ folgt $x_0\in[0,1]$.\\
	Aus $f(-3)<0$ und $f(-4)>0$ folgt $x_2\in[-3,-4]$.\\
	$\Rightarrow f$ besitzt mindestens 2 Nullstellen.
\end{enumerate}
\item [10.6.] Die Hyperbelfunktionen oder hyperbolischen Funktionen Cosinus hyperbolicus und
Sinus hyperbolicus sind definiert durch$$cosh(x):=\frac{1}{2}(e^x+e^{-x}) \text{ und } sinh(x):=\frac{1}{2}(e^x-e^{-x})(x\in\R)$$\\
\begin{enumerate}
	\item Bestätige die Formel $$cosh^2(x) - sinh^2(x) = 1\text{      } x\in\R$$\\
	\begin{align*}
		\eqnf{\frac{1}{4}(e^x+e^{-x})^2-\frac{1}{4}(e^x-e^{-x})^2}{1}
		\eqn{e^{2x}+e^{-2x}+2-(e^{2x}+e^{-2x}-2)}{4}
		\eqn{4}{4}
	\end{align*}
	\item Zeige, dass die Funktion $sinh: \R \to \R$ streng monoton wachsend und bijektiv ist, und
	dass die Umkehrfunktion $arsinh: \R \to \R$ (Areafunktion: Area Sinus hyperbolicus) stetig
	ist und durch
	$arsinh(x) = \ln (x + \sqrt{x^2+1})$
	gegeben ist.\\
	
	\begin{align*}
		\sqrt{x^2+1} > x\\
		\ln(x2) \text{ ist streng monoton wachsend}\\
		\Rightarrow \ln(\sqrt{x^2+1})\text{ ist streng monoton wachsend}\\
	\end{align*}
		Dadurch ist laut der Stetigkeit der Umkehrfunktion, auch die Umkehrfunktion von arcshin(x), also sinh(x), stetig und streng monoton wachsend.\\
		Da sinh(x) auch auf ein die Erforderungen für den Satz über die Stetigkeit der Umkehrfunktion erfüllt, wissen wir wiederum das arcsin(x) stetig ist und sinh(x) bijektiv\\
	
\end{enumerate}


\end{enumerate}
\end{document}