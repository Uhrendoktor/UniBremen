\documentclass{article}
\usepackage{esvect, fullpage, fancyhdr, amsfonts, amsmath, amssymb, mathtools, polynom}

\begin{document}
\section{Coloumbsches Gesetz}
\begin{enumerate}
	\item Experimentell ermittelt
	\item Eine elektrische Ladung $Q_2$ übt auf jede andere Ladung $Q_1$ eine Kraft $\vec{F}$
	\begin{enumerate}
		\item proportional zu den Ladungen $Q_1$ und $Q_2$ aus
		\item umgekehrt proportional $r^2$ aus, wenn $r$ der Abstand zwischen den beiden Ladungen ist.
		\item aus, die die Richtung der Verbindungslinie zwischen $Q_1$ und $Q_2$ hat.
		%grafik
	\end{enumerate}
	\item Ladungen unterschiedlichen Vorzeichens ziehen sich an; Ladungen gleichen Vorzeichens stoßen sich ab.
	$$\Rightarrow \vec{F}_{12} = k\frac{Q_1\cdot Q_2}{r^2}$$
	$\rightarrow$ Faktor $k$ muss materialabhängig sein.
	$$k=\frac{1}{4\pi\epsilon};  \epsilon=\epsilon_0\cdot\epsilon_r$$
	$\epsilon$: Dielektizitätskonstane (Permitivität)
	$$\Rightarrow \vec{F_{12}} = \frac{Q_1\cdot Q_2}{4\pi\epsilon r^2}\vec{e_r}$$
	$_1$: Ort der "Wirkung"\\
	$_2$: Ort der "Ursache"
\end{enumerate}
\section{Übung A.2}
Die Beiträge der zwei abstoßenden Kräfte sind gleich groß, da sowohl die Ladungen, als auch die Abstände gleich sind:
$$\Rightarrow \vec{F_{23}} = \vec{F_{21}} = \frac{Q^2}{4\pi\epsilon_o l^2}$$
\begin{center}
\begin{tabular}{r c c l}
Außerdem: & $\vec{F_{24}}$&$=$&$\frac{Q^2}{4\pi\epsilon_0 (l\sqrt{2})^2} = \frac{Q^2}{e\pi\epsilon_0 2l^2}$\\
mit: & $\vec{F_n}$&$=$&$\vec{F_{21}} = \vec{F_{23}}$
\end{tabular}
\end{center}
\begin{align*}
\Rightarrow F_g =& 2\vec{F_n}\cos 45\deg\\
=& \sqrt{2}\vec{F_n}\\
=& \sqrt{2}\frac{Q^2}{\pi 4 \epsilon_0 l^2}
\end{align*}
$F_g$ ist aufgrund der geometrischen Anordnungen der Ladungen $Q_1$ und $Q_2$ und $Q_3$ diagonal nach Außen gerichtet, da sich die Ladungen abstoßen und gleich groß sind. Die anziehend wirkende Kraft $\vec{F_{24}}$ ist $\vec{F_g}$ gleichgerichtet, aber mit umgekehrtem Vorzeichen.
$$\Rightarrow \vec{F_{ges}}=\vec{F_g}-\vec{F_{24}}=\frac{Q^2}{4\pi\epsilon_0 l^2}(\sqrt{2}-\frac{1}{2})\approx 0,914\frac{Q^2}{4\pi\epsilon_0 l^2}$$
\section{Übung A.1}
Alle 3 Ladungen sind gleich weit vom Mittelpunkt entfernt.\\
\begin{align*}
	a: \text{gegeben; } \Rightarrow b =& \frac{\frac{a}{2}}{\cos 30\deg}=\frac{5\cdot 10^{-2}m}{\cos 30\deg}\approx5,77\cdot 10^{-2}m\\
	\Rightarrow E =& \frac{Q}{4\pi\epsilon_0 b^2}\\
	\Rightarrow E_1 =& \frac{Q_1}{4\pi\epsilon_0 b^2}=\frac{2,5\cdot 10^{-8} AsVm}{4\pi\cdot 8,854\cdot 10^{-12}As\cdot (5,77)^2\cdot 10^{-4} m^2}\approx6.75\cdot 10^4 \frac{V}{m}\\
	E_2 =& \frac{Q_2}{Q_1}E_1 \approx 4,05\cdot 10^4\frac{V}{m}\\
	E_3 =& \frac{Q_3}{Q_1}E_1\approx 5,4\cdot 10^4 \frac{V}{m}
\end{align*}\\
Um den Betrag der resultierenden Feldstärke zu bestimmen, müssen die drei Feldstärken in ihre Komponenten zerlegt werden:\\
\begin{align*}
E_{1,x} &= E_1\cos 30\deg \approx 5,85\cdot 10^4 \frac{V}{m}\\
E_{1,y} &= E_1\cos 60\deg \approx 3,375\cdot 10^4 \frac{V}{m}\\
E_{2,x} &= 0\\
E_{2,y} &= -E_2 = -4,05\cdot 10^4\frac{V}{m}\\
E_{3,x} &= E_3\cos 30\deg \approx 4,676\cdot 10^4\frac{V}{m}\\
E_{3,y} &= -E_3\cos 60\deg \approx -2,7\cdot 10^4\frac{V}{m}\\
\Rightarrow \vec{E} &= \vec{E_1}+\vec{E_2}+\vec{E_3} \text{ (Überlagerung)}\\
\Rightarrow E&\coloneqq \left[\begin{pmatrix}
5,85\\3,375
\end{pmatrix}+\begin{pmatrix}
0\\-4,05
\end{pmatrix}+\begin{pmatrix}
4,676\\-2,7
\end{pmatrix}\right]\cdot 10^4\frac{V}{m}\\
=\begin{pmatrix}
10,52\\-3,375
\end{pmatrix}\cdot 10^4\frac{V}{m} \Rightarrow |\vec{E}| = \sqrt{(10,52)^2+(-3,375)^2}\cdot 10^4\frac{V}{m}\\
\approx 11,05\cdot 10^4\frac{V}{m}\\
\cos\alpha &= \frac{E_x}{|\vec{E}|} = \frac{10,52}{11,05} = 0,952 \Rightarrow \alpha\approx 18\deg\\
\end{align*}
\section{Übung A.3}
3 seperate Kapazitäten mit $\mu F$
\begin{enumerate}
\item 1Kap. in Reihe $C_{ges} = 1\mu F$
\item 2Kap. in Reihe $C_{ges} = 0,5|mu F$
\item 3Kap. in Reihe $C_{ges} = \frac{1}{3}\mu F$
\item 2Kap. parallel $C_{ges} = 2\mu F$
\item 3Kap. parallel $C_{ges} = 3\mu F$
\item 2Kap in Reihe parallel zu 1 Kap. $C_{ges}=(1+\frac{1}{2})\mu F=\frac{3}{2}\mu F$
\item 2Kap parallel in Reihe zu 1Kap. $C_{ges}= \frac{1}{(1+\frac{1}{2})\mu F}=\frac{2}{3}\mu F$
\end{enumerate}
\end{document}