\documentclass{HM}
\begin{document}
\begin{enumerate}
\item[7.3] Untersuche  folgende rekursiv definierte Folgen auf Beschränktheit, Monotonie und Konvergenz. berechne gegebenenfalls den Grenzwert.
\begin{enumerate}
\item $a_1\coloneqq 1, a_{n+1}\coloneqq \frac{1}{4}a_n - \frac{1}{2}$ für $n\in\N$.\\
Durch betrachtung der Folgenglieder:
$$\{1,-0.25,-0.5625,-0.640625,\dots\}$$
scheint es als würde diese Folge monoton fallen und bei $\frac{-2}{3}$ unten beschränkt sein.\\\\
Durch betrachtung der Folgendefinition fällt auf das die Folge bei positiven und leicht negativen $a_n$ immer fällt aber bei größeren negativen $a_n$ steigt, der Wert bei dem sich das verhalten ändert ist ungefär $\frac{-2}{3}$.\\

Damit ist die letze Möglichkeit für diese Folge zu divergieren wenn sie in eine stabile oszilation verfällt.\\
Aber da $\frac{1}{4}a_{n+1}-\frac{1}{2}=a_n \lightning$\\
muss a konvergent sein.
\begin{align*}
	\Rightarrow a_n \to a, a_{n+1} \to a\\
	\Rightarrow \eqnfi{\frac{1}{4}a-\frac{1}{2}}{a}{-\frac{1}{4}a}\\
	\eqni{a(1-\frac{1}{4})}{-\frac{1}{2}}{\div (1-\frac{1}{4})}
	\eqn{a}{-\frac{2}{3}}
\end{align*}
Also konvergiert a nach $-\frac{2}{3}$
\item $a_1\coloneqq 2, a_{n+1}\coloneqq \frac{1}{a_n}$ für $n\in\N$.\\
Wenn man sich die Folgenglieder dieser Folge anschaut: $$\{2,\frac{1}{2},2,\frac{1}{2},\dots\}$$
sieht man, dass diese Folge zwischen den Weten $2$ und $\frac{1}{2}$ osziliert.\\
Damit is klar das sie divigent ist und die folgenden maximal Werte hat\\
$$\max a = 2$$
$$\min a = \frac{1}{2}$$\\

\end{enumerate}
\item[7.4] Die Folge $(a_n)$ sei rekursiv definiert durch
$$a_1\coloneqq 1, a_{n+1}\coloneqq\sqrt{1+a_n}$$.\\
Ist die Folge konvergent? Bestimme gegebenenfalls den Grenzwert $a$.\\
\item[7.5] Bestimme jeweils die Folge der Partialsummen und den Grenzwert:
\begin{enumerate}
\item $\sum\limits_{k=2}^\infty\frac{1}{k^2-1}$;\\
$$n=[2,\infty);(s_n)_n=\left(\sum_{k=2}^{n}\frac{1}{k^2-1}\right)_n$$
\begin{align*}
	\text{Partialbruchzerlegung}\\
	\eqnf{\frac{1}{k^2-1}}{\frac{A}{k-1}+\frac{B}{k+1}}
	\eqn{1}{A(k+1)+B(k-1)}\\
	k\to1&\Rightarrow A=\frac{1}{2}\\
	k\to-1&\Rightarrow B=-\frac{1}{2}\\
	\Rightarrow \frac{1}{k^2-1}&=\frac{1}{2(k-1)}-\frac{1}{2(k+1)}\\\\	
	\Rightarrow (s_n)_n&=\left(\sum_{k=2}^{n}\frac{1}{2(k-1)}-\frac{1}{2(k+1)}\right)_n\\
	\lim_{n\to\infty}(s_n)_n&=\frac{1}{2}-\frac{1}{6}+\frac{1}{4}-\frac{1}{8}+\frac{1}{6}-\frac{1}{10}+\dots+\frac{1}{2(n+1)+2}\\
	\text{Teleskopsumme}\\
	\lim_{n\to\infty}\frac{1}{2(n+1)+2}&=0\\
	\Rightarrow \lim_{n\to\infty}(s_n)_n&=\frac{1}{2}+\frac{1}{4} = \frac{3}{4}\\
	\Rightarrow \sum\limits_{k=2}^\infty\frac{1}{k^2-1} = \frac{3}{4}
\end{align*}
\item $\sum\limits_{k=1}^\infty\frac{(1+i)^k}{2^k}$.\\
$$n=[1,\infty);(s_n)_n=\left(\sum_{k=1}^{n}\frac{(1+i)^k}{2^k}\right)_n$$\\
\begin{align*}
	(s_n)_n&=\left(\sum_{k=0}^{n}\frac{(1+i)^k}{2^k}\right)_n-s_0\\
	z&=\frac{1+i}{2}\\
	|z|&=\sqrt{\frac{1}{4}-\frac{1}{4}}=0\\
	\text{Geometrische Reihe}:\\
	|z| < 1:\sum_{k=0}^{\infty}z^k&=\frac{1}{1-z}\\
	\Rightarrow \lim_{n\to\infty}(s_n)_{n\ge0} &= \frac{2}{1-i}-1\\
	\Leftrightarrow \lim_{n\to\infty}(s_n)_{n\ge0} &= \frac{2+2i}{(1-i)(1+i)}-1\\
	\Leftrightarrow \lim_{n\to\infty}(s_n)_{n\ge0} &= i\\	
	\\
	\Rightarrow \sum\limits_{k=1}^\infty\frac{(1+i)^k}{2^k} &= i\\
\end{align*}
\end{enumerate}
\item[7.6] Untersuche folgende Reihen auf Konvergenz oder Divergenz:
\begin{enumerate}
\item $\sum\limits_{n=1}^\infty\frac{n^2}{2^n}$;\\
Quotientenregel:
\begin{align*}
	&\left|\frac{(n+1)^2}{2^(n+1)}\frac{2^n}{n^2}\right|\\
	\Leftrightarrow &\frac{(n+1)^2}{n^2\cdot 2}\\
	\Leftrightarrow &\left(\frac{n+1}{\sqrt{2}n}\right)^2\\
	\Leftrightarrow &\left(\frac{1}{\sqrt{2}}+\frac{1}{\sqrt{2}n}\right)^2\\
	\Leftrightarrow &\frac{1}{2} + \frac{1}{2n^2} + \frac{1}{n}\\
	\Rightarrow &\lim_{n\to\infty} = \frac{1}{2}\\
\end{align*}
$$\sum\limits_{n=1}^\infty\frac{n^2}{2^n} < \infty$$\\

\item $\sum\limits_{n=1}^\infty(1+\frac{1}{n^2})^n$;\\

\item $\sum\limits_{n=1}^\infty\frac{1}{7^n}\begin{pmatrix}
3n\\
n
\end{pmatrix}$;\\
\begin{align*}
	\sum\limits_{n=1}^\infty\frac{1}{7^n}\begin{pmatrix}
		3n\\
		n
	\end{pmatrix} = \sum\limits_{n=1}^\infty\frac{1}{7^n} \frac{(3n)!}{n!(2n)!}\\
	\text{Quotientenregel:}\\
	\frac{(3n+1)!}{7^{n+1}(n+1)!(2n+2)!}\frac{7^nn!(2n)!}{(3n)!}\\
	\Leftrightarrow \frac{(3n+3)(3n+2)(3n+1)n!(2n)!(3n)!}{7(n+1)(2n+2)(2n+1)n!(2n)!(3n)!}\\
	\frac{(3n+3)(3n+2)(3n+1)}{7(n+1)(2n+2)(2n+1)}\\
	\end{align*}
\item $\sum\limits_{n=1}^\infty\frac{(-1)^n}{\sqrt{n}}$\\
Leipnitz-Kriterium für alternierende Reihen:\\
da $\lim_{n\to\infty} \frac{1}{\sqrt{n}} = 0$ (momoton Fallend) ist $$\sum\limits_{n=1}^\infty\frac{(-1)^n}{\sqrt{n}}\text{ ist kovergent}$$ 

\end{enumerate}

\end{enumerate}
\end{document}