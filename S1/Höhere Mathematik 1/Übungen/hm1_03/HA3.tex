\documentclass{HM}

\begin{document}
\begin{enumerate}
\item [3.3]
\begin{enumerate}
\item Bestimme die Polardarstellungen von $\frac{i\sqrt{3} - 1}{2}$ und von $\left(\frac{1-i\sqrt{3}}{1+i}\right)^5$.
\item Bestimme alle Lösungen der Gleichung $z^3 = -8$.
\end{enumerate}
\item [3.4] Berechne sämtliche möglichen Produkte aus den gegebenen Matrizen
$$
A=\begin{pmatrix}
1 & 2 & 3\\
2 & 0 & 1
\end{pmatrix},
B=\begin{pmatrix}
0 & 1\\
2 & -1
\end{pmatrix},
C=\begin{pmatrix}
-2 & t\\
1 & 0\\
1 & -1
\end{pmatrix},
D=\begin{pmatrix}
1 & 2 & 1\\
-1 & 1 & 0
\end{pmatrix},
$$
wobei $t$ ein reeler Parameter ist.
\item [2.5] Berechne
$$\begin{matrix}
1 & 1\\
0 & 1
\end{matrix}^n$$
für n = 1,2,3,4, stelle eine Vermutung für eine Formel für allgemeines $n\in\N$ und beweise diese Formel durch Induktion.
\item [3.6] Die \textit{Pauli-Matrizen} sind definiert durch
$$\sigma_1 \coloneqq \begin{matrix}
0 & 1\\
1 & 0
\end{matrix},
\sigma_2 \coloneqq \begin{matrix}
0 & -i\\
i & 0
\end{matrix},
\sigma_1 \coloneqq \begin{matrix}
1 & 0\\
0 & -1
\end{matrix}
$$
mit der imaginären Einheit $i$. zeige für alle $j,k = 1,2,3:$\\
$$\sigma_j\sigma_k = \delta_{jk}E_2+i\sum_{l=1}^{3}\epsilon_{jkl}\sigma_l,$$
wobei $\delta_{jk}$ das Kronecker-Delta ist und
$$\epsilon_{jkl}\begin{cases}
0 &\text{falls mindestens 2 der Inizes }j,k,l \text{denselben Wert haben,}\\
1 &\text{falls} (j,k,l)\in\lbrace(1,2,3),(2,3,1),(3,1,2)\rbrace,\\
-1 &\text{falls} (j,k,l)\in\lbrace(3,2,1),(2,1,3),(1,3,2)\rbrace,
\end{cases}$$
das \textit{Levi-Civita-Symbol}.
\end{enumerate}
\end{document}